\documentclass[12pt]{book}

\usepackage{amsthm, amsmath, amsfonts, amssymb}

\usepackage{yhmath}



%% PREAMBLE

%% Page formatting
%\textwidth=6.5in
%\textheight=7.5in
%\headheight=0.75in
%\headsep=0.25in
%\parindent=0.2in
%\topmargin=0.0in
%\oddsidemargin=0.0in
%\evensidemargin=0.0in

%% Environments
\newcounter{exc}
\numberwithin{exc}{section}
\numberwithin{figure}{section}
\newenvironment{exer}{\vspace{0.1in}
\noindent \textbf{Problems for Section~\thesection} \vspace{0.1in}
\begin{list}{\arabic{exc}.}{\usecounter{exc}}}{\end{list}}

%% Theorem-like environments
\newtheorem{theorem}{Theorem}[section]
\newtheorem{fact}[theorem]{Fact}
\newtheorem{corr}[theorem]{Corollary}
\numberwithin{equation}{theorem}

%% Abbreviations
\def\beqn{\begin{equation}}
\def\eeqn{\end{equation}}
\def\beqa{\begin{eqnarray*}}
\def\eeqa{\end{eqnarray*}}
\def\bthm{\begin{theorem}}
\def\ethm{\end{theorem}}
\def\bpf{\begin{proof}}
\def\epf{\end{proof}}
\def\ii{\item}
\def\lpar{\left(}
\def\rpar{\right)}
\def\ang{\angle}
\def\sang{\sin \angle}
\def\half{\frac{1}{2}}

\def\vA{\vec{A}}
\def\vB{\vec{B}}
\def\vC{\vec{C}}
\def\CC{\mathbb{C}}
\newcommand{\RR}{\mathbb{R}}
\def\norm#1{\lVert #1 \rVert}

\def\dang{\measuredangle} %% Directed angle
\def\line#1{\overleftrightarrow{#1}}
\def\ray#1{\overrightarrow{#1}} 
\def\seg#1{\overline{#1}}
\def\arc#1{\wideparen{#1}}

\def\fixme#1{\textbf{FIXME! (#1)}}

%% END PREAMBLE

\makeindex

\begin{document}

%% Begin with Roman numerals
\pagenumbering{roman}

%% Title page
\begin{titlepage}
\begin{center}
\textbf{\large Geometry Unbound}

\vspace{1in}
Kiran S. Kedlaya

version of 18 Jan 2006
\end{center}

\vspace{1in}

\noindent
\copyright 2006 Kiran S. Kedlaya. 
Permission is granted to copy, distribute and/or modify this document
under the terms of the GNU Free Documentation License, Version 1.2
or any later version published by the Free Software Foundation;
with no Invariant Sections, no Front-Cover Texts, and no Back-Cover Texts.
Please consult the section of the Introduction
entitled ``License information'' for further details.

\medskip
\noindent
Disclaimer: it is the author's belief that all use of quoted material, such as
statements of competition problems, is in compliance with the ``fair use''
doctrine of US copyright law. However, no guarantee is made or implied that
the fair use doctrine will apply to all derivative works, or in the copyright
law of other countries.


\end{titlepage}
\pagebreak

%% Contents
\tableofcontents
\newpage

%% Preface
%\setcounter{chapter}{-1}
\setcounter{secnumdepth}{-1}
\chapter{Introduction}

\section{Origins, goals, and outcome}

The original text underlying this book was a set of notes\footnote{The
original notes have been circulating on the Internet since 1999, under
the pedestrian title ``Notes on Euclidean Geometry''.} I compiled,
originally as a participant and later as an instructor, for the Math
Olympiad Program (MOP),\footnote{The program has actually been called the
Math Olympiad Summer Program (MOSP) since 1996, but in accordance to common custom, we refer to the original acronym.} \index{MOP (Math Olympiad Program)}
\index{MOSP (Math Olympiad Summer Program)}
the annual summer program to prepare U.S. 
high school students for the International Mathematical Olympiad 
(IMO).\index{IMO (International Mathematical Olympiad)}
Given the overt mission of the MOP, the notes as originally compiled
were intended to bridge the gap between the knowledge of Euclidean geometry
of American IMO prospects and that of their counterparts from other
countries. To that end, they included a large number of challenging
problems culled from Olympiad-level competitions from around the world.

However, the resulting book you are now reading shares with the MOP
a second mission, which is more covert and even a bit subversive.
In revising it, I have attempted to
usher the reader from the comfortable world
of Euclidean geometry to the gates of ``geometry'' as the term is defined
(in multiple ways)
by modern mathematicians, using the solving of routine and nonroutine
problems as the vehicle for discovery. In particular, I have aimed to
deliver something more than ``just another problems book''.

In the end, I became unconvinced that I would succeed in this mission
through my own efforts alone; as a result, the manuscript remains in some
ways unfinished. For one, it still does not include figures (though
some of these do exist online; see the chapter ``About the license'');
for another,
I would ideally like to include some additional material in Part III
(examples: combinatorial geometry, constructibility).

Rather than continue endlessly to ``finish'' the manuscript,
I have instead decided to carry the spirit of the distribution of the
notes to a new level, by deliberately releasing an incomplete manuscript
as an ``open source'' document using the
GNU Free Documentation License; 
\index{GNU Free Documentation License} (for more on which see the 
chapter ``About the license'').
My hope is that this will encourage readers to make use of this still
unpolished material
in ways I have not foreseen.

\section{Methodology}

This book is not written in the manner of a typical textbook.
(Indeed, it is not really designed to serve as a textbook at all,
though it could certainly be used as one with highly motivated students.)
That is, we do not
present full developments of key theorems up front, leaving
only routine exercises for the reader to consider. For one, we leave
strategic gaps in the exposition for the reader to fill in. For another,
we include a number of nonroutine problems, of the sort found on the
IMO or related national competitions. The reader may or may not succeed
in solving these, but attempting them should provide a solid test of
one's understanding. In any case, solutions to the exercises
and problems are included in the back; we have kept these brief, and they
are only intended to make sense once you have already thought a bit about the
corresponding exercises/problems on your own.

In addition to the MOP (and in some sense the Socratic method), \index{Socratic method}
inspirations for this approach include the
famous Moore method \index{Moore method} of learning through problems, 
and the number theory curriculum of the late Arnold Ross's 
\index{Ross, Arnold} renowned
summer mathematics program\footnote{Arnold Ross 
may no longer be with us, but 
fortunately his program is: its web site is
\texttt{http://www.math.ohio-state.edu/ross/}.} for high school students.
We also take inspiration from the slender classic 
\emph{Geometry Revisited} by H.S.M. Coxeter \index{Coxeter, Harold Scott MacDonald}
and S. Greitzer, \index{Greitzer, Samuel}
among whose pages this author discovered the beauty of Euclidean
geometry so carefully hidden by many textbook writers. Indeed,
we originally considered titling this book \textit{``Geometry Revisited''
Revisited} in homage to the masters; we ultimately chose instead to
follow Aeschylus \index{Aeschylus}
and Percy Bysshe Shelley \index{Shelley, Percy Bysshe}
in depicting geometry as a titanic subject
released from the shackles of school curricula.

\section{Structure of the book}

Aside from this introduction, the book is divided into four parts.
The first part, ``Rudiments'', is devoted to the foundations of Euclidean geometry
and to some of the most pervasive ideas within the subject.
The second part, ``Special situations'', treats some common environments of
classical synthetic geometry; it is here where one encounters many of
the challenging Olympiad problems which helped inspire this book.
The third part, ``The roads to modern geometry'', consists of two\footnote{We
would like to have additional such chapters, perhaps in a subsequent
edition of the book, perhaps in a derivative version.} chapters
which treat slightly more advanced topics (inversive and projective geometry).
The fourth part, ``Odds and ends'', 
is the back matter of the book, to be consulted
as the need arises; it includes hints for the exercises
and problems (for more on the difference, see below),
plus bibliographic references, suggestions for further reading, 
information about the open source license, and an index.

Some words about terminology \index{terminology}
are in order at this point. 
For the purposes of this book, a \emph{theorem} 
\index{theorem|textbf} is an important result 
which either is given with its proof, or is given without its proof
because inclusion of a proof would lead too far afield. In the 
latter case, a reference is provided. 
A \emph{corollary} \index{corollary|textbf} is a result which is 
important in its own right, but is easily deduced from a nearby theorem.
A \emph{fact} \index{fact|textbf} is a result which is
important but easy enough to deduce that its proof is left to the reader.


Most sections of the text are accompanied by a section labeled
``Problems'', which are additional assertions which the reader
is challenged to verify. Some of these are actually what we would call
\emph{exercises}, \index{exercise|textbf} i.e., results which the reader 
should not have
any trouble proving on his/her own, given what has come before.
By contrast, a true \emph{problem} \index{problem|textbf} is a result that 
can be obtained using
the available tools, but which also requires some additional insight.
In part to avoid deterring the reader from trying the more challenging 
problems (but also to forestall some 
awkwardness in cross-referencing), we have used 
the term ``problem'' in both cases.
Hints have been included in the back matter
of the book for selected problems; in order that the hints
may also cover facts,
some problems take the form ``Prove Fact 21.13.''
In order to keep the book to a manageable size, and also to avoid challenging
the reader's willpower, solutions have not been included; they may be
instead found online at 

I have attributed my source for each problem to the best of my 
knowledge. Problems from the USA Mathematical Olympiad (USAMO),
International Mathematical Olympiad (IMO), USA Team Selection Test
(TST), and William Lowell Putnam
competition (Putnam) are listed by year and
number; problems from other national or regional contests
are listed by country/contest
and year. 
Problems I obtained from MOP are so labeled when I was unable to determine
their true origins; most of these probably come from national contests.
\emph{Arbelos}\footnote{For the origin of the name ``arbelos'', see
Section~\ref{sec:inv magic}.}
refers to Samuel Greitzer's student publication from 1982--1987
\cite{bib:greitzer},
and \emph{Monthly} refers to the \textit{American Mathematical Monthly}.
Problems listed as ``Original'' are my own problems which
have not before appeared in print (excluding prior versions of this
book). Attributions to other people or web sites should be self-explanatory.

\section{Acknowledgments}

The acknowledgments for a book such as this cannot help but be at once
tediously voluminuous and hopelessly inadequate. 
That being so, there is nothing to done other than to proceed forthwith.

Let me start with those most directly involved. Thanks to Reid 
Barton \index{Barton, Reid} for assembling a partial set of solutions to
the included problems. Thanks to Marcelo Alvisio \index{Alvisio, Marcelo}
for expanding this solution set, for reporting numerous typos in the
1999 manuscript, and for rendering the missing diagrams from the 1999
manuscript using \textit{The Geometer's 
Sketchpad\textsuperscript{\textregistered}}.
Thanks to Arthur Baragar \index{Baragar, Arthur}
for helpful (though not yet carried out)
advice concerning the rendering of diagrams.

Let me next turn to those whose contributions are more diffuse.
I first learned Euclidean geometry in the manner of this book
from my instructors and later colleagues at the MOP, including
Titu Andreescu, \index{Andreescu, Titu} R\u{a}zvan Gelca,
\index{Gelca, R\u{a}zvan}
Anne Hudson, \index{Hudson, Anne}
Gregg Patruno, \index{Patruno, Gregg}
and Dan Ullman. \index{Ullman, Dan}
The participants
of the 1997, 1998, and 1999 MOPs also deserve thanks for working through
the notes that formed the basis for this book.

I owe a tremendous expository debt to Bjorn Poonen \index{Poonen, Bjorn}
and Ravi Vakil, \index{Vakil, Ravi}
my collaborators on the 1985-2000 Putnam compilation
\cite{bib:putnam}. In that volume,
we embarked on a grand experiment: to
forge a strong expository link between challenging ``elementary''
problems and ``deep'' mathematics. The warm reception received by that
volume has emboldened me to apply to the present book
some of what we learned from this experiment.

Thanks to the compilers of the wonderfully comprehensive
MacTutor History of Mathematics, \index{MacTutor History of Mathematics} available 
online at
\begin{center}
\texttt{http://www-gap.dcs.st-and.ac.uk/\~{}history/index.html}.
\end{center}
We have used MacTutor as our reference for historical comments, English spellings of names, and birth and death dates. (All dates are A.D. 
unless denoted B.C.E.\footnote{The latter stands for ``Before the Common
Era'', while the former might be puckishly deciphered as
``Arbitrary Demarcation''.}) \index{B.C.E.}

\newpage
\pagenumbering{arabic}
\part{Rudiments}

\setcounter{secnumdepth}{1}
\chapter{Construction of the Euclidean plane}

The traditional axiomatic development of Euclidean geometry originates
with the treatment by Euclid of Alexandria
(325?--265? B.C.E.) \index{Euclid of Alexandria} in the classic
\textit{Elements}, \index{Elements@\textit{Elements} (of Euclid)}
\index{Euclid's \textit{Elements}}
and was modernized
by David Hilbert (1862--1943) \index{Hilbert, David}
in his 1899 \textit{Grundlagen der Geometrie (Foundations
of Geometry)}. For the purposes of this book, however, it is more
convenient to start with the point of view of a coordinate plane, as
introduced by Ren\'e Descartes\footnote{This attribution explains the term ``Cartesian coordinates'' \index{Cartesian coordinates} to refer to this type of geometric description.} (1596--1690) \index{Descartes, Ren\'e}
and Pierre de Fermat (1601--1665). \index{Fermat, Pierre de} 
We will return to the axiomatic
point of view in due course, when we discuss hyperbolic geometry in 
Chapter~\ref{chap:inversion}; however, the coordinate-based point of view will
also recur when we dabble briefly in algebraic geometry (see Section~\ref{sec:alggeo}).

Of course one must assume \emph{something} in order to get started.
What we are assuming are the basic properties of the real numbers, which
should not be too much of an imposition. 
The subtlest of these properties is the
\emph{least upper bound property}: \index{least upper bound
property (of real numbers)|textbf} every set of real numbers which is
bounded above has a least upper bound. More precisely, if $S$ is a set
of real numbers and there exists a real number $x$ such that
$x \geq y$ for all $y \in S$, then there is a (unique) real number $z$
such that:
\begin{enumerate}
\item[(a)] $z \geq y$ for all $y \in S$;
\item[(b)] if $x \geq y$ for all $y \in S$, then $x \geq z$.
\end{enumerate}

One theme we carry through our definitions is that certain numerical
quantities (lengths of segments along a line, areas, arc and angle
measures) should be treated with special algebraic rules, including
systematic sign conventions. In so doing, one can make some statements
more uniform, by eliminating some dependencies on the relative positions
of points. This uniformity was unavailable to Euclid in the absence
of negative numbers, hampering efforts to maintain logical consistency;
see Section~\ref{sec:angles} for a tricky example.

In any case, while the strictures
of logic dictate that this chapter must occur first, the reader need not
be so restricted. We recommend skipping this chapter on first reading
and coming back a bit later, once one has a bit of a feel for what
is going on. But do make sure to come back at some point: for mathematicians (such as the reader and the author)
to communicate, it is always of the utmost importance to agree on the
precise definitions of even the simplest of terms.\footnote{As mathematician/storyteller Lewis Carroll, \index{Carroll, Lewis (Charles Lutwidge Dodgson)} \textit{n\'e} Charles Lutwidge Dodgson, \index{Dodgson, Charles Lutwidge (Lewis Carroll)} put it in the voice of \emph{Through the Looking Glass} character Humpty Dumpty: ``When I use a word... it means exactly what I choose it to
mean---neither more nor less.''}

\section{The coordinate plane, points and lines}

We start by using Cartesian coordinates to define the basic geometric
concepts: points, lines, and so on.
The reader should not think his/her intelligence is being insulted by
our taking space to do this: it is common in a mathematical text to begin
by defining very simple objects, if for no other reason than to make sure
the author and reader agree on the precise meaning and usage of fundamental
words, as well as on the notation to be used to symbolize them (see previous footnote).

For our purposes, the \emph{plane} \index{plane|textbf}
$\RR^2$ is the set of ordered pairs $(x,y)$ of real numbers; 
we call those pairs
the \emph{points} \index{point|textbf} of the plane. A \emph{line}
\index{line|textbf} (or to be more precise, a \emph{straight line}\footnote{The
term ``straight line'' may be redundant in English, but it is not so in
other languages. For instance, the Russian term for a curve
literally translates as ``curved line''.})
will be any subset of the plane of the form
\[
\{(x,y) \in \RR^2: ax + by + c = 0\}
\]
for some real numbers $a,b,c$ with $a$ and $b$ not both zero. Then as
one expects, any two distinct points $P_1 = (x_1, y_1)$ and $P_2 = (x_2, y_2)$
lie on a unique line, namely
\[
\line{P_1P_2} = \{ ((1-t)x_1 + tx_2, (1-t)y_1 + ty_2): t \in \RR \}.
\]
Similarly, we define the
\emph{ray} \index{ray|textbf}
\[
\ray{P_1P_2} = \{ ((1-t)x_1 + tx_2, (1-t)y_1 + ty_2): t \in [0,\infty) \}
\]
and the \emph{segment} (or \emph{line segment})
\index{line!segment|textbf} \index{segment|textbf}
\[
\seg{P_1P_2} = \{ ((1-t)x_1 + tx_2, (1-t)y_1 + ty_2): t \in [0,1] \}.
\]
Any segment lies on a unique line, called the \emph{extension}
\index{extension (of a segment)|textbf} of the segment.
We say points $P_1, \dots, P_n$ are \emph{collinear} \index{collinearity 
(of points)|textbf}
if they lie on a single line; if $\ell$ is that line, 
we say that $P_1, \dots, P_n$ lie on $\ell$ \emph{in order}
\index{order (of points on a line)|textbf} if for any distinct
$i,j,k \in \{1, \dots,n\}$ with $i<j$, we have $i<k<j$ if and only if
$P_k$ lies on the segment $P_iP_j$. For $n=3$, we also articulate this
by saying that $P_2$ lies \emph{between} \index{betweenness (of points on a line)|textbf} $P_1$ and $P_3$.
We say lines $\ell_1, \dots, \ell_n$ are
\emph{concurrent} \index{concurrence!of lines|textbf} if they contain (or
``pass through'') a single point.

We will postpone defining angles for the moment, but we may as well define
parallels and perpendiculars now. We say two lines $ax+by+c=0$ and $dx+ey+f=0$ are
\emph{parallel} \index{parallelness (of lines)|textbf} if $ae-bd = 0$, and
\emph{perpendicular} \index{perpendicularity (of lines)|textbf}
if $ae+bd = 0$. Then the following facts are easily
verified.
\begin{fact}
Through any given point, there is a unique line parallel/perpendicular to any
given line.
\end{fact}
For $\ell$ a line and $P$ a point, the intersection of $\ell$ with the
perpendicular to $\ell$ through $P$ is called the \emph{foot}
\index{foot of a perpendicular} of the perpendicular through $P$.

\begin{fact}
Given three lines $\ell_1, \ell_2, \ell_3$, the following relations hold.
\begin{center}
\begin{tabular}{ccc}
If $\ell_1$ and $\ell_2$ are: &
and $\ell_2$ and $\ell_3$ are: &
then $\ell_1$ and $\ell_3$ are: \\
parallel & parallel & parallel \\
parallel & perpendicular & perpendicular \\
perpendicular & parallel & perpendicular \\
perpendicular & perpendicular & parallel.
\end{tabular}
\end{center}
\end{fact}

Given a segment $\seg{P_1P_2}$, there is a unique
point $M$ on $\seg{P_1P_2}$ with $P_1M = MP_2$, called the
\emph{midpoint} \index{midpoint!of a segment|textbf} of
$\seg{P_1P_2}$. There is a unique line through $M$ perpendicular
to $\line{P_1P_2}$, called the \emph{perpendicular bisector}
\index{perpendicular bisector (of a segment)|textbf}
of $\seg{P_1P_2}$.

Of course there is nothing special about having only two dimensions; one can
construct an $n$-dimensional Euclidean space for any $n$. In particular, it is not 
unusual to do this for $n=3$, resulting in what we call 
\emph{space geometry}\footnote{The term ``solid geometry'' \index{solid geometry} 
\index{geometry!solid} 
is more common, but less consistent.} \index{space geometry} \index{geometry!space}
as opposed to \emph{plane geometry}.
\index{plane geometry} \index{geometry!plane}
Although we prefer for simplicity not to discuss space geometry, we will make 
occasional reference to it in problems.

\section{Distances and circles}

The
\emph{distance} \index{distance!between two points|textbf}
between two points $P_1 = (x_1, y_1)$ and $P_2 = 
(x_2, y_2)$ is defined
by
\[
P_1P_2 = d(P_1, P_2) = \sqrt{(x_1-x_2)^2 + (y_1 - y_2)^2};
\]
we also describe this quantity as the
\emph{length} \index{length (of a segment)|textbf} of the
segment $\seg{P_1P_2}$. If $\seg{P_1P_2}$ and $\seg{P_3P_4}$ are collinear segments (i.e., the points $P_1,P_2,P_3,P_4$ are collinear), we define the
\emph{signed ratio of lengths}
\index{ratio of lengths, signed (of collinear segments)|textbf} 
\index{signed ratio of lengths (of collinear segments)|textbf}
of the segments $\seg{P_1P_2}$
and $\seg{P_3P_4}$ to be the ratio $P_1P_2/P_3P_4$ if the intersection of 
the rays $\ray{P_1P_2}$ and $\ray{P_3P_4}$ is a ray, and $-P_1P_2/P_3P_4$ otherwise. (In the latter case, the intersection of the two rays may be
a segment, a point, or the empty set.)

From the distance formula, one can verify the \emph{triangle inequality}.
\index{triangle inequality|textbf}
\begin{fact}[Triangle inequality]
Given points $P_1, P_2, P_3$, we have
\[
P_1P_2 + P_2P_3 \geq P_1P_3,
\]
with equality if and only if $P_3$ lies on the segment $\seg{P_1P_2}$.
\end{fact}

We can also define the distance from a point to a line.
\begin{fact}
Let $P$ be a point and $\line{QR}$ a line. Let $S$ be the intersection of 
$\line{QR}$
with the line through $P$ perpendicular to $\line{QR}$. 
Then the minimum distance
from $P$ to any point on $\line{QR}$ is equal to $PS$.
\end{fact}
We call this minimum the \emph{distance} from $P$ to $\line{QR}$, and denote it
$d(P,\line{QR})$.
\index{distance!from a point to a line}

With a notion of distance in hand, we may define
a \emph{circle} \index{circle|textbf} (resp.\ a \emph{disc}
\index{disc|textbf} or \emph{closed disc},
\index{disc!closed|textbf} \index{closed!disc|textbf}
an \emph{open disc}) \index{disc!open|textbf}
\index{open!disc|textbf}
as the set of points $P$ in the plane with the property that 
$OP = r$ (resp.\ $OP \leq r$)
for some point $O$ (the \emph{center}) \index{center!of
a circle/disc|textbf} and
some positive real number $r$ (the \emph{radius}); \index{radius
(of a circle/disc)|textbf} note that both $O$ and $r$ are uniquely determined by the circle. We call the quantity
$2r$ the \emph{diameter} \index{diameter!of a circle/disc|textbf} 
of the circle/disc.
Given a closed disc with center $O$ and radius $r$, we call the circle
of the same center and radius the \emph{boundary} of the disc;
\index{boundary!of a disc|textbf}
we call the open disc of the same center and radius the
\emph{interior} of the circle or of the closed disc.
\index{interior!of a circle/disc|textbf}
\begin{fact}
Any three distinct points which do not lie on a straight line lie on
a unique circle.
\end{fact}

Any segment joining the center of a circle to a point on the circle is
called a \emph{radius}\footnote{This is the first of numerous occasions
on which we use the same word to denote both a segment and its length.
This practice stems from the fact that Euclid did not have an independent
concept of ``length'', and instead viewed segments themselves as
``numbers'' to be manipulated arithmetically.} \index{radius
(of a circle/disc)|textbf} of the circle. 
Any segment joining two points on a circle is
called a \emph{chord} \index{chord (of a circle)|textbf}
of the circle. A chord passing through the center is
called a \emph{diameter}; clearly its length is twice the radius of the 
circle.

\begin{fact} \label{fact:intersect line circle}
Any line and circle intersect at either zero, one, or two points.
Any two distinct circles intersect at either zero, one, or two points.
\end{fact}
A line and circle that meet at exactly one point are said to be
\emph{tangent}. \index{tangent line (of a circle)|textbf}
If $\omega$ is a circle, $A$ is a point on $\omega$, and $\ell$ is a line
through $A$, we will speak frequently of the ``second 
intersection \index{second intersection (of a line and circle)|textbf} of
$\ell$ and $\omega$''; when the two are tangent, we mean this to be $A$ itself.

Two distinct circles which meet in exactly one point are also said to be
\emph{tangent}. In this case, either one circle lies inside the other, in
which case the two are said to be \emph{internally tangent}, 
\index{internal tangency (of circles)|textbf}
or neither circle
contains the other, in which case the two are said to be 
\emph{externally tangent}.
\index{external tangency (of circles)|textbf}

Two or more circles with the same center are said to be \emph{concentric};
\index{concentricity (of circles)|textbf} concentric circles which do not 
coincide also do not intersect.

\begin{fact}
Through any point $P$ on a circle $\omega$, 
there is a unique line tangent to $\omega$:
it is the line perpendicular to the radius of $\omega$ ending at $P$.
\end{fact}

It will be useful later (in the classification of rigid motions; see
Theorem~\ref{thm:classify rigid}) to have in hand the ``triangulation
principle'';  \index{triangulation|principle (of navigation)|textbf} 
this fact was used once upon a time for navigation at sea, and nowadays figures in the satellite-based navigation technology known as the Global Positioning System. \index{Global Positioning System}
\begin{fact} \label{fact:triangulation principle}
Let $A,B,C$ be distinct points. Then any point $P$ in the plane
is uniquely determined by the three distances $PA, PB, PC$; that is, if
$P,Q$ are points in the plane with $PA=QA, PB=QB, PC=QC$, then
$P=Q$. 
\end{fact}

For $n \geq 4$, if $P_1, \dots, P_n$ are distinct points and $\omega$ is a circle,
we say that \emph{$P_1, \dots, P_n$ lie on $\omega$ in that order}
if $P_1, \dots, P_n$ lie on $\omega$ and the polygon
$P_1\cdots P_n$ is simple (hence convex).

\begin{exer}
\ii
Prove Fact~\ref{fact:triangulation principle}.
\ii \label{ex:appower}
Let $\omega_1$ and $\omega_2$ be circles with respective centers
$O_1$ and $O_2$ and respective radii $r_1$ and $r_2$,
and let $k$ be a real number
not equal to 1. Prove that the set of points $P$ such that
\[
PO_1^2 - r_1^2 = k(PO_2^2 - r_2^2)
\]
is a circle. (This statement will be reinterpreted later in terms
of the power of a point \index{power of a point} with respect to a circle;
see Section~\ref{sec:powerofapoint}.)
\ii (IMO 1988/1)
Consider two circles of radii $R$ and $r$ ($R > r$) with the same
center.  Let $P$ be a fixed point on the smaller circle and $B$ a variable
point on the larger circle.  The line $\line{BP}$ 
meets the larger circle again at
$C$.  The perpendicular $\ell$ to $\line{BP}$ at $P$ meets the smaller circle again at
$A$.  (As per our convention, if $\ell$ is tangent to the circle at $P$,
 then we take $A = P$.)
  \begin{enumerate}
  \item[(i)] Find the set of values of $BC^2 + CA^2 + AB^2$.
  \item[(ii)] Find the locus of the midpoint of $\seg{AB}$.
  \end{enumerate}
\end{exer}

\section{Triangles and other polygons}

The word ``polygon'' can mean many slightly different things, depending on
whether one allows self-intersections, repeated vertices, degeneracies,
etc. So one has to be a bit careful when defining it, to make sure that
everyone agrees on what is to be allowed.

Let $P_1,\dots, P_n$ be a sequence of at least three points in the plane.
The \emph{polygon} (or \emph{closed polygon})
\index{polygon|textbf} \index{polygon, closed|textbf}
\index{closed polygon|textbf} with vertices $P_1, \dots, P_n$
is the $(n+1)$-tuple $(P_1, \dots, P_n, U)$,
where $U$ is the
union of the segments
$\overline{P_1P_2}, \dots, \overline{P_{n-1}P_n}, \overline{P_nP_1}$.
We typically refer to this polygon as $P_1\cdots P_n$;
each of the $P_i$ is called 
a \emph{vertex}\footnote{The standard
plural of ``vertex'' is ``vertices'', although ``vertexes'' is also acceptable.
What is not standard and should be avoided 
is the back-formation ``vertice'' as a synonym of ``vertex''.}
\index{vertex!of a polygon|textbf}
of the polygon, and each of the segments making up $U$ is called a
\emph{side}. \index{side (of a polygon)|textbf}

The \emph{perimeter} of a polygon
\index{perimeter (of a polygon)|textbf} is the sum of the lengths
of its underlying segments. It is often convenient to speak of the
\emph{semiperimeter} of a polygon,
\index{semiperimeter (of a polygon)|textbf} which is simply half
of the perimeter.

A polygon is \emph{nondegenerate}
\index{nondegenerate polygon|textbf}
\index{polygon!nondegenerate|textbf}
if no two of its vertices are equal and no vertex lies on a segment 
of the polygon other than the two of which it is an endpoint.
Note that for a nondegenerate polygon, the union of segments
uniquely determines the vertices up to cyclic shift and reversal of
the list.

A polygon 
is \emph{simple} (or \emph{non-self-intersecting})
\index{polygon!simple|textbf} \index{simple polygon|textbf}
\index{polygon!non-self-intersecting|textbf}
\index{non-self-intersecting polygon|textbf}
if it is nondegenerate and no two segments of the polygon intersect
except at a shared endpoint.

For $P_1\cdots P_n$ a polygon, a \emph{diagonal} of $P_1\cdots P_n$
is any segment joining two nonconsecutive vertices.
A simple polygon is \emph{convex} if any two diagonals intersect (possibly
at an endpoint).
\index{polygon!convex|textbf}
\index{convex!polygon|textbf}
\begin{fact}
If the points $P_1, \dots, P_n$ lie on a circle, then the polygon
$P_1\cdots P_n$ is simple if and only if it is convex.
\end{fact}

If $P_1 \cdots P_n$ is a convex polygon, we define the
\emph{interior} 
\index{interior!of a convex polygon|textbf}
of $P_1\cdots P_n$ to be the set of points $Q$ such that
for each $i$, the segment $P_iQ$ intersects the polygon only at its
endpoint $P_i$. 
%(One can define the interior of any simple polygon, but we will postpone
%doing so until we discuss interiors of more general sets in 
%Section~\ref{sec:properties}.)

A nondegenerate polygon of three, four, five or six sides
is called a \emph{triangle}, \index{triangle|textbf}
\emph{quadrilateral}, \index{quadrilateral|textbf}
\emph{pentagon}, \index{pentagon|textbf}
or \emph{hexagon}, \index{hexagon|textbf} respectively.
Since triangles will occur quite often in our discussions, we adopt
some special conventions to deal with them. We will often refer to the triangle
with vertices $A,B,C$ as $\triangle ABC$, and we will list its sides
in the order $\seg{BC}, \seg{CA}, \seg{AB}$. We will often refer to its
side lengths as $a = BC, b = CA, c = AB$.

Let $ABC$ be a triangle. If two of the sides $AB,BC,CA$ have equal lengths, we
say $\triangle ABC$ is \emph{isosceles}; if all three sides have equal lengths, we say $\triangle ABC$ is \emph{equilateral}. If the angles of
$ABC$ are all acute, we say $ABC$ is \emph{acute} \index{acute!triangle|textbf}
\index{triangle!acute|textbf}.

Let $ABCD$ be a convex quadrilateral. If lines $\line{AB}$ and $\line{CD}$
are parallel, we say $ABCD$ is a \emph{trapezoid}. \index{trapezoid|textbf} 
If in addition lines $\line{BC}$ and $\line{DA}$ are parallel, we say
$ABCD$ is a \emph{parallelogram}. \index{parallelogram|textbf}
If in addition $\line{AB} \perp \line{BC}$, we say $ABCD$ is a \emph{rectangle}.
\index{rectangle|textbf}
If in addition $AB=BC=CD=DA$, we say $ABCD$ is a \emph{square}.
\index{square|textbf}

\section{Areas of polygons}

If $P_1\cdots P_n$ is a polygon and $P_i = (x_i, y_i)$, we define the
\emph{directed/signed area}
\index{area!directed (of a polygon)|textbf} 
\index{directed area (of a polygon)|textbf}
\index{area!signed (of a polygon)|textbf} 
\index{signed area (of a polygon)|textbf}
of $P_1\cdots P_n$, denoted $[P_1\cdots P_n]_{\pm}$, by the formula
\[
[P_1\cdots P_n]_{\pm} = \frac{1}{2} (x_1y_2 - x_2y_1 + x_2y_3 - x_3y_2 + \cdots
+ x_{n-1}y_n - x_n y_{n-1} + x_n y_1 - x_1 y_n).
\]
This formula is sometimes called the \emph{surveyor's formula}
\index{formula, surveyor's|textbf}
\index{surveyor's formula|textbf}
or the \emph{shoelace formula};
\index{formula, shoelace|textbf}
\index{shoelace formula|textbf}
the latter name serves as a mnemonic in the following fashion.
If one draws the $2 \times (n+1)$ matrix
\[
\begin{pmatrix} x_1 & x_2 & \cdots & x_n & x_1 \\
y_1 & y_2 & \cdots & y_n & y_1
\end{pmatrix},
\]
the terms of the shoelace formula are obtained by 
multiplying terms along the diagonals and attaching signs as follows:

We define the \emph{area} of the polygon $P_1\cdots P_n$, 
\index{area!of a polygon|textbf}
denoted $[P_1\cdots P_n]$,
to be the absolute value of its directed area.
\begin{fact}
\begin{itemize}
\item
For any polygon $P_1\cdots P_n$,
\[
[P_1\cdots P_n]_{\pm} = [P_2\cdots P_n P_1]_{\pm}
= -[P_n\cdots P_2P_1]_{\pm}.
\]
\item
For any polygons $P_1\cdots P_nXY$ and $YXQ_1\cdots Q_m$,
\[
[P_1\cdots P_nXY]_{\pm} + [YXQ_1\cdots Q_m]_{\pm}
= [P_1\cdots P_nQ_1\cdots Q_m]_{\pm}.
\]
\item
For any triangle $ABC$,
\[
[ABC] = \frac{1}{2} BC \times d(A, \line{BC}).
\]
In particular, $[ABC] \neq 0$.
\item
For any convex quadrilateral $ABCD$, $[ABC]_{\pm}$ and $[ABD]_{\pm}$
have the same (nonzero) sign.
\item
For any simple polygon $P_1\cdots P_n$, the directed areas
$[P_iP_jP_k]_{\pm}$ all have the same sign, and it is the same as the sign
of $[P_1\cdots P_n]_{\pm}$. (This follows from the previous parts of this
Fact; do you see how?)
\end{itemize}
\end{fact}

For $P_1\cdots P_n$ a convex polygon,
we call the sign of $[P_1\cdots P_n]_{\pm}$ the
\emph{orientation} of $P_1\cdots P_n$;
\index{orientation (of a convex polygon)}
we refer to positive and negative orientations also as
``counterclockwise'' and ``clockwise'', respectively.

\section{Areas of circles and measures of arcs}

Everything we have discussed so far was described purely in terms of
basic algebraic operations on the real numbers: addition, subtraction,
multiplication, division and square roots. The area of a circle and the
measure of an arc cannot be described
quite so simply; one must use the least upper bound property.

Given a circle $\omega$, the \emph{area} of the circle
\index{area!of a circle} is defined to be the least upper bound of the
set of areas of convex polygons $P_1\cdots P_n$ with vertices on $\omega$;
note that this set is indeed bounded, for instance by the area of any
square containing $\omega$ in its interior.
\begin{fact}
\begin{itemize}
\item
There exists a constant $\pi$ such that the area of a circle of radius
$r$ is equal to $\pi r^2$.
\item
The area of a circle is also equal to the greatest lower bound of the
set of areas of convex polygons $P_1\cdots P_n$ containing $\omega$ in its
interior.
\end{itemize}
\end{fact}

Next we consider arcs and their measures.
Given three distinct points $A,B,C$ on a circle $\omega$,
we define the \emph{arc} \index{arc|textbf}
$\arc{ABC}$ as the set of points $D \in \omega$
such that the quadrilateral $ABCD$ is not simple (including the
points $A,B,C$).
Since $\omega$ is uniquely determined by any arc, we may
unambiguously speak of the \emph{center} and \emph{radius} of an arc.
\index{center!of an arc|textbf} \index{radius!of an arc|textbf}
There is a unique point
$M$ on $\arc{ABC}$ with $AM = MC$ (namely the intersection of $\arc{ABC}$
with the perpendicular bisector of $AC$), called the
\emph{midpoint} \index{midpoint!of an arc|textbf} of $\arc{ABC}$.
If the line $\line{AB}$ passes through the center of $\omega$, we call
$\arc{ABC}$ a \emph{semicircle}. \index{semicircle|textbf}

The polygon $P_1 \cdots P_n$ is said to be
\emph{inscribed} \index{inscribed (polygon in an arc/circle)|textbf}
in $\arc{ABC}$ or circle $\omega$ if $P_1, \dots, P_n$ all lie on $\arc{ABC}$.
We also say that the polygon is \emph{circumscribed}
\index{circumscription (of a polygon by an arc/circle|textbf}
by the arc/circle. A polygon which can be circumscribed by some circle
is said to be \emph{cyclic}, \index{cyclic (polygon)|textbf} and the
unique circle which circumscribes it is called its
\emph{circumcircle} (or \emph{circumscribed circle})
\index{circumcircle (of a cyclic polygon)|textbf}
\index{circumscribed circle|see{circumcircle}}
of the polygon; points which form the vertices of a 
cyclic polygon are said to be \emph{concyclic}.
\index{concyclic (points)}
The center and radius of the circumcircle of a cyclic polygon are
referred to as the \emph{circumcenter} \index{circumcenter (of a cyclic polygon)|textbf} and \emph{circumradius}, \index{circumradius (of a cyclic polygon)|textbf} respectively, of the polygon.
Note that any triangle is cyclic, so we may speak of the circumcenter and
circumradius of a triangle without any further assumptions.

To define the measure of an arc, we use the following fact.
\begin{fact}
Let $A$ and $B$ be points on a circle $\omega$ with center $O$ and radius
$r$, and let the lines tangent to $\omega$ at $A$ and $B$ meet at $C$. Then
\[
\frac{2[OAB]}{r^2} \leq \frac{AB}{r} \leq \frac{2[OACB]}{r^2}.
\]
\end{fact}
It follows that the least upper bound of the perimeters of polygons
inscribed in an arc/circle of radius $r$ exists, and is equal to
$\frac{1}{r}$ times the least upper bound of the areas of polygons
inscribed in the arc/circle. We call this quantity the
\emph{circumference} of the arc/circle
\index{circumference (of an arc/circle)|textbf}
and call $\frac{1}{r}$ times the circumference the
\emph{measure} of the arc/circle, denoted $m(\arc{ABC})$.
\index{measure (of an arc/circle)|textbf}
In particular, by the previous fact, the measure of any circle is
equal to $2\pi$.

\begin{fact} \label{fact:arcsum}
If $ABCDE$ is a convex polygon inscribed in a circle $\omega$, then the measure
of $\arc{ACE}$ is equal to the sum of the measures of $\arc{ABC}$
and $\arc{CDE}$.
\end{fact}

As with areas, it is sometimes convenient to give arcs a signed measure.
For $\arc{ABC}$ not a semicircle,
we define the \emph{signed measure} \index{signed measure (of an arc/circle)|textbf} of the arc $\arc{ABC}$,
denote $m_{\pm}(\arc{ABC})$, to be $m(\arc{ABC})$ times the sign
of $[ABC]_{\pm}$. We regard this as a quantity ``modulo $2\pi$'', i.e.,
only well-defined up to adding multiples of $2\pi$. (If $\arc{ABC}$ is a
semicircle, then $\pi$ and $-\pi$ differ by a multiple of $2\pi$, so
we may declare either one to be $m_{\pm}(\arc{ABC})$.)
Despite the ambiguity thus introduced, it still makes sense to add,
subtract and test for equality signed measures, and one has the following
nice properties which are most definitely false for the ordinary measure.
\begin{itemize}
\item The signed measure $m_{\pm}(\arc{ABC})$ 
depends only on $A,C$ and the circle
on which $\arc{ABC}$ lies.
\item If $A,B,C,D,E$ lie on a circle, then 
$m_{\pm}(\arc{ABC})+m_{\pm}(\arc{CDE}) = m_{\pm}(\arc{ACE})$.
\end{itemize}

\section{Angles and the danger of configuration dependence}
\label{sec:angles}

Given distinct points $A,B,C$ in the plane, choose a circle $\omega$
centered at $B$, let $A'$ and $C'$ be the intersections of the rays
$\ray{BA}$ and $\ray{BC}$, respectively, with $\omega$, and let $B'$
be any point on $\omega$ such that the quadrilateral $BA'B'C'$ is convex.
Then $m(\arc{A'B'C'})$ is independent of the choices of $\omega$
and $B'$; we call it the
\emph{angle measure}, \index{angle measure|textbf}
or simply the \emph{angle}, \index{angle|textbf}
between the rays $\ray{BA}$ and $\ray{BC}$ and denote it 
$\ang ABC$. 
We also use ``angle'' to describe the set consisting of the union
of the two rays $\ray{BA}$ and $\ray{BC}$, and use the symbol $\ang ABC$ for
this set as well.

The implied unit of angle measure above is called the \emph{radian}.
\index{radian (angle measure)|textbf} 
By tradition, we also measure angles in units of
\emph{degrees} \index{degree (angle measure)|textbf}, where $180^\circ$
equal $\pi$ radians.

We say that the angle $\ang ABC$ is \emph{acute, right, obtuse} 
\index{acute!angle|textbf} \index{angle!acute|textbf}
\index{right!angle|textbf} \index{angle!right|textbf}
\index{obtuse!angle|textbf} \index{angle!obtuse|textbf}
according to whether its measure is less than, equal to, or greater than
$\pi/2$ (or in degrees, $90^\circ$).

The \emph{interior} \index{interior!of an angle} of the angle $\ang ABC$
consists of all points $D$ such that the quadrilateral $ABCD$ is convex.
The set of points $D$ in the interior of $\ang ABC$ such that $\ang ABD = \ang DBC$,
together with $B$ itself, form a ray; that ray
is called the \emph{internal angle bisector}
\index{internal angle bisector|textbf} \index{angle bisector!internal|textbf}
of $\ang ABC$. The same term is applied to the line containing that ray.
The line perpendicular to the internal angle bisector is called the
\emph{external angle bisector} \index{external angle bisector|textbf}
\index{angle bisector!external|textbf} of $\ang ABC$.
We refer to the internal and external angle bisectors of $\ang ABC$ also as
the internal and external angle bisectors, respectively, of the triangle
$\triangle ABC$ at the vertex $B$.

\begin{fact}
\begin{itemize}
\item
If $ABC$ is a triangle, then $\angle ABC + \angle BCA + \angle CAB = \pi$.
\item
If $ABCD$ is a convex quadrilateral, then
$ABCD$ is cyclic if and only if $\ang ABC = \pi - \ang CDA$
if and only if $\ang ABD = \ang ACD$.
\item
If $ABCD$ is a convex quadrilateral and $\omega$ is the circumcircle of
$\triangle ABC$, then $\line{AD}$ is tangent to $\omega$ at $A$ if and only
if $\ang ABC = \ang DAC$.
\end{itemize}
\end{fact}

In the \emph{Elements}, \index{Elements@\textit{Elements} (of Euclid)}
\index{Euclid's \textit{Elements}}
all quantities are treated as unsigned, as
was necessary at the time in the absence of negative numbers. (Indeed,
Euclid \index{Euclid of Alexandria}
did not attach any numbers to geometrical figures; rather, he
would conflate a segment with its length, a polygon with its area, and so on.)
As we have seen, this approach makes certain statements dependent on
relative positions of points. It is common to not worry much about such
issues, but one actually has to be careful.
As an example, we offer the following ``pseudotheorem''
\index{pseudotheorem} and corresponding ``pseudoproof''
\index{pseudoproof} from \cite{bib:max}; we encourage the reader to concoct other plausible pseudotheorems and pseudoproofs!

\newtheorem*{pseudo}{Pseudotheorem}
\begin{pseudo}
All triangles are isosceles.
\end{pseudo}
\begin{proof}[Pseudoproof]
Let $\triangle ABC$ be a triangle, and let $O$ be the intersection of the 
internal angle bisector of $A$ with the perpendicular bisector of $\seg{BC}$,
as in Figure~\ref{fig:pseudothm}.
\begin{figure}[ht]
\caption{The Pseudotheorem.} 
\label{fig:pseudothm}
%% Diagram has $O$ inside.
\end{figure}

Let $D, Q, R$ be the feet of perpendiculars from $O$ to $\line{BC}, \line{CA}, 
\line{AB}$, 
respectively. By symmetry across $OD$, $OB = OC$, while by symmetry 
across $AO$, $AQ = AR$ and $OQ = OR$. Now the right triangles $\triangle ORB$ 
and $\triangle OQC$ have equal legs $OR = OQ$ and equal hypotenuses $OB = OC$, 
so they are congruent, giving $RB = QC$. Finally, we conclude
\[
AB = AR + RB = AQ + QC = AC,
\]
and hence the triangle $\triangle ABC$ is isosceles.
\end{proof}

\begin{exer}
\ii
Where is the error in the proof of the Pseudotheorem?
\end{exer}

\section{Directed angle measures}
\label{sec:directed}

To avoid the problems that led to the Pseudotheorem, it will be useful to
have a sign convention for angles.\footnote{This sign convention would seem to be rather old, but we do not have precise information about its origins.}
The obvious choice would be to regard angle measures as being defined
modulo $2\pi$, but it turns out better to regard them modulo $\pi$, as we
shall see.

Given three distinct points $A,B,C$, define the \emph{directed angle measure},
\index{directed angle!measure|textbf}
or simply the \emph{directed angle},
\index{directed angle!formed by three points|textbf}
$\dang ABC$ as the signed arc measure $m_{\pm}(\arc{A'B'C'})$ as a
quantity modulo $\pi$.

Yes, you read correctly: although the signed arc
measure is well-defined up to multiples of $2\pi$, we regard the directed
angle measure as only well-defined up to adding multiples of $\pi$.
One consequence is that we can unambiguously
define the directed angle (modulo $\pi$)
\index{directed angle!between two lines|textbf}
between two lines $\ell_1$ and $\ell_2$ as follows.
If $\ell_1$ and $\ell_2$ are parallel, declare the directed angle
$\dang(\ell_1,\ell_2)$ to be zero. Otherwise, declare
$\dang(\ell_1,\ell_2) = \dang ABC$ for $A$ a point on $\ell_1$ but not
on $\ell_2$, $B$ the intersection of $\ell_1$ and $\ell_2$, and
$C$ a point on $\ell_2$ not on $\ell_1$.

Note that to avert some confusion, we will systemically distinguish between the words ``signed'', referring to arc measures modulo $2\pi$, and ``directed'', referring to angle measures modulo $\pi$, even though in common usage these two terms might be interchanged. Such an interchange would be dangerous for us!

One can now verify the following rules of ``directed angle 
arithmetic'', all of which are independent of configuration.
\index{arithmetic!of directed angles}
\begin{fact} \label{fact:directed angles}
Let $A,B,C,D,O,P$ denote distinct points in the plane.
\begin{enumerate}
\ii
$\dang ABC = -\dang CBA$.
\ii
$\dang APB + \dang BPC = \dang APC$.
\ii
$\dang ABC = \dang ABD$ if and only if $B,C,D$ are collinear.
In particular, $\dang ABC = 0$ if and only if $A,B,C$ are collinear.
\ii
$\dang ABD = \dang ACD$ if and only if $A,B,C,D$ are concyclic.
\ii
$\dang ABC = \dang ACD$ if and only if $CD$ is tangent to the circle 
passing through $A,B,C$.
\ii
$\dang ABC + \dang BCA + \dang CAB = 0$.
\ii
$2\dang ABC = \dang AOC$ if $A,B,C$ lie on a circle centered at $O$.
\ii
$\dang ABC$ equals $\half$ of the measure of the arc $\arc{AC}$ of
the circumcircle of $ABC$.
\end{enumerate}
\end{fact}
For example, if $A,B,C,D$ lie on a circle in that order, then we 
have $\ang ABD = \ang ACD$ as undirected angles. On the other hand, if 
they lie on a circle in the order $A,B,D,C$, then we have
$\ang ABD = \pi - \ang DCA$, so in terms of directed angles
\[
\dang ABD = \pi - \dang DCA = - \dang DCA = \dang ADC.
\]
It should be noted that this coincidence is a principal reason why 
one works modulo $\pi$ and not $2\pi$! (The other principal reason is 
of course so that collinear points always make an angle of 0.)

The last two assertions in Fact~\ref{fact:directed angles} ought to raise some eyebrows, because division by 2 is a dangerous thing when working modulo $\pi$.
To be precise, the
equation $2\angle A = 2\angle B$ of directed angles does not imply 
that $\angle A = \angle B$, for the possibility also exists that 
$\angle A = \angle B + \pi/2$. (Those familiar with elementary number 
theory will recognize an analogous situation: one cannot divide by 2 
in the congruence $2a \equiv 2b \pmod{c}$ when $c$ is even.) This 
explains why we do not write $\dang ABC = \half \dang AOC$: the latter 
expression is not well-defined.\
On the other hand, directed arcs can be unambiguously measured mod 
$2\pi$, so dividing a signed arc measure by 2 gives a directed angle measure
mod $\pi$. 

If all of 
this seems too much to worry about, do not lose hope; the conventions 
are easily learned with a little practice. We will illustrate this
in Section~\ref{sec:angle chasing}.

\chapter{Algebraic methods}

Since our very construction of the Euclidean plane was rooted in
the algebra of the coordinate plane, it is clear that algebraic techniques
have something to say about Euclidean geometry; indeed, we have already encountered a few problems that are naturally treated in terms of coordinates,
and we will encounter more later (e.g., Theorem~\ref{thm:ap}).
However, coordinatizing a typical problem in Euclidean geometry leads 
to a complicated mess; one typically succeeds more easily by adopting more high-level techniques.\footnote{An analogous relationship in computer science is that between a processor-level machine language and a high-level programming language like C++, Java, or Perl.}

We will spend much of the second part of the book introducing so-called ``synthetic'' techniques; for now, we introduce some techniques which, while still rooted in algebra, offer some advantages over blind coordinate manipulations.

\section{Trigonometry}

Define the points $O = (0,0)$ and $P = (1,0)$.
Given a signed angle measure $\theta$ (modulo $2\pi$),
let $Q$ be the point on the circle of radius 1 centered at $O$ such that
the signed angle measure of $\angle POQ$ is equal to $\theta$. Let
$\cos \theta$ and $\sin \theta$ denote the $x$-coordinate and
$y$-coordinate, respectively, of $Q$; these define the \emph{cosine}
\index{cosine|textbf} and \emph{sine}. \index{sine|textbf} By the distance
formula, we have the identity
\[
\cos^2 \theta + \sin^2 \theta = 1.
\]
\begin{figure}[ht]
\caption{Definition of the trigonometric functions.}
\end{figure}

Also define
the \emph{tangent}, \index{tangent (trigonometric function)|textbf} \emph{secant}, \index{secant|textbf} \emph{cosecant}, \index{cosecant|textbf} and \emph{cotangent} \index{cotangent|textbf} functions as follows (when these expressions
are well-defined):
\[
\tan \theta  = \frac{\sin \theta}{\cos \theta}, \qquad
\sec \theta = \frac{1}{\cos \theta}, \qquad
\csc \theta = \frac{1}{\sin \theta}, \qquad
\cot \theta = \frac{\cos \theta}{\sin \theta}.
\]

We do not intend to conduct here a full course in trigonometry; we will content ourselves to summarizing
the important facts and provide a few problems where trigonometry can 
or must be employed. Throughout the following discussion,
let $ABC$ denote a triangle, write $a = BC, b = CA, c = AB$, and write $A,B,C$ for the measures of (undirected) angles $\ang CAB$, $\ang ABC$, $\ang BCA$,
respectively. Let $s = (a+b+c)/2$ denote the semiperimeter \index{semiperimeter (of a polygon)} of $\triangle ABC$.

\begin{fact}[Law of Sines] \index{Law of Sines|textbf}
The area of $\triangle ABC$ \index{area formula!for a triangle}
equals $\frac{1}{2} a b \sin C$. In 
particular,
\[
\frac{a}{\sin A} = \frac{b}{\sin B} = \frac{c}{\sin C}.
\]
\end{fact}

\begin{fact}[Extended Law of Sines] \index{Extended Law of Sines|textbf}
\index{Law of Sines|Extended|textbf}
If $R$ is the circumradius of $\triangle ABC$, then $BC = 2R \sin A$.
\end{fact}

\begin{fact}[Law of Cosines] \index{Law of Cosines}
In $\triangle ABC$,
\[
c^{2} = a^{2} + b^{2} - 2 a b \cos C.
\]
\end{fact}

\begin{fact}[Addition formulae] \index{addition formulae (for trigonometric functions)|textbf}
For any real numbers $\alpha$ and $\beta$,
\beqa
\cos (\alpha +\beta) &=& \cos \alpha \cos \beta - \sin \alpha \sin \beta \\
\cos (\alpha-\beta) &=& \cos \alpha \cos \beta + \sin \alpha \sin \beta \\
\sin (\alpha+\beta) &=& \sin \alpha \cos \beta + \cos \alpha \sin \beta \\
\sin (\alpha-\beta) &=& \sin \alpha \cos \beta - \cos \alpha \sin \beta.
\eeqa
\end{fact}
Using the addition formulae, one can convert products of sines and 
cosines to sums, and vice versa.
\begin{fact}[Sum-to-product formulae] \index{sum-to-product formulae
(for trigonometric functions)|textbf}
\beqa
\sin \alpha + \sin \beta &=& 2 \sin \frac{\alpha+\beta}{2} \cos \frac{\alpha-\beta}{2} \\
\sin \alpha - \sin \beta &=& 2 \cos \frac{\alpha+\beta}{2} \sin \frac{\alpha-\beta}{2} \\
\cos \alpha + \cos \beta &=& 2 \sin \frac{\alpha+\beta}{2} \cos \frac{\alpha-\beta}{2} \\
\cos \alpha - \cos \beta &=& -2 \sin \frac{\alpha+\beta}{2} \sin \frac{\alpha-\beta}{2}.
\eeqa
\end{fact}
In particular, one has the double and half-angle formulae.
\begin{fact}[Double-angle formulae] \index{double-angle formulae (for
trigonometric functions)|textbf}
\beqa
\sin 2\alpha &=& 2 \sin \alpha \cos \alpha \\
\cos 2\alpha &=& 2 \cos^2 \alpha - 1 = 1 - 2 \sin^2 \alpha \\
\tan 2\alpha &=& \frac{2 \tan \alpha}{1 - \tan^2 \alpha}.
\eeqa
\end{fact}
\begin{fact}[Half-angle formulae] \index{half-angle formulae (for trigonometric functions)|textbf}
\beqa
\sin \frac \alpha 2 &=& \pm \sqrt{\frac{1 - \cos \alpha}{2}} \\
\cos \frac \alpha 2 &=& \pm \sqrt{\frac{1 + \cos \alpha}{2}} \\
\tan \frac \alpha 2 &=& \csc \alpha - \cot \alpha.
\eeqa
\end{fact}
The half-angle formulae take a convenient form for triangles.
\begin{fact}
In $\triangle ABC$,
\beqa
\sin \frac C2 &=& \sqrt{\frac{(s-a)(s-b)}{ab}} \\
\cos \frac C2 &=& \sqrt{\frac{s(s-c)}{ab}}.
\eeqa
\end{fact}
It may be helpful at times to express certain other quantities 
associated with a triangle in terms of the angles.
\begin{fact}
If $\triangle ABC$ has inradius $r$ and circumradius $R$, then
\[
r = 4R \sin \frac{A}{2} \sin \frac{B}{2} \sin \frac{C}{2}.
\]
\end{fact}
We leave the construction of other such formulae to the reader.

\begin{exer}
\ii
For any triangle $ABC$, prove that $\tan A + \tan B + \tan C = \tan 
A \tan B \tan C$ and that $\cot A/2 + \cot B/2 + \cot C/2 = \cot A/2 
\cot B/2 \cot C/2$.
\ii
Show that if none of the angles of a convex quadrilateral $ABCD$
 is a right angle, then
\[
\frac{\tan A + \tan B + \tan C + \tan D}{\tan A \tan B \tan C \tan D}
 = \cot A + \cot B + \cot C + \cot D.
\]
\ii
Find a formula for the area of a triangle \index{area formula!for a triangle}
in terms of two angles and 
the side opposite the third angle. More generally, given any data 
that uniquely determines a triangle, one can find an area formula in 
terms of that data. Some of these can be found in
Fact~\ref{fact:area formulas}; can you come up with some others?
\ii
(USAMO 1996/5)
Triangle $ABC$ has the following property: there is an interior point 
$P$ such that $\angle PAB = 10^{\circ}$, $\angle PBA = 20^{\circ}$, 
$\angle PCA = 30^{\circ}$ and $\angle PAC = 40^{\circ}$. Prove that 
triangle $ABC$ is isosceles. (For an added challenge, find a 
non-trigonometric solution!)
\item (IMO 1985/1)
A circle has center on the side $AB$ of a cyclic quadrilateral $ABCD$.
The other three sides are tangent to the circle. Prove that $AD+DC=AB$.
\end{exer}

\section{Vectors}
 \label{sec:vec}

A \emph{vector} \index{vector|textbf}
in the plane can be defined either as an arrow, where
addition of arrows proceeds by the ``tip-to-tail'' rule illustrated
in Figure~\ref{fig:vectoradd}, 
or as an ordered pair $(x, y)$ recording the difference in the
$x$ and $y$ coordinates between the tip and the tail. Vectors in a Euclidean space of three or more dimensions may be defined similarly.
\begin{figure}[ht]
\caption{Vector addition.}
\label{fig:vectoradd}
\end{figure}

It is important to remember that a vector is not a point, but 
rather the ``difference of two points''; it encodes relative, not 
absolute, position. In practice, however, one chooses a point as the 
origin and identifies a point with the vector from the origin 
to that point. (In effect, one puts the tails of all of the arrows in 
one place.)

The standard operations on vectors include addition and subtraction, 
multiplication by real numbers (positive, negative or zero), and the 
\emph{dot product}, \index{dot product|textbf} defined geometrically as
\[
\vA \cdot \vB = \norm{\vA} \cdot \norm{\vB} \cos \angle AOB,
\]
where $O$ is the origin, and in coordinates as
\[
(a_{x}, a_{y}) \cdot (b_{x}, b_{y}) = a_{x}b_{x} + a_{y}b_{y}.
\]
The key fact here is that $\vA \cdot \vB = 0$ if and only if $\vA$ 
and $\vB$ are perpendicular.

A more exotic operation is the \emph{cross product}, \index{cross product|textbf} which is defined 
for a pair of vectors in three-dimensional space as follows:
\[
(a_{x}, a_{y}, a_{z}) \times (b_{x}, b_{y}, b_{z}) = (a_{y}b_{z} - 
a_{z}b_{y}, a_{z}b_{x} - a_{x}b_{z}, a_{x}b_{y} - a_{y}b_{x}).
\]
Geometrically speaking, $\vA \times \vB$ is 
perpendicular to both $\vA$ and $\vB$ and has length
\[
\norm{\vA \times \vB} =  \norm{\vA} \cdot \norm{\vB} \sin \angle AOB.
\]
This length equals the area of the parallelogram with vertices $0,
\vA, \vA+\vB, \vB$, or twice the area of the triangle with vertices
$0, \vA, \vB$.
The sign ambiguity can be resolved by the \emph{right-hand 
rule} (see Figure~\ref{fig:righthand}): 
if you point the fingers of your right hand along $\vA$, then 
swing them toward $\vB$, your thumb points in the direction of $\vA 
\times \vB$.
\begin{figure}[ht]
\caption{The right-hand rule.}
\label{fig:righthand}
\end{figure}

\begin{fact}
The following identities hold:
\begin{enumerate}
\ii
(Triple scalar product identity) $\vA \cdot (\vB \times \vC) = \vB
\cdot (\vC \times \vA) = \vC \cdot (\vA \times \vB)$. (Moreover, this
quantity equals the volume of a parallelepiped with edges $\vA, \vB,
\vC$, although we did not rigorously define either volumes or parallelepipeds.)
\ii
(Triple cross product identity) $\vA \times (\vB \times \vC) = (\vC
\cdot \vA) \vB - (\vB \cdot \vA)\vC$.
\end{enumerate}
\end{fact}


\begin{exer}
\ii (Romania, 1997)
Let $ABCDEF$ be a convex hexagon, and let $P = \line{AB} \cap \line{CD},
 Q = \line{CD} 
\cap \line{EF}, R = \line{EF} \cap \line{AB}$, $S = \line{BC} \cap \line{DE}, T = \line{DE} \cap \line{FA}, U = \line{FA} \cap 
\line{BC}$. Prove that
\[
\frac{PQ}{CD} = \frac{QR}{EF} = \frac{RP}{AB}
\quad \mbox{if and only if} \quad \frac{ST}{DE} = \frac{TU}{FA} = \frac{US}{BC}.
\]
\ii (R\u{a}zvan Gelca) 
Let $ABCD$ be a convex quadrilateral and $O = \seg{AC} \cap \seg{BD}$. Let $M, N$
be points on $\seg{AB}$ so that $AM = MN = NB$, and let $P, Q$ be points on
$\seg{CD}$ so that $CP = PQ = QD$. Show that triangles $\triangle
MOP$ and $\triangle NOQ$ have
the same area.
\ii (MOP 1996)
Let $ABCDE$ be a convex pentagon, and let $F,G,H,I,J$ be the 
respective midpoints 
of $\seg{CD}, \seg{DE}, \seg{EA}, \seg{AB}, \seg{BC}$. If 
$\seg{AF}, \seg{BG}, \seg{CH}, \seg{DI}$ pass through a common 
point, show that $\seg{EJ}$ also passes through this point.
\ii (Austria-Poland, 1979)
Let $A,B,C,D$ be points in space, let $M$ be
the midpoint of $\seg{AC}$, and let $N$ be 
the midpoint of $\seg{BD}$. Prove that
\[
4MN^{2} = AB^{2} + BC^{2} +CD^{2} + DA^{2} -AC^{2} - BD^{2}.
\]
\end{exer}

\section{Complex numbers}
\label{sec:cplx}

The set of \emph{complex numbers} \index{complex numbers|textbf}
consists of the expressions of the form $a+bi$
for $a,b$ real numbers, added and multiplied according to the rules
\begin{align*}
(a+bi) + (c+di) &= (a+c) + (b+d)i \\
(a+bi) \times(c+di) &= (ac-bd) + (ad+bc)i.
\end{align*}
We identify the real number $a$ with the complex number $a+0i$, and we write $bi$ for the complex number $0+bi$. In particular, the complex number $i = 1i$ is a square root of $-1$.

One uses complex numbers in Euclidean geometry by identifying a point with Cartesian coordinates $(x,y)$ with the complex number $x+yi$. This represents an extension of vector techniques, incorporating a convenient interpretation of
angles (and of similarity transformations; see Section~\ref{sec:complex sim}).

Another interesting use of complex numbers is to prove 
inequalities. This
use exploits the fact that the \emph{magnitude} \index{magnitude|of a complex
number|textbf}
\[
|a+bi| = \sqrt{a^2 + b^2}
\]
is multiplicative:
\[
|(a+bi) \times (c+di)| = |a+bi|\times |c+di|.
\]
Consider the following example (compare with Problem~\ref{ex:ptineq}).
\index{Ptolemy's inequality}
\begin{theorem}[Ptolemy's inequality] \label{thm:ptineq}
Let $A,B,C,D$ be four points in the plane. Then
\[
AC \cdot BD \leq AB \cdot CD + BC \cdot DA,
\]
with equality if and only if the quadrilateral $ABCD$ is convex (or
degenerate) and cyclic.
\end{theorem}
\begin{proof}
Regard $A,B,C,D$ as complex numbers; then we have an identity
\[
(A-C)(B-D) = (A-B)(C-D) + (B-C)(A-D).
\]
However, the magnitude of $(A-C)(B-D)$ is precisely the product of the 
lengths of the segments $AC$ and $BD$, and likewise for the other 
terms. Thus the desired inequality is simply the triangle inequality 
applied to these three quantities! (The equality condition is left as 
an exercise.)
\end{proof}

Although we will not have use of it for a while (not until 
Section~\ref{sec:alggeo}), we mention now an important, highly 
nontrivial fact about complex numbers.
\index{fundamental theorem of algebra}
\begin{theorem}[Fundamental theorem of algebra]
For every polynomial $P(z) = a_n z^n + \cdots + a_0$ with
$a_n,\dots,a_0$ complex numbers, and $a_n, \dots, a_1$ not all zero,
there exists a complex number $z$ such that $P(z) = 0$.
\end{theorem}
This theorem is traditionally attributed to 
(Johann) Carl Friedrich 
Gauss\footnote{Note that in German, ``Gauss'' is sometimes spelled
``Gau\ss''. That last character is not a beta, but rather an ``s-zed'', 
a ligature
of the letters ``s'' and ``z''.} 
(1777-1855), \index{Gauss (Gau\ss), (Johann) Carl Friedrich}
who gave the first proof (and several in addition) 
that would pass modern standards of rigor. However,
a correct proof (which relies on the existence of splitting fields, a concept
unavailable at the time) had already been sketched by Leonhard Euler
(1707--1783) \index{Euler, Leonhard}.

\begin{exer}
\ii
Prove that $x,y,z$ lie at the corners of an equilateral triangle if 
and only if either $x + \omega y + \omega^{2} z = 0$ or $x + 
\omega z + \omega^{2} y = 0$, where $\omega = e^{2\pi i/3}$.

\ii
Construct equilateral triangles externally (internally) on 
the sides of an arbitrary triangle $\triangle ABC$. 
Prove that the circumcenters of 
these three triangles form another equilateral triangle. This 
triangle is known as the \emph{inner (outer) Napoleon\footnote{Yes,
that's French emperor 
Napoleon Bonaparte (1769--1821), \index{Napoleon Bonaparte}
\index{Bonaparte, Napoleon} though it's not clear how his name came to
be attached to this result.} triangle} \index{Napoleon triangles 
(of a triangle)|textbf}
of $\triangle ABC$.

\ii
Let $P,Q,R,S$ be the circumcenters of squares constructed externally on 
sides $\seg{AB}$, $\seg{BC}$, $\seg{CD}$, $\seg{DA}$, 
respectively, of a convex quadrilateral $ABCD$. 
Show that the segments $\seg{PR}$ and $\seg{QS}$ are perpendicular to each other 
and equal in length.

\ii
Let $ABCD$ be a convex quadrilateral. Construct squares $CDKL$ and
$ABMN$ externally on sides $\seg{AB}$ and $\seg{CD}$. 
Show that if the midpoints of
$\seg{AC}, \seg{BD}, \seg{KM}, \seg{NL}$ do not coincide, 
then they form a square.

\ii (IMO 1977/1)
Equilateral triangles $\triangle ABK$, $\triangle BCL$, $\triangle CDM$, 
$\triangle DAN$ are constructed 
inside the square $ABCD$. Prove that the midpoints of the four 
segments $\seg{KL}$, $\seg{LM}$, $\seg{MN}$, $\seg{NK}$ 
and the midpoints of the eight segments 
$\seg{AK}, \seg{BK}$, $\seg{BL}, \seg{CL}$, $\seg{CM}, \seg{DM}$, $\seg{DN}, 
\seg{AN}$ are the twelve vertices of a 
regular dodecagon. (Nowadays the IMO tends to avoid geometry problems 
such as this one, which have no free parameters, but they are 
relatively common in single-answer contests such as ARML.)

\ii
Given a point $P$ on a circle and the
vertices $A_1,\,A_2,\dots,A_n$ of an inscribed regular $n$-gon, prove that:
\begin{enumerate}
\ii $PA_1^2+PA_2^2+\cdots+PA_n^2$ is a constant (independent of $P$). 
\ii $PA_1^4+PA_2^4+\cdots+PA_n^4$ is a constant (independent of $P$).
\end{enumerate}

\ii (China, 1998)
Let $P$ be an arbitrary point in the plane of triangle
$\triangle 
ABC$ with side lengths $BC=a$, $CA=b$, $AB=c$, and put $PA=x$, $PB=y$,
$PC=z$.  Prove that
\[
ayz+bzx+cxy\geq abc,
\]
with equality if and only if $P$ is the circumcenter of $\triangle ABC$.
\end{exer}

\section{Barycentric coordinates and mass points}

Given a triangle $\triangle ABC$ and a point $P$, we define the
\emph{barycentric coordinates}\footnote{The word ``barycentric'' comes from the Greek ``barys'', meaning ``heavy''; it is cognate to the term ``baryon'' in particle physics.} \index{barycentric coordinates|textbf} \index{coordinates|barycentric|textbf} of 
$P$ with respect to $ABC$ as the triple
\[
\left( \frac{[PBC]_{\pm}}{[ABC]_{\pm}},
\frac{[APC]_{\pm}}{[ABC]_{\pm}},
\frac{[ABP]_{\pm}}{[ABC]_{\pm}} \right)
\]
of real numbers. Note that these numbers always add up to 1, 
and that they are all positive if and only if $P$ lies in the interior of
$\triangle ABC$.

A related concept is that of ``mass points''. A \emph{mass point}
\index{mass points} consists of a pair $(P,r)$, where $P$ is a point and $r$ is a positive real number. These points may be ``added'' 
\index{addition|of mass points|textbf} as follows:
\[
((x_1,y_1), r_1) + (x_2,y_2),r_2) =
\left(\left( \frac{r_1x_1 + r_2x_2}{r_1+r_2}, \frac{r_1y_1 + r_2y_2}{r_1+r_2}
\right), r_1+r_2 \right).
\]
In terms of vectors, we have
\[
(P_1,r_1) + (P_2,r_2) =
\frac{r_1}{r_1+r_2}P_1 +  \frac{r_2}{r_1+r_2}P_2.
\]
In terms of physics, the location of the sum of two mass points is the ``center of mass'' of a pair of appropriate masses at the two points. This addition law is associative, so we may likewise add three or more mass points unambiguously.

The relation of mass points to barycentric coordinates is as follows.
\begin{fact}
Let $\triangle ABC$ be a triangle. Then for any positive real numbers
$r_1, r_2, r_3$, the sum of the mass points $(A,r_1)$, $(B, r_2)$, $(C, r_3)$ is located at the point with barycentric coordinates
\[
\left( \frac{r_1}{r_1 + r_2 + r_3}, \frac{r_2}{r_1 + r_2 + r_3}, \frac{r_3}{r_1 + r_2 + r_3} \right).
\]
\end{fact}

\chapter{Transformations}
\label{chap:transform}

In geometry, it is often useful to study transformations of the 
plane (i.e.\ functions mapping the plane to itself) preserving certain 
properties. In fact, Felix Klein (1849-1925) went so far as to define 
``geometry'' as the study of properties invariant under a particular set 
of transformations!

In this section we describe some fundamental transformations and how they interact with properties of  ``figures'' in the
plane. Here and throughout, by a \emph{figure} \index{figure} we simply mean a set of points in the plane. Also, we follow standard usage in mathematical English and refer to a function also as a ``map''.

\section{Congruence and rigid motions}

Let $F_1$ and $F_2$ be two figures and suppose $f: F_1 \to F_2$ is a bijection
(one-to-one correspondence).
We say that \emph{$F_1$ and $F_2$ are congruent (via $f$)}, \index{congruence (of two figures|textbf} and write $F_1 \cong F_2$,
if we have an equality of distances $PQ = f(P)f(Q)$ for all
$P,Q \in F_1$. When $F_1$ and $F_2$ are polygons with the same number
of vertices and $f$ is not specified, we assume it is the map that takes
the vertices of $F_1$ to the vertices of $F_2$ in the order that they are
listed. For instance, the fact that $\triangle ABC \cong \triangle DEF$
are congruent means that $AB=DE, BC = EF, CA = FD$.

A \emph{rigid motion} \index{rigid motion|textbf}
of the Euclidean plane is a map from the plane to itself which preserves 
distances; that is, if $P$ maps to $P'$ and $Q$ to $Q'$, then we have 
$PQ = P'Q'$. In other words, a rigid motion maps every figure to a congruent
figure. Here is a list of examples of rigid motions (which we will soon
find to be exhaustive; see Theorem~\ref{thm:classify rigid}):
\begin{itemize}
\item
\textit{Translation}: \index{translation|textbf}
each point moves a fixed distance in a fixed direction, 
so that $PQQ'P'$ is always a parallelogram.
\item
\textit{Rotation} \index{rotation|textbf}
with center $O$ and angle $\theta$: each point 
$P$ maps to the point $P'$ such that $OP = OP'$ and $\ang POP' = \theta$, 
where the angle is signed (i.e., measured modulo $2\pi$, not $\pi$).
We refer to a rotation with angle $\pi$ also as a 
\emph{half-turn}. \index{half-turn|textbf}
\item
\textit{Reflection} \index{reflection|textbf} 
through the line $\ell$: each point $P$ maps to the point 
$P'$ such that $\ell$ is the perpendicular bisector of $PP'$.
\item
\textit{Glide reflection} \index{glide reflection|textbf} 
along the line $\ell$: reflection 
through $\ell$ followed by a translation along $\ell$.
\end{itemize}
\begin{theorem} 
Given two congruent figures, each not contained in any line,
there is a unique rigid motion that maps one onto the other (matching
corresponding points).
\end{theorem} \label{thm:rigid unique}
Note that the rigid motion may not be unique if it is not required to match corresponding points between the two figures: for instance, a regular polygon
is mapped to itself by more than one rotation (as for that matter is a circle).

\begin{proof}
We first address the uniqueness. If there were two rigid motions 
carrying the first figure to the second, then composing one with the 
inverse of the other would yield a nontrivial rigid motion leaving 
one entire figure in place. By assumption, however, this figure 
contains three noncollinear points $A, B, C$, and a point $P$ is 
uniquely determined by its distances to these three points (Fact~\ref{fact:triangulation principle}), so every point is fixed by the rigid motion, a 
contradiction. Thus the motion is unique if it exists.

Now we address existence.
let $A, B, C$ be three noncollinear points of the first figure, and
$A', B', C'$ the corresponding points of the second figure. There 
exists a translation mapping $A$ to $A'$; following that with a 
suitable rotation (since $AB = A'B'$), we can ensure that $B$ also 
maps to $B'$. Now we claim $C$ maps either to $C'$ or to its 
reflection across $\line{A'B'}$; in other words, given two points $A,B$ and 
a point $C$ not on $\line{AB}$, $C$ is determined up to reflection across 
$\line{AB}$ by the distances $AC$ and $BC$. This holds because 
this data fixes $C$ to lie on two distinct circles, which 
can only intersect in two points (Fact~\ref{fact:intersect line circle}).

Now if $P$ is any point of the first figure, then $P$ is uniquely 
determined by the distances $AP, BP, CP$ (again by
Fact~\ref{fact:triangulation principle}),
and so it must map to the corresponding point of the second figure. 
This completes the proof of existence.
\end{proof}

Note that rigid motions carry convex polygons to convex polygons.
It follows that a rigid motion either preserves the orientation of all
convex polygons, or reverses the orientation of all convex polygons.
We say two congruent figures are \emph{directly congruent}
\index{direct congruence (of two figures)|textbf} if the unique
rigid motion taking one to the other (provided by
Theorem~\ref{thm:rigid unique}) preserves orientations, and
\emph{oppositely congruent} \index{opposite!congruence (of two figures)|textbf} otherwise.

\index{classification!of rigid motions} \index{rigid motion!classification of}
\begin{theorem} \label{thm:classify rigid}
A rigid motion preserves orientations if and only if it is a translation or
a rotation. A rigid motion reverses orientations if and only if it is a reflection or a glide reflection.
\end{theorem}
\begin{proof}
Let $ABC$ be a triangle carried to the triangle $A'B'C'$ under the rigid
motion. First suppose the rigid motion preserves orientations; by the
uniqueness assertion in Theorem~\ref{thm:rigid unique}, it suffices to exhibit
either a translation or a rotation carrying $\triangle ABC$ to $\triangle A'B'C'$.
If the perpendicular bisectors of $\seg{AA'}$ and $\seg{BB'}$ are parallel,
then $ABB'A'$ is a parallelogram, so there is a translation taking $A$ to
$A'$ and $B$ to $B'$. Otherwise, let these perpendicular bisectors meet
at $O$. Draw the circle through $B$ and $B'$ centered at $O$; there
are (at most) two points on this circle whose distance to $A'$ is
the length $AB$. One point is the reflection of $B$ across the perpendicular
bisector of $\seg{AA'}$; by our assumption, this cannot equal $B'$. Thus $B'$
is the other point, which is the image of $B$ under the rotation about $O$
taking $B$ to $B'$.
\begin{figure}[ht]
\caption{Proof of Theorem~\ref{thm:classify rigid}.}
\end{figure}

Next suppose the rigid motion reverses orientations; again by
Theorem~\ref{thm:rigid unique}, it suffices to exhibit either a reflection or a glide reflection carrying $\triangle ABC$ to $\triangle A'B'C'$.
The lines through which $\line{AB}$ reflects to a line parallel to $\line{A'B'}$
form two perpendicular families of parallel lines. In each family there is
one line passing through the midpoint of $\seg{AA'}$; the glide reflection through 
this line taking $A$ to $A'$ takes $B$ either to $B'$ or to its half-turn
about $A'$. In the latter case, switching to the other family gives a
glide reflection taking $B$ to $B'$. As in the first case,
$C$ automatically goes to $C'$, and we are done.
\end{proof}

In particular, the composition of two rotations is either a rotation 
or translation. In fact, one can say more.
\begin{fact}
The composition of a rotation of angle $\theta_{1}$ with a rotation of 
angle $\theta_{2}$ is a rotation of angle $\theta_{1} + \theta_{2}$ if 
this is not a multiple of $2\pi$, and a translation otherwise.
\end{fact}
On the other hand, given two rotations, it is not 
obvious where the center of their composition is; in particular, it
generally depends on the order of the rotations. (In the language of abstract algebra, 
the group of rigid motions \index{group (of rigid motions)} is not commutative.)

\begin{fact}
Let $\triangle ABC$ and $\triangle A'B'C'$ be two triangles. The following conditions are
all equivalent to $\triangle ABC \cong \triangle A'B'C'$.
\begin{enumerate}
\item[(a)] (SSS criterion) We have $AB = A'B'$, $BC = B'C'$, $CA = C'A'$
(this equivalence is by definition).
\item[(b)] (SAS criterion) We have $AB = A'B'$, $BC = B'C'$, and
$\angle ABC = \angle A'B'C'$.
\item[(c)] (ASA criterion) We have $AB = A'B'$, and all three (or even any two) of $\ang ABC$, $\ang BCA$, $\ang CAB$ are equal to the corresponding
angles $\ang A'B'C'$, $\ang B'C'A'$, $\ang C'A'B'$.
\end{enumerate}
\end{fact}

\begin{exer}
\ii
Show that there is no ``SSA criterion'' for congruence, by exhibiting
two noncongruent triangles $\triangle ABC, \triangle A'B'C'$ with
$AB=A'B'$, $BC = B'C'$, $\angle BCA = \angle B'C'A'$.
\ii
(MOP 1997)
Consider a triangle $ABC$ with $AB = AC$, and points $M$ and $N$ on
$\seg{AB}$ and $\seg{AC}$, respectively. The lines $\line{BN}$ and $\line{CM}$ 
intersect at $P$.
Prove that $\line{MN}$ and $\line{BC}$ are parallel if and only if 
$\ang APM = \ang APN$.
\item (Butterfly theorem)
Let $A,B,C,D$ be points occurring in that order on circle $\omega$ and
put $P = \seg{AC} \cap \seg{BD}$. 
Let $EF$ be a chord of $\omega$ passing through $P$,
and put $Q = \seg{BC} \cap \seg{EF}$ and $R = \seg{DA} \cap \seg{EF}$. 
Then $PQ = PR$.
\ii
(IMO 1986/2)
A triangle $A_1A_2A_3$ and a point $P_0$ are given in the plane. We
define $A_s = A_{s-3}$ for all $s \geq 4$. We construct a sequence of
points $P_1, P_2, P_3, \dots$ such that $P_{k+1}$ is the image of $P_k$
under rotation with center $A_{k+1}$ through angle $120^\circ$
clockwise (for $k=0,1,2,\dots$). Prove that if $P_{1986} = P_0$, then
$\triangle A_1A_2A_3$ is equilateral.
\ii (MOP 1996) \label{ex:threeeq}
%%(IMO 96 Test 1, problem 3(b))
Let $AB_1C_1$, $AB_2C_2$, $AB_3C_3$ be directly congruent equilateral
triangles. Prove that the pairwise intersections of the circumcircles
of triangles $AB_1C_2$, $AB_2C_3$, $AB_3C_1$ form an equilateral triangle
congruent to the first three. 
\end{exer}

\section{Similarity and homotheties}

Let $F_1$ and $F_2$ be two figures and suppose $f: F_1 \to F_2$ is a bijection.
We say that \emph{$F_1$ and $F_2$ are similar (via $f$)}, \index{similarity (of two figures|textbf} and write $F_1 \sim F_2$, if
there exists a positive real number $c$ such that for all $P,Q \in F_1$,
$f(P)f(Q) = cPQ$. Again, when $F_1$ and $F_2$ are polygons with the same number
of vertices and $f$ is not specified, we assume it is the map that takes
the vertices of $F_1$ to the vertices of $F_2$ in the order that they are
listed. For instance, the fact that $\triangle ABC \sim \triangle DEF$
means that $AB/DE = BC/EF = CA/FD$.

\begin{fact}
Let $\triangle ABC$ and $\triangle A'B'C'$ be two triangles. The following conditions are
all equivalent to $\triangle ABC \sim \triangle A'B'C'$.
\begin{enumerate}
\item[(a)] (SSS criterion) We have $AB/A'B'= BC/B'C' = CA/C'A'$
(this equivalence is by definition).
\item[(b)] (SAS criterion) We have $AB/A'B' = BC/B'C'$, and
$\angle ABC = \angle A'B'C'$.
\item[(c)] (AA criterion) All three (or even any two) of $\ang ABC$, $\ang BCA$, $\ang CAB$ are equal to the corresponding
angles $\ang A'B'C'$, $\ang B'C'A'$, $\ang C'A'B'$.
\end{enumerate}
\end{fact}

A \emph{similarity} \index{similarity|textbf}
of the Euclidean plane is a map from the plane to itself for which
there exists a positive real number $c$ such that whenever
$P$ maps to $P'$ and $Q$ to $Q'$, we have 
$P'Q' = c PQ$; the constant $c$ is called the \emph{ratio of 
similitude} \index{ratio of similitude|textbf} of the similarity.
Note that a rigid motion \index{rigid motion}
is precisely a similarity with ratio of similitude 1.
By imitating the proof of Theorem~\ref{thm:rigid unique}, we have the following
result.
\begin{fact} \label{fact:sim unique}
Given two similar figures, each not contained in a line, there is a unique similarity that maps one onto the other (matching corresponding points).
\end{fact}
As for rigid motions, a similarity either preserves the orientation of all
convex polygons, or reverses the orientation of all convex polygons. We say two similar figures are \emph{directly similar}
\index{direct similarity (of two figures)|textbf} if the unique
similarity taking one to the other (provided by
Fact~\ref{fact:sim unique}) preserves orientations, and
\emph{oppositely similar} \index{opposite!similarity (of two figures)|textbf} otherwise. 

In the spirit of Theorem~\ref{thm:classify rigid}, we will shortly
give a classification of similarities (Theorem~\ref{thm:sim}); before
doing so, however, we introduce a special class of similarities which by themselves are already surprisingly useful.
Given a point $O$ and a nonzero real number $r$, the \emph{homothety} 
\index{homothety|textbf} (or \emph{dilation} \index{dilation|textbf}
or \emph{dilatation} \index{dilatation|textbf})
with center $O$ and ratio $r$ maps each point $P$  to the point 
$P'$ on the line $\line{OP}$ such that the signed ratio of lengths
\index{ratio of lengths, signed (of collinear segments)|textbf} 
\index{signed ratio of lengths (of collinear segments)|textbf}
$OP'/OP$ is equal to $r$. Note that $r$ is allowed to be negative; in
particular, a homothety of ratio $-1$ is simply a
half-turn. \index{half-turn}
\begin{figure}[ht]
\caption{A homothety.}
\end{figure}

Homotheties have the property that they map every segment of a figure to 
a parallel segment. Aside from translations (which might be thought of 
as degenerate homotheties with center ``at infinity''), this property 
characterizes homotheties; the following theorem is often useful as a 
concurrence criterion. \index{concurrence!criterion}
\begin{fact}
Two directly similar but not congruent figures with corresponding sides parallel 
are homothetic. In particular, the lines $\line{AA'}$, where $A$ and $A'$ are 
corresponding points, all pass through a common point.
\end{fact}
As for rotations, we conclude that the composition of two homotheties 
is a homothety, though again it is less than obvious where the 
center is!

\begin{exer}
\ii
Given a triangle $ABC$, construct (with straightedge and compass)
a square with one vertex on $\seg{AB}$, 
one vertex on $\seg{CA}$, and two (adjacent) vertices on $\seg{BC}$.
\ii
(USAMO 1992/4)
Chords $\, \overline{AA'}, \, \overline{BB'}, \, \overline{CC'} 
\,$ of a sphere meet at an interior point $\, P \,$ but are not 
contained in a plane.  The sphere through $\, A,B,C,P \,$ 
is tangent to the sphere through $\, A', B', C', P$. 
Prove that $\, AA' = BB' = CC'$. 
\ii
(Putnam 1996/A-2)
Let $C_1$ and $C_2$ be circles whose centers are 10 units apart, and
whose radii are 1 and 3. Find, with proof, the locus of all points $M$
for which there exists points $X$ on $C_1$ and $Y$ on $C_2$ such that
$M$ is the midpoint of the line segment $\seg{XY}$.
\ii
Given three nonintersecting circles, draw the intersection of the 
external tangents to each pair of the circles. Show that these three 
points are collinear.
\ii
(Russia, 2003) \label{ex:rus03homot}
Let $ABC$ be a triangle with $AB \neq AC$. Point $E$ is such that
$AE = BE$ and $\line{BE} \perp \line{BC}$. Point $F$ is such that $AF = CF$ and
$\line{CF} \perp \line{BC}$. Let $D$ be the point on line $\line{BC}$ such that
$\line{AD}$ is tangent to the circumcircle of triangle $ABC$. Prove that
$D,E,F$ are collinear.
\end{exer}

\section{Spiral similarities}
\label{sec:spir}

Let $P$ be a point,
let $\theta$ be a signed angle measure (i.e., measured modulo $2\pi$)
and let $r$ be a nonzero real number.
The \emph{spiral similarity} \index{spiral similarity|textbf}
of angle $\theta$ and ratio $r$ centered 
at $P$ consists of a homothety \index{homothety} of ratio $r$ centered at $P$ 
followed by a rotation \index{rotation}
of angle $\theta$ centered at $P$. (The order 
of these operations do not matter; one easy way to see this is to 
express both operations in terms of complex numbers.)
In particular, a spiral similarity is the composition of a similarity and a rigid motion, and hence is a similarity.

\begin{exer}
\ii
(USAMO 1978/2)
The squares $ABCD$ and $A'B'C'D'$ represent maps of the same region, drawn to
different scales and superimposed. Prove that there is only one point
$O$ on the small map which lies directly over point $O'$ of the large
map such that $O$ and $O'$ represent the same point of the country.
Also, give a Euclidean construction (straightedge and compass) for $O$.
\ii \label{ex:mop98spiral}
(MOP 1998)
Let $ABCDEF$ be a cyclic hexagon with $AB=CD=EF$. Prove that the intersections
of $\seg{AC}$ with $\seg{BD}$, of $\seg{CE}$ with $\seg{DF}$, and of $\seg{EA}$ 
with $\seg{FB}$ form a
triangle similar to $\triangle BDF$.
\ii \label{ex:sevenpt}
Let $C_1, C_2, C_3$ be circles such that $C_1$ and $C_2$ meet at
distinct points $A$ and $B$, $C_2$ and $C_3$ meet at distinct points
$C$ and $D$, and $C_3$ and $C_1$ meet at distinct points $E$ and $F$.
Let $P_1$ be an arbitrary point on $C_1$, and define points $P_2,
\dots, P_7$ as follows:
\begin{center}
\begin{tabular}{ll}
$P_2$ & is the second intersection of line $\line{P_1A}$ with $C_2$; \\
$P_3$ & is the second intersection of line $\line{P_2C}$ with $C_3$; \\
$P_4$ & is the second intersection of line $\line{P_3E}$ with $C_1$; \\
$P_5$ & is the second intersection of line $\line{P_1B}$ with $C_2$; \\
$P_6$ & is the second intersection of line $\line{P_2D}$ with $C_3$; \\
$P_7$ & is the second intersection of line $\line{P_3F}$ with $C_1$. \\
\end{tabular}
\end{center}

\noindent
Prove that $P_7 = P_1$.
\end{exer}

\section{Complex numbers and the classification of similarities}
\label{sec:complex sim}

One can imagine generating similarities in a rather complicated fashion, e.g., take a homothety about one center followed by a rotation about a different center.  It turns out that this does not really yield anything new; this can be seen most easily by interpreting similarities in terms of complex numbers.
\index{classification!of spiral similarities}
\begin{theorem} \label{thm:sim}
Every orientation-preserving similarity is either a translation or 
a spiral similarity.
\end{theorem}
\begin{proof}
First we show that every orientation-preserving similarity can be expressed in terms 
of homothety, translation, and rotation. Let $A$ and $B$ be two points 
with images $C$ and $D$. If we perform a homothety about $A$ of ratio 
$CD/AB$, then a translation mapping $A$ to $C$, then a suitable 
rotation, we get another similarity mapping
$A$ and $B$, respectively, to $C$ and $D$. On the other 
hand, if $P$ is any point not on the line $\line{AB}$, and $Q$ and $Q'$ are 
its images under the original similarity and the new similarity, 
then the triangles $\triangle ABP$, $\triangle CDQ$, $\triangle CDQ'$ 
are all similar. This implies that $C,Q,Q'$ are collinear and that
$D,Q,Q'$ are collinear, forcing $Q = Q'$. In 
other words, the original similarity coincides with the new one.

The basic transformations can be expressed in terms of complex 
numbers as follows:

\begin{center}
        \begin{tabular}{ll}
        Translation by vector $v$ & $z \mapsto z + v$ \\
        Homothety of ratio $r$, center $x$ & $z \mapsto r(z-x) + x$ \\
        Rotation by angle $\theta$, center $x$ & $z \mapsto e^{i\theta}(z-x) + 
        x $
        \end{tabular}
\end{center}
The point is that each of these maps has the form $z \mapsto az + b$ 
for some complex numbers $a,b$ with $a \neq 0$, and hence all orientation-preserving similarities have this form.

If $a=1$, the map $z \mapsto az+b$ of complex numbers represents a translation by $b$. If $a \neq 1$,  then let $t = b/(1-a)$ be the 
unique solution of $t = at + b$. Then our map can be written 
$z \mapsto a(z-t) + t$, which is clearly a spiral similarity about $t$.
\end{proof}

\begin{exer}
\ii
Let $A,C,E$ be three points on a circle. A $60^{\circ}$ rotation about 
the center of the circle
carries $A,C,E$ to $B,D,F$, respectively. Prove that the triangle 
whose vertices are 
the midpoints of $\seg{BC}, \seg{DE}, \seg{FA}$ is equilateral.

\end{exer}

\section{Affine transformations}

The last type of transformation we introduce in this chapter
is the most general, at 
the price of preserving the least structure. However, for sheer 
strangeness it does not rival either inversion (see Chapter~\ref{chap:inversion}) or projective 
transformations (see Chapter~\ref{chap:projective}).

An \emph{affine transformation} \index{transformation!affine|textbf} \index{affine transformation|textbf} is a map from the plane to itself of the form
\[
(x, y) \mapsto (ax + by + c, dx + ey + f)
\]
for some real numbers $a,b,c,d,e,f$ with $ae-bd \neq 0$; this last condition ensures that the map is a bijection.
From the proof of Theorem~\ref{thm:sim}, we see that every similarity is an affine transformation.
However, there are more exotic 
affine transformations, including the stretch $(x,y) \mapsto 
(x,cy)$ and the shear $(x,y) \mapsto (x+y,y)$.

These last examples demonstrate that angles and distances do not behave predictably under affine transformation. However, one does have the following result, as well as a partial converse (Problem~\ref{ex:characterize affine}).
\begin{fact}
Affine transformations preserve collinearity of points, parallelness of lines, and concurrence of lines. Moreover, the affine transformation
\[
(x, y) \mapsto (ax + by + c, dx + ey + f)
\]
multiplies areas by a factor of $|ae-bd|$, and preserves orientation if and only if $ae-bd > 0$.
\end{fact}

\begin{fact} \label{ex:aff}
Any three noncollinear points can be mapped to any three other 
noncollinear points by a unique affine transformation.
\end{fact}

As an example of the use of the affine transformation, we offer the 
following theorem.
\begin{theorem} \label{thm:sample affine}
Let  $ABCDE$ be a convex pentagon, and let $F = \line{BC} \cap \line{DE}$, 
$G = \line{CD} \cap 
\line{EA}$, $H = \line{DE} \cap \line{AB}$, $I = \line{EA} \cap \line{BC}$, 
$J = \line{AB} \cap \line{DE}$. Suppose 
that the areas of the triangles $\triangle AHI, \triangle BIJ, \triangle CJF, 
\triangle DFG, \triangle EGH$ are all 
equal. Then the lines $\line{AF}, \line{BG}, \line{CH}, \line{DI}, \line{EJ}$ 
are all concurrent. 
\end{theorem}
\begin{figure}[ht]
\caption{Diagram for Theorem~\ref{thm:sample affine}.}
\end{figure}
\begin{proof}
Everything in the theorem is preserved by affine transformations,
so we may place three 
of the points anywhere we want. Let us assume that $A, C, D$ form an 
isosceles triangle with $AC = AD$ and $\angle CDA = \pi/5$, which is 
to say that $A, C, D$ are three vertices of a regular pentagon.

Our first observation is that since $\triangle CJF$ and $\triangle DFG$ have equal areas, 
so do $\triangle CFG$ and $\triangle FDJ$, by adding the area of $\triangle CDH$ to both sides. By 
the base-height formula, this means $GJ$ is parallel to $CD$, and 
similarly for the other sides. In particular, since $\triangle ACD$ was assumed 
to be isosceles, $F$ lies on the internal angle bisector of $\ang DAC$,
and $J$ and $C$ are the reflections of $G$ and $D$ across $\line{AF}$.

We next want to show that $B$ and $E$ are mirror images 
across $\line{AF}$. To that end,
let $E'$ and $H'$ be the reflections of $E$ and $H$, respectively. 
Since the lines $\line{FC}$ and $\line{FD}$ are mirror images across
$\line{AF}$, we know that $E'$ lies on 
$\line{BD}$, and similarly $H'$ lies on $\line{AC}$. 
Suppose that $E'D < BD$, or 
equivalently that $E$ is closer than $B$ is to the line $\line{CD}$. Then we 
also have $DH' < CI'$; since $CJ = DG$, we deduce $JH' < JI$. Now it 
is evident that the triangle $E'H'J$, being contained in $\triangle BJI$, has 
smaller area; on the other hand, it has the same area as $\triangle EHG$, which 
by assumption has the same area as $\triangle BJI$, a contradiction. So we 
cannot have $E'D < BD$, or $E'D > BD$ by a similar argument. We 
conclude $E'D = BD$, i.e.\ $B$ and $E$ are mirror images across $\line{AF}$. 
\begin{figure}[ht]
\caption{Proof of Theorem~\ref{thm:sample affine}.}
\end{figure}

In particular, this implies that $\line{BE}$ is parallel to $\line{CD}$. Since we 
could just as well have put $B, D, E$ at the vertices of an isosceles 
triangle, we also may conclude $AC \parallel DE$ and so forth.

Now let $\ell$ be the line through 
$C$ parallel to $\line{AD}$; by the above argument, we know $B$ is the 
intersection of $\ell$ with $\line{DF}$. On the other hand, $B$ is also the 
intersection of $\ell$ with the line through $A$ parallel to $\line{CF}$. If 
we move $F$ towards $A$ along the internal angle bisector of $\ang DAC$,
the intersection of $\line{DF}$ with $\ell$ moves away from $C$, but the 
intersection of the parallel to $\line{CF}$ through $A$ with $\ell$ moves 
closer to $C$. Hence these can only coincide for at most one choice 
of $F$, and of course they do coincide when $ABCDE$ is a regular pentagon. 
We conclude that $ABCDE$ is the image of a regular pentagon under an 
affine transformation, which in particular implies that $\line{AF}, \line{BG}, 
\line{CH}, \line{DI}, \line{EJ}$ are concurrent.
\begin{figure}[ht]
\caption{Proof of Theorem~\ref{thm:sample affine}.}
\end{figure}
\end{proof}

Some authors choose to independently speak of  \emph{oblique
coordinates}, \index{oblique coordinates|textbf} \index{coordinates|oblique|textbf}
measured with respect to two coordinate axes which are not necessarily perpendicular. We prefer to think of these as the result of perfoming an affine transformation to a pair of Cartesian coordinate axes.

\begin{exer}
\ii \label{ex:characterize affine}
Prove that any transformation that takes collinear points to collinear points is an affine transformation.
\index{characterization!of affine transformations}
\index{affine transformation!characterization of}
\ii \label{ex:seventh2}
Let $\triangle ABC$ be a triangle, and let $X,Y,Z$ be points on sides
$\seg{BC}$, $\seg{CA}$, $\seg{AB}$, respectively, satisfying
\[
\frac{BX}{XC} = \frac{CY}{YA} = \frac{AZ}{ZB} = k,
\]
where $k$ is a given constant greater than 1.
Find, in terms of $k$,
the ratio of the area of the triangle formed by the three segments
$\seg{AX}$, $\seg{BY}$, $\seg{CZ}$
to the area of $\triangle ABC$. (Compare Problem~\ref{ex:seventh}, which is the 
case $k= 2$.)
\ii \label{ex:samearea}
In the hexagon $ABCDEF$, opposite sides are equal and parallel. Prove that
triangles $\triangle ACE$ and $\triangle BDF$ have the same area.
 \ii
(Greece, 1996)
In a triangle $ABC$ the points $D, E, Z, H, \Theta$ are the midpoints 
of the segments $\seg{BC}, \seg{AD}, \seg{BD}, \seg{ED}, \seg{EZ}$, 
respectively. If $I$ is the 
point of intersection of $\seg{BE}$ and $\seg{AC}$, and $K$ is the point of 
intersection of $\seg{H\Theta}$ and $\seg{AC}$, prove that
\begin{enumerate}
\ii $AK = 3 CK$;
\ii $HK = 3 H\Theta$;
\ii $BE = 3 EI$;
\ii the area of $\triangle ABC$ is 32 times that of $\triangle E\Theta H$.
\end{enumerate}
\ii (Sweden, 1996)
Through a point in the interior of a triangle with area $T$, draw lines
parallel to the three sides, partitioning the triangle into three triangles
and three parallelograms. Let $T_1, T_2, T_3$ be the areas of the three
triangles. Prove that
\[
\sqrt T = \sqrt{T_1} + \sqrt{T_2} + \sqrt{T_3}.
\]
\ii (France, 1996)
Let $\triangle ABC$ be a triangle. For any line $\ell$ not parallel to any side 
of $\triangle ABC$, let $G_{\ell}$ be the vector average of the intersections
of $\ell$ with $\line{BC}, \line{CA}, 
\line{AB}$. Determine the 
locus of $G_{\ell}$ as $\ell$ varies.
\end{exer}


\chapter{Tricks of the trade}

We conclude our presentation of fundamentals with a chapter that highlights a small core of basic techniques 
that prove useful in a large number of problems. The point is to show 
how much one can accomplish even with very little advanced knowledge.

\section{Slicing and dicing}

One of the most elegant ways of establishing a geometric result is 
to dissect \index{dissection}
the figure into pieces, then rearrange the pieces so 
that the result becomes obvious. The quintessential example of this 
technique is the proof of the Pythagorean theorem\footnote{The list of other
authors who have given proofs of the Pythagorean theorem in this vein is a long
one, but surely the oddest name on it is U.S. President James A. Garfield
(1831--1881).\index{Garfield, James A.}}
\index{Pythagorean theorem|textbf}
\index{theorem!Pythagorean|textbf}
given by
the Indian mathematician Bhaskara (Bhaskaracharya) (1114-1185),
\index{Bhaskara (Bhaskaracharya)}
which consists of a picture plus only one word.

\begin{theorem}[Pythagoras]
If $\triangle ABC$ is a right triangle with hypotenuse $\seg{BC}$, then $AB^{2} + 
BC^{2} = AC^{2}$.
\end{theorem}
\begin{proof}
Behold! 
\end{proof}


Other useful dissections include a proof of the fact that the area of 
a triangle
is half its base times its height (Figure~\ref{fig:baseheight}),
a proof that the median to the
hypotenuse of a right triangle divides the triangle into two isosceles
triangles (Figure~\ref{fig:medianhyp}),
and in three dimensions, an embedding of a tetrahedron in a
rectangular parallelepiped (Figure~\ref{fig:tetrainbox}).
\begin{figure}[ht]
\caption{Area of a triangle equals half base times height.}
\label{fig:baseheight}
\end{figure}
\begin{figure}[ht]
\caption{Median to the hypotenuse of a right triangle.}
\label{fig:medianhyp}
\end{figure}
\begin{figure}[ht]
\caption{A tetrahedron embedded in a box.}
\label{fig:tetrainbox}
\end{figure}


\begin{exer}
\ii
(MOP 1997)
Let $Q$ be a quadrilateral whose side lengths are $a,b,c,d$, in that
order. Show that the area of $Q$ does not exceed $(ac+bd)/2$.
\ii
Let $\triangle ABC$ be a triangle and $M_{A}, M_{B}, M_{C}$ the midpoints of the 
sides $\seg{BC}, \seg{CA}, \seg{AB}$, respectively. Show that the triangle 
with side 
lengths $AM_{A}, BM_{B}, CM_{C}$ has area $3/4$ that of 
$\triangle ABC$.
\ii \label{ex:seventh}
In triangle $\triangle ABC$, points $D, E, F$ are marked on sides $\seg{BC}, \seg{CA}, 
\seg{AB}$, 
respectively, so that
\[
\frac{BD}{DC} = \frac{CE}{EA} = \frac{AF}{FB} = 2.
\]
Show by a dissection argument that the triangle formed by the lines 
$\line{AD}, \line{BE}, \line{CF}$ has area $1/7$ 
that of $\triangle ABC$. (Compare Problem~\ref{ex:seventh2}.)
\ii
Give a dissection solution to Problem~\ref{ex:samearea}.
\ii
(For those familiar with space geometry)
The 1982 SAT (an American college entrance exam) 
\index{SAT (formerly Scholastic Aptitude Test)} included 
a question asking for
the number of faces of the polyhedron obtained by gluing 
a regular tetrahedron to a square pyramid along one of the triangular 
faces. The answer expected by the test authors was 7,
since the two polyhedra have 9 faces, 2 of which are removed by 
gluing. However, a student taking the exam pointed out that this
is incorrect! What is the correct answer, and why?
\ii \label{ex:tetocta}
(For those familiar with space geometry)
A regular tetrahedron and a regular octahedron have edges of the same
length. What is the ratio between their volumes?
\ii
Given four segments which form a convex cyclic quadrilateral of
a given radius, the same is true no matter what order the segments occur in.
Prove that the resulting quadrilaterals all have the same area.
(This will be evident later from Brahmagupta's formula;
\index{Brahmagupta's formula}
see Fact~\ref{fact:brahmagupta}.) 
\end{exer}

\section{Angle chasing}
\label{sec:angle chasing}

A surprising number of propositions in Euclidean geometry can be 
established using nothing more than careful 
bookkeeping of angles, which allows one to detect similar triangles, 
cyclic quadrilaterals, and the like. In some problem circles, this
technique is known as ``angle chasing''. \index{angle chasing|textbf}
\index{chasing (of angles)|see{angle chasing}}
As an example of angle chasing in action, we offer a theorem 
first published in 1838 by one A. Miquel. \index{Miquel!theorem|textbf}
\index{theorem!Miquel's|textbf}
\begin{figure}[ht]
\caption{Miquel's theorem.}
\label{fig:miquel}
\end{figure}
\begin{theorem}
Let $\triangle ABC$ be a triangle and let $P,Q,R$ be any points on the sides 
$\seg{BC}, \seg{CA}, \seg{AB}$, 
respectively. Then the circumcircles of triangles $ARQ, BPR, 
CQP$ pass through a common point.
\end{theorem}
\begin{proof}
Let $T$ be the second intersection (other than $R$)
of the circumcircles of $ARQ$ and $BPR$. 
By collinearity of points,
\[
\angle TQA = \pi - \angle CQT, \quad
\angle TRB = \pi - \angle ART, \quad
\angle TPC = \pi - \angle BPT.
\]
In a convex cyclic quadrilateral, opposite angles are supplementary. 
Therefore
\[
\angle TQA = \pi - \angle ART,
\angle TRB = \pi - \angle BPT.
\]
We conclude $\angle TPC = \pi - \angle CQT$. Now conversely, a 
convex quadrilateral whose opposite angles are supplementary is cyclic. 
Therefore $T$ also lies on the circumcircle of $\triangle CQP$, as desired.
\end{proof}

A defect of the angle chasing technique is that 
the relevant theorems depend on the configuration of the points 
involved, particularly on whether certain points fall between 
certain other points. For example, one might ask whether the above 
theorem still holds if $P,Q,R$ are allowed to lie on the extensions 
of the sides of $\triangle ABC$. It does hold, but the above proof breaks down 
because some of the angles claimed to be equal become supplementary, 
and vice versa. 

The trick to getting around this is to use ``mod $\pi$ directed angles''
\index{directed angle} \index{angle!directed}
as described in Section~\ref{sec:directed}. To illustrate how
``directed angle chasing'' works, we give an example which is both simple
and important: it intervenes in our proof of
Pascal's theorem (Theorem~\ref{thm:pascal}). \index{Pascal's theorem}
\index{theorem!Pascal's}
\begin{figure}[ht]
\caption{Diagram for Theorem~\ref{thm:parchord}.}
\end{figure}
\begin{theorem} \label{thm:parchord}
Suppose that the circles $\omega_1$ and $\omega_2$ intersect at
distinct points $A$ and $B$.
 Let $\seg{CD}$ be any chord on $\omega_{1}$, and let $E$ and 
$F$ be the second intersections of the lines $\line{CA}$ and $\line{BD}$, 
respectively, with $\omega_{2}$. Then $\line{EF}$ is parallel to $\line{CD}$.
\end{theorem}
\begin{proof}
We chase directed angles as follows:
\beqa
\dang CDF &=& \dang CDB \qquad \mbox{(collinearity of $B,D,F$)} \\
&=& \dang CAB \qquad \mbox{(cyclic quadrilateral $ABCD$)} \\
&=& \dang EAB \qquad \mbox{(collinearity of $A,C,E$)} \\
&=& \dang EFB \qquad \mbox{(cyclic quadrilateral $ABEF$)}.
\eeqa
Hence the lines $\line{CD}$ and $\line{EF}$ make the same angle with $\line{BF}$, 
and so are parallel.
\end{proof}

Directed angles can be expressed in terms of lines
as well as in terms of
points: the directed angle $\dang(\ell_1,\ell_2)$
between lines $\ell_1$ and $\ell_2$ can be interpreted as the angle of 
any rotation \index{rotation}
carrying $\ell_1$ to a line parallel to $\ell_2$.
 This alternate perspective simplifies some proofs, as in the 
following example; for a situation where this diagram occurs, see 
Problem~\ref{ex:postsim}.
\begin{theorem} \label{thm:presim}
Let $\triangle ABC$ be a triangle. Suppose that the lines $\ell_{1}$ and 
$\ell_{2}$ are perpendicular, and meet each side (or its extension) in a pair of points 
symmetric across the midpoint of the side. Then the intersection of 
$\ell_{1}$ and $\ell_{2}$ is concyclic with the midpoints of the 
three sides.
\end{theorem}
\begin{figure}[ht]
\caption{Diagram for Theorem~\ref{thm:presim}.}
\end{figure}
\begin{proof}
Let $M_{A}, M_{B}, M_{C}$ be the midpoints of the sides $\seg{BC}, \seg{CA}, 
\seg{AB}$, 
respectively, and put $P = \ell_{1} \cap \ell_{2}$. Since the lines 
$\ell_{1}, \ell_{2}, \line{BC}$ form a right triangle and $M_{A}$ is the 
midpoint of the hypotenuse of that triangle, 
the triangle formed by the points $P, 
M_{A}$, $\ell_{2} \cap BC$ is isosceles with 
\[
\dang(\line{M_{A}P}, \ell_{2}) = \dang(\ell_{2}, \line{BC}).
\]
\begin{figure}[ht]
\caption{Proof of Theorem~\ref{thm:presim}.}
\end{figure}
By a similar argument,
\[
\dang(\ell_{2}, \line{M_{B}P}) = \dang(\line{CA}, \ell_{2}),
\]
and adding these gives
\[
\dang M_{A}P M_{B} = \dang ACB = \dang M_{A}M_{C}M_{B}
\]
since the sides of the triangle $\triangle M_{A}M_{B}M_{C}$ are parallel to 
those of $\triangle ABC$. We conclude that $M_{A}, M_{B}, M_{C}, P$
are concyclic, as desired.
\end{proof}

\begin{exer}
\ii
(USAMO 1994/3)
A convex hexagon $ABCDEF$ is inscribed in a circle such that 
$AB=CD=EF$ and diagonals $\seg{AD},\seg{BE},\seg{CF}$ are 
concurrent. Let $P$ be the 
intersection of $\seg{AD}$ and $\seg{CE}$. Prove that $CP/PE = (AC/CE)^{2}$.
\ii
(IMO 1990/1)
Chords $\seg{AB}$ and $\seg{CD}$ of a circle intersect at a point $E$ inside the 
circle. Let $M$ be an interior point of the segment $\seg{EB}$. The tangent 
line of $E$ to the circle through $D,E,M$ intersects the lines $\line{BC}$ 
and $\line{AC}$ at $F$ and $G$, respectively. If $AM/AB = t$, find $EG/EF$ 
in terms of $t$.
\ii
Let $\triangle A_{0}B_{0}C_{0}$ be a triangle and $P$ a point. Define a new 
triangle whose vertices $A_{1}, 
B_{1}, C_{1}$ as the feet of the perpendiculars from $P$ to 
$B_{0}C_{0}, C_{0}A_{0}, A_{0}B_{0}$, respectively. Repeat the construction
twice, starting with $\triangle A_1B_1C_1$, to produce
the triangles $\triangle A_{2}B_{2}C_{2}$ and $\triangle A_{3}B_{3}C_{3}$. 
Show that 
$\triangle A_{3}B_{3}C_{3}$ is similar to $\triangle A_{0}B_{0}C_{0}$.
\ii (MOP 1991) %% MOP 1991/1995?
Two circles intersect at points $A$ and $B$. An arbitrary line 
through $B$ intersects the first circle again at $C$ and the second circle again
at $D$. The tangents to the first circle at $C$ and the second at $D$ intersect 
at $M$. Through the intersection of $\line{AM}$ and $\line{CD}$, 
there passes a line parallel
to $CM$ and intersecting $\line{AC}$ at $K$. Prove that $\line{BK}$ 
is tangent to the second circle.
\ii \label{ex:invcirc}
Let $\omega_1,\,\omega_2,\,\omega_3,\,\omega_4$ be four circles in the plane. 
Suppose that $\omega_1$ and $\omega_2$ intersect at $P_1$ and $Q_1$, 
$\omega_2$ and $\omega_3$ intersect at $P_2$ and $Q_2$,  
$\omega_3$ and $\omega_4$ intersect at $P_3$ and $Q_3$,
and $\omega_4$ and $\omega_1$ intersect at $P_4$ and $Q_4$.  Show that if
$P_1,\,P_2,\,P_3$, and $P_4$ lie on a line or circle, then
$Q_1,\,Q_2,\,Q_3$, and $Q_4$ also lie on a line or circle.
(This is tricky; see the proof of Theorem~\ref{thm:invcirc}.)
\end{exer}

\section{Working backward}

A common stratagem, when trying to prove that a given point has a
desired property, is to construct a phantom point with the desired
property, then reason backwards to show that it coincides with the
original point. We illustrate this point with an example.
\begin{figure}
\caption{Diagram for Theorem~\ref{thm:back1}.}
\end{figure}
\begin{theorem} \label{thm:back1}
Suppose the triangles $\triangle ABC$ and $\triangle
AB'C'$ are directly similar. Then 
the points $A, B, C$, $\line{BB'} \cap \line{CC'}$ lie on a circle.
\end{theorem}
\begin{proof}
Since we want to show that $\line{BB'} \cap \line{CC'}$ lies on the circle through
$A, B, C$, and analogously on the circle through $A, B', C'$, we
define the point $P$ to be the intersection of these two circles. Then
\[
\dang APB = \dang ACB = \dang AC'B' = \dang APB'
\]
and so $P$ lies on the line $\line{BB'}$, and similarly on the line 
$\line{CC'}$.
\end{proof}

\begin{exer}
\ii (IMO 1994/2) \label{ex:imo94}
Let $\triangle ABC$ be an isosceles triangle with $AB = AC$. Suppose that
\begin{enumerate}
        \item[(i)]  $M$ is the midpoint of $\seg{BC}$ and $O$ is the point on the line 
        $\line{AM}$ such that $\line{OB}$ is perpendicular to $\line{AB}$;

        \item[(ii)]  $Q$ is an arbitrary point on the segment $\seg{BC}$ different from 
        $B$ and $C$;

        \item[(iii)]  $E$ lies on the line $\line{AB}$ and $F$ lies on the line $\line{AC}$ 
        such that $E$, $Q$, $F$ are distinct and collinear.
\end{enumerate}
Prove that $\line{OQ}$ is perpendicular to $\line{EF}$ if and only if $QE = QF$.
\ii (USAMO 2005/3)
Let $\triangle ABC$ be an acute-angled triangle, and let $P$ and $Q$ be two
points on side $\seg{BC}$. Construct point $C_1$ in such a way that
convex quadrilateral $APBC_1$ is cyclic, $\line{QC_1}
\parallel \line{CA}$, and $C_1$ and $Q$ lie on opposite sides of line
$\line{AB}$. Construct point $B_1$ in such a way that convex
quadrilateral $APCB_1$ is cyclic, $\line{QB_1} \parallel \line{BA}$, and $B_1$
and $Q$  lie on opposite sides of line $\line{AC}$.  Prove that points
$B_1, C_1,P$, and $Q$ lie on a circle.
\ii (Morley's theorem) \label{ex:morley} \index{Morley's theorem|textbf}
\index{theorem!Morley's|textbf}
Let $\triangle ABC$ be a triangle, and for each side, draw the intersection of 
the two angle trisectors closer to that side. (That is, draw the 
intersection of the trisectors of $A$ and $B$ closer to $AB$, 
and so on.) Prove that these three intersections determine an 
equilateral triangle. 
\begin{figure}[ht]
\caption{Morley's theorem (Problem~\ref{ex:morley}).}
\end{figure}
\end{exer}

\part{Special situations}

\chapter{Concurrence and collinearity}

This chapter is devoted to the study of two fundamental and 
reciprocal questions: 
when are three given points collinear, and when are three given lines
concurrent or parallel? (The idea that parallel lines should be considered
``concurrent'' is an idea from the theory of perspective; it
will be fleshed out in our discussion of projective
geometry in Chapter~\ref{chap:projective}.)
This study only begins here; 
the themes of collinearity and concurrence recur throughout this text,
so it is worth mentioning places where they occur beyond this chapter.
\begin{itemize}
\item The Pascal and Brianchon theorems (Section~\ref{sec:pb}).
\index{Pascal's theorem} \index{theorem!Pascal's}
\item The radical axis theorem (Section~\ref{sec:radaxis}).
\index{radical axis theorem} \index{theorem!radical axis}
\end{itemize}

\section{Concurrent lines: Ceva's theorem}

We begin with a simple but useful result, published in 1678 by the Italian 
engineer Giovanni Ceva
(1647-1734). \index{Ceva, Giovanni}
In his honor, a segment 
joining a vertex of a triangle to a point on the opposite side is 
called a \emph{cevian}\footnote{Depending on who you ask, this word
is pronounced either CHAY-vee-un or CHEH-vee-un. I've heard other
pronunciations as well, but I don't recommend them.}.
\index{cevian (of a triangle)|textbf}
\begin{figure}[ht]
\caption{Ceva's theorem (Theorem~\ref{thm:ceva}).}
\end{figure}

\index{Ceva!theorem|textbf}
\index{theorem!Ceva's|textbf}
\begin{theorem}[Ceva] \label{thm:ceva}
Let $\triangle ABC$ be a triangle, and let 
$P, Q, R$ be points on the lines $\line{BC}, 
\line{CA}, \line{AB}$, respectively, none equal to any of
$A,B,C$. Then the lines 
$\line{AP}, \line{BQ}, \line{CR}$ are concurrent or
parallel if 
and only if 
\begin{equation} \label{eq:ceva}
\frac{BP}{PC} \frac{CQ}{QA} \frac{AR}{RB} = 1
\end{equation}
as an equality of signed ratios of lengths.
\index{ratio of lengths, signed (of collinear segments)} 
\index{signed ratio of lengths (of collinear segments)}
\end{theorem}
\begin{proof}
First suppose that  $\line{AP}, \line{BQ}, \line{CR}$ 
concur at a point $T$. Then 
the ratio of lengths $BP/PC$ is equal, by similar triangles, to the 
ratio of the distances from $B$ and $C$ to $\line{AP}$. On the other 
hand, that ratio is also equal to the ratio of areas $[ATB]/[CTA]$, since
we can calculate these areas as half of base times height, with $\seg{AT}$ 
as the base. Moreover, the signed ratios $BP/PC$ and 
$[ATB]_{\pm}/[CTA]_{\pm}$
also have the same sign, so are in fact equal.
\begin{figure}[ht]
\caption{Proof of Ceva's theorem (Theorem~\ref{thm:ceva}).}
\end{figure}

By this argument, we get
\[
\frac{BP}{PC} \frac{CQ}{QA} \frac{AR}{RB} = \frac{[ATB]_\pm}{[CTA]_\pm}
\frac{[BTC]_\pm}{[ATB]_\pm} \frac{[CTA]_\pm}{[BTC]_\pm} = 1.
\]
In case $\line{AP}, \line{BQ}, \line{CR}$ are parallel, we may deduce the
same conclusion by continuity, or directly: we leave this to the reader.

Conversely, suppose that (\ref{eq:ceva}) holds; we will apply the
trick of working backward. The lines $\line{AP}$ and 
$\line{BQ}$ meet at some point $T$, and the line $\line{CT}$ 
meets $\line{AB}$ at some point $R'$. (If $\line{AP}$ and $\line{BQ}$
are parallel, interpret $\line{CT}$ as the common parallel to these lines
through $C$, and the previous sentence will still make sense.)
By construction, $\line{AP}, \line{BQ}, \line{CR'}$ 
are concurrent. However, using Ceva in the other direction 
(which we just proved), we find that
\[
\frac{BP}{PC} \frac{CQ}{QA} \frac{AR'}{R'B} = 1.
\]
Combining this equation with (\ref{eq:ceva}) yields
\[
\frac{AR}{RB} = \frac{AR'}{R'B}.
\]
Since $AR + RB = AR' + R'B = AB$, adding 1 to both sides gives
\[
\frac{AB}{RB} = \frac{AB}{R'B}
\]
as a signed ratio of lengths,
from which we conclude that $RB = R'B$, and hence $R = R'$.
\end{proof}

In certain cases, Ceva's Theorem is more easily applied in the 
following form (``trig Ceva''). \index{Ceva's theorem!trigonometric form|textbf}
\index{trig Ceva|textbf}
\begin{fact}[Ceva's theorem, trigonometric form]
Let $\triangle ABC$ be a triangle, and let $P$, $Q$, $R$ be any points in the 
plane distinct from $A, B, C,$ respectively. Then $\line{AP}, \line{BQ}, 
\line{CR}$ are 
concurrent if and only if
\[
\frac{\sang CAP}{\sang PAB} \frac{\sang ABQ}{\sang QBC}
\frac{\sang BCR}{\sang RCA} = 1.
\]
\end{fact}
One can either deduce this from Ceva's theorem or prove it directly.
Be careful when using trig Ceva
with directed angles, as signs matter: the ratio $(\sang CAP)/(\sang PAB)$
must be defined in terms of angles modulo $2\pi$, but the sign of the
ratio itself only depends on the line $\line{AP}$, not on the choice of $P$
on one side or the other of $A$.

\begin{exer}
\ii
Suppose that the cevians $\seg{AP}, \seg{BQ}, \seg{CR}$ meet at $T$. Prove that
\[
\frac{TP}{AP} + \frac{TQ}{BQ} + \frac{TR}{CR} = 1.
\]
\ii
Let $\triangle ABC$ be a triangle, and 
let $D, E, F$ be points on sides $\seg{BC}, \seg{CA}, \seg{AB}$,
respectively, such that the cevians $\seg{AD}, \seg{BE}, \seg{CF}$ 
are concurrent. Show
that if $M, N, P$ are points on $\seg{EF}, \seg{FD}, \seg{DE}$, 
respectively, then the
lines $\line{AM}, \line{BN}, \line{CP}$ concur if and only if the
 lines $\line{DM}, \line{EN}, \line{FP}$
concur.
(Many special cases of this question occur in the problem literature.)
\ii (Hungary-Israel, 1997)
The three squares $ACC_1A'', ABB_1'A', BCDE$ are constructed externally
on the sides of a triangle $\triangle 
ABC$. Let $P$ be the center of $BCDE$. Prove that
the lines $\line{A'C}, \line{A''B}, \line{PA}$ are concurrent.
\ii (USAMO 1995/3)
Given a nonisosceles, nonright triangle $\triangle ABC$ inscribed in a
circle with center $O$, let
$A_1,B_1,C_1$ be the midpoints of
sides $\seg{BC}, \seg{CA}, \seg{AB}$ respectively.
Point $A_2$ is located on the ray
$\ray{OA_1}$ so that $\triangle OAA_1$ is similar to
$\triangle OA_2A$.  Points $B_2, C_2$ on rays $\ray{OB_1}, \ray{OC_1}$
respectively, are defined similarly.  
Prove that the lines $\line{AA_2}, \line{BB_2}, \line{CC_2}$ are concurrent.
\ii
Given a triangle $\triangle ABC$ and points $X, Y, Z$ such that 
$\ang ABZ = \ang XBC$, $\ang BCX = \ang YCA$, $\ang CAY = \ang ZAB$,
prove that the lines $\line{AX}, \line{BY}, \line{CZ}$ 
are concurrent. (Again, many special cases
of this problem can be found in the literature.)
\ii
Let $A, B, C, D, E, F,P$ be seven points on a circle.
Show that the lines $\line{AD}, \line{BE},
\line{CF}$ are concurrent if and only if
\[
\frac{\sang APB \sang CPD \sang EPF}{\sang BPC \sang DPE \sang FPA} = -1,
\]
where the angles are measured 
modulo $2\pi$. (The only tricky part is the sign.)
\end{exer}

\section{Collinear points: Menelaus's theorem}

When he published his theorem, Ceva \index{Ceva, Giovanni}
also revived interest in an ancient 
theorem attributed to Menelaus\footnote{Not to be confused with
the brother of Agamemnon \index{Agamemnon} in \index{Homer}
\index{Iliad@\textit{Iliad}} Homer's \textit{Iliad}.} 
\index{Menelaus} (70?-130?).
\begin{figure}[ht]
\caption{Menelaus's theorem (Theorem~\ref{thm:menelaus}).}
\end{figure}

\index{theorem!Menelaus's|textbf}
\index{Menelaus's theorem|textbf}
\begin{theorem}[Menelaus] \label{thm:menelaus}
Let $\triangle
ABC$ be a triangle, and let $P, Q, R$ be points on the lines $\line{BC}, 
\line{CA}, \line{AB}$, respectively, none equal to any of $A,B,C$. 
Then $P, Q, R$ are collinear if and only if
\[
\frac{BP}{PC} \frac{CQ}{QA} \frac{AR}{RB} = -1
\]
as an equality of signed ratios of lengths.
\index{ratio of lengths, signed (of collinear segments)} 
\index{signed ratio of lengths (of collinear segments)}
\end{theorem}
\begin{proof}
Assume that $P,Q,R$ are collinear.
Let $x,y,z$ be the directed distances from $A,B,C$, 
respectively, to the line $\line{PQR}$. 
\begin{figure}[ht]
\caption{Menelaus's theorem (Theorem~\ref{thm:menelaus}).}
\end{figure}
Then $BP/PC = -y/z$ and so forth, so
\[
\frac{BP}{PC} \frac{CQ}{QA} \frac{AR}{RB} = (-1)(-1)(-1) \frac{y}{z} 
\frac{z}{x} \frac{x}{y} = -1.
\]
The converse follows by the same argument as for Ceva's theorem.
\end{proof}

An important consequence of Menelaus's theorem is the following result 
of Desargues \index{Desargues, Girard} (for more on whom see
the introduction to Chapter~\ref{chap:projective}), 
which is most easily stated by introducing two pieces of 
terminology. Two triangles $\triangle 
ABC$ and $\triangle DEF$ are said to be 
\emph{perspective from a point} \index{perspective triangles!from a point|textbf}
if the lines $\line{AD}, \line{BE}, 
\line{CF}$ are 
concurrent or parallel. 
The triangles are said to be \emph{perspective from a line}
\index{perspective triangles!from a line|textbf}
if the 
points $\line{AB} \cap \line{DE}, \line{BC} \cap 
\line{EF}, \line{CA} \cap \line{FD}$ are collinear.

\index{Desargues's theorem|textbf} 
\index{theorem!Desargues's|textbf}
\begin{theorem}[Desargues] \label{thm:desargues}
Two triangles $\triangle ABC$ 
and $\triangle DEF$ are perspective from a point if and only 
if they are perspective from a line.
\end{theorem}
\begin{proof}
We only prove that if $\triangle ABC$ and $\triangle DEF$ are 
perpective from a point, 
then they are perspective from a line. We leave it as an exercise to 
deduce the reverse implication from this (stare at the diagram); we 
will do this again later, using duality.
\begin{figure}[ht]
\caption{Proof of Desargues's theorem (Theorem~\ref{thm:desargues}).}
\end{figure}

Suppose that $\line{AD}, \line{BE}, \line{CF}$ concur at $O$, and put
$P = \line{BC} \cap \line{EF}$, 
$Q = \line{CA} \cap \line{FD}$, $R = \line{AB} \cap \line{DE}$. 
To show that $P, Q, R$ are 
collinear, we want to show that
\[
\frac{AR}{RB} \frac{BP}{PC} \frac{CQ}{QA} = -1
\]
and then invoke Menelaus's theorem. To get hold of the first term, we 
apply Menelaus to the points $R, D, E$ on the sides of the triangle 
$\triangle OAB$, giving
\[
\frac{AR}{RB} \frac{BD}{DO} \frac{OE}{EA} = -1.
\]
Analogously,
\[
\frac{BP}{PC} \frac{CE}{EO} \frac{OF}{FB} =
\frac{CQ}{QA} \frac{AF}{FO} \frac{OD}{DC} = -1.
\]
When we multiply these three expressions together and cancel equal terms, 
we get precisely the condition of Menelaus's theorem.
\end{proof}

Another important consequence of Menelaus's theorem is the following result 
of Pappus of Alexandria (290?-350?). \index{Pappus!of Alexandria}
\index{Pappus!theorem|textbf} \index{theorem!Pappus|textbf}
\begin{theorem}[Pappus] \label{thm:pappus}
Let $A,C,E$ be three collinear points, and let $B,D,F$ be
three other collinear points. Then the points $\line{AB} \cap \line{DE}$, 
$\line{BC} \cap 
\line{EF}$, $\line{CD} \cap \line{FA}$ are collinear. 
\end{theorem}
The proof is similar, but more complicated; we omit it, save 
to say that one applies Menelaus repeatedly using the triangle formed 
by the lines $\line{AB}, \line{CD}, \line{EF}$. If
you can't make the cancellation work, see \cite{bib:cg}.

Note that Desargues's and Pappus's theorems only involve points 
and lines, with no mention of distances or angles. This makes them 
``theorems of projective geometry,'' and we will see later 
(Section~\ref{sec:proj trans}) how 
projective transformations often yield simple proofs of 
such theorems.

\begin{exer}
\ii
Prove Pappus's theorem (Theorem~\ref{thm:pappus}) directly in terms of 
Cartesian coordinates; the hope is that you will find this doable but not 
pleasant!
\ii \label{ex:des2}
Prove the reverse implication of Desargues' theorem.
\index{Desargues!theorem} \index{theorem!Desargues's}
\ii \label{ex:harmcon}
Let $A,B,C$ be three points on a line. Pick a point $D$ in the plane, 
and a point $E$ on $\line{BD}$. Then draw the line through $\line{AE} \cap 
\line{CD}$ and 
$\line{CE} \cap \line{AD}$. 
Show that this line meets the line $\line{AC}$ in a point $P$ 
that depends only on $A,B,C$. (The points $A,B,C,P$ are in fact harmonic
conjugates, \index{harmonic conjugates}
for more on which see Section~\ref{sec:cross-ratio}.)
\ii (MOP 1990)
Let $A, B, C$ be three collinear points and $D, E, F$ three other 
collinear points. Put $G = \line{BE} \cap \line{CF}$, 
$H = \line{AD} \cap \line{CF}$, $I = \line{AD} \cap 
\line{CE}$. If $AI = HD$ and $CH = GF$, prove that $BI = GE$.
\ii (Original) \label{ex:apcon1}
Let $\triangle ABC$ be a triangle and 
let $P$ be a point in its interior, not lying on 
any of the medians of $\triangle ABC$. 
Put $A_1 = \line{PA} \cap \seg{BC}$,
$B_1 = \line{PB} \cap \seg{CA}$,
$C_1 = \line{CA} \cap \seg{AB}$,
$A_2 = \line{B_1C_1} \cap \line{BC}$,
$B_2 = \line{C_1A_1} \cap \line{CA}$,
$C_2 = \line{A_2B_2} \cap \line{AB}$.
Prove that the 
midpoints of $\seg{A_1A_2}, \seg{B_1B_2}, \seg{C_1C_2}$ 
are collinear. (See also
Problem~\ref{ex:apcon}.)
\ii (Aaron Pixton) \index{Pixton, Aaron}
Let $\triangle ABC$ 
and $\triangle DEF$ be triangles, and let $P$ be a point.
For each nonzero real number $r$, let $T_r$ be the triangle obtained
from $\triangle ABC$ by a homothety about $P$ of ratio $r$. Suppose that for
some three distinct nonzero real numbers
$r_1, r_2, r_3$, each of $T_{r_1}, T_{r_2}, T_{r_3}$ is perspective
with $\triangle DEF$. Prove that $T_r$ and $\triangle DEF$ are perspective for any $r$.
\end{exer}

\section{Concurrent perpendiculars}

Some of the special points of a triangle are constructed by drawing 
perpendiculars to the sides of a triangle. For example, the 
circumcenter can be constructed by drawing the perpendicular 
bisectors. It is convenient that a result analogous to Ceva's Theorem 
holds for perpendiculars; the analogy is so strong that we can safely 
leave the proof to the reader (see Problem~1). 
\begin{figure}
\caption{Fact~\ref{thm:concperp}.}
\end{figure}

\begin{fact} \label{thm:concperp}
Let $\triangle ABC$ be a triangle, and let $P, Q, R$ be three points in the 
plane. Then the lines through $P, Q, R$ 
perpendicular to $\line{BC}$, $\line{CA}$, $\line{AB}$, 
respectively, are concurrent or parallel if and 
only if
\[
BP^{2} - PC^{2} + CQ^{2} - QA^{2} + AR^{2} - RB^{2} = 0.
\]
\end{fact}
A surprising consequence is that the lines through $P,Q,R$ 
perpendicular to $\line{BC}$, $\line{CA}$, $\line{AB}$, 
respectively, are concurrent or parallel
if and only if the lines through $A,B,C$ perpendicular to 
$\line{QR}$, 
$\line{RP}$, 
$\line{PQ}$, respectively, are concurrent or parallel!

\begin{exer}
\ii \label{ex:perp criterion}
Prove that the lines $\line{AB}$ and $\line{CD}$ 
are perpendicular if and only if 
$AC^{2} - AD^{2} = BC^{2} - BD^{2}$. (Use vectors, coordinates 
or Pythagoras.) Then prove Fact~\ref{thm:concperp}.
\ii (Germany, 1996)
Let $\triangle ABC$ be a triangle, and construct squares $ABB_1A_2$, 
$BCC_1B_2$,
$CAA_1C_2$ externally on its sides. Prove that the perpendicular bisectors
of the segments $\seg{A_1A_2}$, $\seg{B_1B_2}$, $\seg{C_1C_2}$ are concurrent.
\ii
Let $\triangle ABC$ be a triangle, $\ell$ a line and
$L,M,N$ the feet of the perpendiculars to $\ell$ from $A,B,C$, 
respectively. Prove that the perpendiculars to $\line{BC}$, $\line{CA}$,
$\line{AB}$ 
through 
$L,M,N$, respectively, are concurrent. (Their intersection is called 
the \emph{orthopole} \index{orthopole|textbf}
of the line $\ell$ and the triangle $ABC$.)
\ii
(USAMO 1997/2) \label{ex:usa97}
Let $\triangle ABC$ be a triangle, and draw isosceles triangles $\triangle 
DBC$, 
$\triangle AEC$, $\triangle ABF$ external to $\triangle ABC$ (with 
$\seg{BC}, \seg{CA}, \seg{AB}$ as their respective 
bases). Prove that the lines through $A,B,C$ perpendicular to 
$\line{EF}$, $\line{FD}$, $\line{DE}$, 
respectively, are concurrent. (Several solutions are
possible.)
\end{exer}

\chapter{Circular reasoning}

This chapter is of course devoted not to logical fallacies, but to
reasoning about the most fundamental of geometric objects, the circle.
Note that we will gain further insight into the geometry of circles after
introducing inversion in Chapter~\ref{chap:inversion}.

\section{Power of a point}
\label{sec:powerofapoint}

The following is a theorem of Euclidean geometry in the strictest of senses:
it appears in the \textit{Elements}
\index{Elements@\textit{Elements} (of Euclid)}
\index{Euclid's \textit{Elements}}
as Propositions III.35--III.37.

\index{power of a point!theorem|textbf}
\begin{theorem} \label{thm:powerofapoint}
Given a fixed circle $\omega$ and a fixed point $P$, draw a line through $P$ 
intersecting $\omega$ at $A$ and $B$. Then the product $PA \cdot PB$ 
depends only on $P$ and $\omega$, not on the line.
\end{theorem}
\begin{proof}
Draw another line through $P$ meeting $\omega$ at $C$ and $D$, 
labeled as in one of the diagrams. 
\begin{figure}[ht]
\caption{Proof of the power of a point theorem 
(Theorem~\ref{thm:powerofapoint}).}
\end{figure}
Then
\[
\dang PAC = \dang BAC = \dang BDC  = -\dang PDB
\]
as directed angles, so the triangles $\triangle PAC$ and $\triangle PDB$ 
are (oppositely) 
similar, giving $PA/PD = PC/PB$, or equivalently $PA\cdot PB = PC\cdot 
PB$.
\end{proof}

We may view $PA \cdot PB$ as a signed quantity by the same convention
as for signed ratios of lengths.
\index{ratio of lengths, signed (of collinear segments)} 
\index{signed ratio of lengths (of collinear segments)}
This signed quantity
is called the \emph{power} of $P$ \index{power of a point|textbf}
with respect to $\omega$; note that it is positive if $P$ lies outside
$\omega$, zero if $P$ lies on $\omega$, and negative if $P$ lies inside
$\omega$.
If $O$ is the center of $\omega$ and $r$ is the radius, we may
choose $\line{OP}$ as our line and so express the power as
\[
(OP + r)(OP - r) = OP^{2} - r^{2}.
\]
Note that
for $P$ outside $\omega$, the limiting case $A=B$ 
means that $\line{PA}$ is tangent to $\omega$ at $A$.

The power of a point theorem has an occasionally useful converse.
\index{power of a point!converse}
\begin{fact} \label{fact:power}
If the lines $\line{AB}$ 
and $\line{CD}$ meet at $P$, and there is an equality
$PA \cdot PB = PC \cdot PD$ of signed products of lengths,
then $A,B,C,D$ are concyclic.
\end{fact}

\begin{exer}
\ii
If $A,B,C,D$ are concyclic and $\line{AB} \cap \line{CD} = E$, prove that 
$(AC/BC)(AD/BD) = AE/BE$.
\ii
(\textit{Mathematics Magazine}, Dec. 1992)
Let $\triangle ABC$ be an acute triangle, let $H$ be the foot of the altitude 
from $A$, and let $D, E, Q$ be the feet of the perpendiculars from an 
arbitrary point $P$ in the triangle onto $\seg{AB}, \seg{AC}, \seg{AH}$, 
respectively. 
Prove that
\[
|AB \cdot AD - AC \cdot AE| = BC \cdot PQ.
\]
\ii
Draw tangents $\seg{OA}$ and $\seg{OB}$ from a point $O$ to a given circle. 
Through $A$ is drawn a chord $\seg{AC}$ parallel to $\line{OB}$; 
let $E$ be the 
second intersection of $\line{OC}$ with the circle. Prove that the line 
$\line{AE}$ bisects 
the segment $\seg{OB}$.
\ii (MOP 1995)
Given triangle $\triangle ABC$, let $D, E$ be any points on $\seg{BC}$.
 A circle
through $A$ cuts the lines $\line{AB}, \line{AC}, \line{AD}, \line{AE}$ 
at the points $P, Q, R,
S$, respectively. Prove that
\[
\frac{AP \cdot AB - AR \cdot AD}{AS \cdot AE - AQ \cdot AC} =
\frac{BD}{CE}.
\]
\ii (IMO 1995/1)
Let $A,B,C, D$ be four distinct points on a line, in that order. The 
circles with diameters $\seg{AC}$ and $\seg{BD}$ 
intersect at $X$ and $Y$. 
The line $\line{XY}$ meets $\line{BC}$ at $Z$. Let $P$ be a point on the line 
$\line{XY}$ other than $Z$. The line $\line{CP}$ 
intersects the circle with diameter 
$\seg{AC}$ at $C$ and $M$, and the line $\line{BP}$ intersects the circle 
with diameter $\seg{BD}$ at $B$ and $N$. Prove that the lines $\line{AM}, 
\line{DN}, \line{XY}$ are concurrent.
\ii (USAMO 1998/2)
Let $\omega_1$ and $\omega_2$ be  concentric circles, with
$\omega_2$ in the interior of  $\omega_1$. From a point $A$
on $\omega_1$ one draws the tangent $\seg{AB}$ to $\omega_2$ ($B\in \omega_2$).
Let $C$ be the second point of intersection 
of $\line{AB}$ and $\omega_1$, and 
let   $D$ be the midpoint of 
$\seg{AB}$. A line passing through $A$ intersects $\omega_2$
at $E$ and $F$ in such a way that the perpendicular  bisectors of 
$\seg{DE}$ and $\seg{CF}$ intersect at a point $M$ on $\seg{AB}$.
Find, with proof,  the ratio $AM/MC$.
\end{exer}

\section{Radical axis}
\label{sec:radaxis}

Given two circles, one with center $O_{1}$ and radius $r_{1}$, the 
other with center $O_{2}$ and radius $r_{2}$, what is the set of 
points with equal power with respect to the two circles? By our 
explicit formula for the power of a point, this is simply the set of 
points $P$ such that $PO_{1}^{2} - r_{1}^{2} = PO_{2}^{2} - 
r_{2}^{2}$, or equivalently such that $PO_{1}^{2} - PO_{2}^{2} = 
r_{1}^{2} - r_{2}^{2}$. By Problem~\ref{ex:perp criterion}, 
this set is a straight line 
perpendicular to $\line{O_{1}O_{2}}$; we call this line the \emph{radical 
axis} of the two circles. \index{radical axis|textbf}

\index{radical axis!theorem|textbf}
\index{theorem|radical axis|textbf}
\begin{theorem}[Radical axis theorem]
Let $\omega_{1}, \omega_{2}, \omega_{3}$ be three circles. Then the 
radical axes of $\omega_{1}$ and $\omega_{2}$, of $\omega_{2}$ and 
$\omega_{3}$, and of $\omega_{3}$ and $\omega_{1}$ either all 
coincide, or are concurrent (or parallel).
\end{theorem}
\begin{proof}
A point on two of the radical axes has equal power with respect to all 
three circles. Hence if two of the axes coincide, so does the third, 
and otherwise if any two of the axes have a common point, this point 
lies on the third axis as well.
\end{proof}
\begin{corr}
The common chords of three mutually intersecting circles lie on 
concurrent lines.
\end{corr}

If the radical axes coincide, the three circles are said to be 
\emph{coaxial}\footnote{The word is also spelled ``coaxal'', as in
\cite{bib:cg}.} \index{coaxial (coaxal) circles|textbf}; 
\index{circle!coaxial (coaxal)|textbf}
otherwise, the intersection of the three radical axes 
is called the \emph{radical center} \index{radical center (of three circles)|textbf}
\index{center!radical (of three circles)|textbf}
of the circles. (As usual, this 
intersection could be ``at infinity'', if the three lines are 
parallel.) There are three 
types of coaxial families, depending on whether the circles have zero, 
one, or two intersections with the common radical axis; these three 
cases are illustrated in Figure~\ref{fig:coaxial}.
(Note: the zero and two cases become identical in the complex
projective plane; see Section~\ref{sec:alggeo}.)
\begin{figure}[ht]
\caption{Some coaxial families of circles.}
\label{fig:coaxial}
\end{figure}
A useful
criterion for recognizing and applying the coaxial property is the
following simple observation and partial converse.
\begin{fact}
If three circles are coaxial, their centers are collinear. Conversely,
if three circles pass through a common point and have collinear
centers, they are coaxial.
\end{fact}

Like the power-of-a-point theorem, the radical axis theorem has an 
occasionally useful converse.
\index{radical axis!converse}
\begin{fact} \label{fact:radaxconv}
Suppose that $ABCD$ and $CDEF$ are cyclic quadrilaterals, and the lines 
$\line{AB}, \line{CD}, \line{EF}$ 
are concurrent. Then $EFAB$ is also cyclic. More generally,
if $\omega_1,\omega_2$ are two circles with radical axis $\ell$,
$A,B$ are points on $\omega_1$, $C,D$ are points on $\omega_2$, and
$\line{AB}$ and $\line{EF}$ 
meet at a point on $\ell$, then $A,B,E,F$ are concyclic.
(We may allow $A=B$ by taking $\line{AB}$ to be the tangent line to $\omega_1$
at that point, and likewise we may allow $C=D$.)
\end{fact}

\begin{exer}
\ii \label{ex:radaxconv}
Prove Fact~\ref{fact:radaxconv}. (Hint: draw a third circle and
apply the radical axis theorem.)

\ii
Use the radical axis theorem to give another solution for 
Problem~\ref{ex:usa97}.

\ii (MOP 1995)
Let $\seg{BB'}, \seg{CC'}$ 
be altitudes of triangle $\triangle ABC$, and assume $AB \neq
AC$. Let $M$ be the midpoint of $\seg{BC}$, $H$ the orthocenter of 
$\triangle ABC$,
and $D$ the intersection of $\line{BC}$ and $\line{B'C'}$. Show that 
$\line{DH}$ is
perpendicular to $\line{AM}$.

\ii
(IMO 1994 proposal) %% Also inversion
A circle $\omega$ is tangent to two parallel lines $\ell_{1}$ and 
$\ell_{2}$. A second circle $\omega_{1}$ is tangent to $\ell_{1}$ at 
$A$ and to $\omega$ externally at $C$. A third circle $\omega_{2}$ is 
tangent to $\ell_{2}$ at $B$, to $\omega$ externally at $D$ and to 
$\omega_{1}$ externally at $E$. Let $Q$ be the intersection of $\line{AD}$ 
and $\line{BC}$. Prove that $QC = QD = QE$.

\ii
(India, 1995)
Let $\triangle
ABC$ be a triangle. A line parallel to $\line{BC}$ meets sides 
$\seg{AB}$ and $\seg{AC}$ 
at $D$ and $E$, respectively. Let $P$ be a point inside triangle 
$\triangle ADE$, and let $F$ and $G$ be the intersection of $\line{DE}$ 
with $\line{BP}$ and 
$\line{CP}$, respectively. Show that $A$ lies on the radical axis of the 
circumcircles of $\triangle PDG$ and $\triangle PFE$.

\ii \label{ex:imo85}
(IMO 1985/5)
A circle with center $O$ passes through the vertices $A$ and $C$ of 
triangle $\triangle
ABC$, and intersects the segments $\seg{AB}$ and $\seg{BC}$ again at 
distinct points $K$ and $N$, respectively. The circumscribed circles 
of the triangle $\triangle ABC$ and $\triangle
KBN$ intersect at exactly two distinct 
points $B$ and $M$. Prove that angle $\angle OMB$ is a right angle. 

\ii \label{ex:tst04}
(TST 2004/4)
Let $\triangle ABC$ be a triangle, and let
$D$ be a point in its interior. Construct a circle $\omega_1$
passing through $B$ and $D$, and a circle $\omega_2$ passing through
$C$ and $D$, such that the point of intersection of $\omega_1$ and
$\omega_2$ other than $D$ lies on the line $\line{AD}$. Denote by
$E, F$ the points where $\omega_1, \omega_2$ intersect side
$\seg{BC}$, respectively, and by $X, Y$ the intersections
$\line{DF} \cap \line{AB}, \line{DE} \cap \line{AC}$, respectively.
Prove that $\line{XY}$ is parallel to $\line{BC}$.
\end{exer}

\section{The Pascal-Brianchon theorems} \label{sec:pb}

Although Blaise Pascal (1623--1662) \index{Pascal, Blaise} is 
most famous for ``Pascal's triangle\footnote{The famous triangle
was actually known in ancient China. However, Pascal investigated
the triangle much more deeply, in his foundational work on probability
theory.}'', he also left behind an amazing theorem about hexagons
inscribed in circles.
His original proof, which was favorably described 
by calculus pioneer Gottfried Wilhelm von Leibniz (1646--1716),
\index{Leibniz, Gottfried Wilhelm von}
has unfortunately been lost; we present here an ingenious 
proof essentially due to
Jan van Yzeren \index{van Yzeren, Jan}
(A simple proof of Pascal's hexagon theorem, 
\textit{Monthly}, December 1993).
%% AMM 100 (Dec 1993) 930-931

\begin{figure}
\caption{Pascal's theorem (Theorem~\ref{thm:pascal}) and van Yzeren's proof.}
\label{fig:pascal}
\end{figure}

\index{Pascal's theorem|textbf} \index{theorem!Pascal's|textbf}
\begin{theorem}[Pascal] \label{thm:pascal}
Let $ABCDEF$ be a hexagon inscribed in a circle. Then the 
intersections $\line{AB} \cap \line{DE}$, $\line{BC} \cap \line{EF}$,
$\line{CD} \cap \line{FA}$
are collinear.
\end{theorem}
\begin{proof}
Put $P = \line{AB} \cap \line{DE}$, $Q = \line{BC} \cap \line{EF}$, 
$R = \line{CD} \cap \line{FA}$. Draw the
circle $\omega$ through $C, F, R$,
and extend the lines $\line{BC}$ and $\line{EF}$ to meet this
circle again at $G$ and $H$, respectively; see Figure~\ref{fig:pascal}. By
Theorem~\ref{thm:parchord}, we have $\line{BE} \parallel \line{GH}$, 
$\line{ED} \parallel \line{HR}$, $\line{AB} \parallel
\line{RG}$.

Now notice that the triangles $\triangle RGH$ and $\triangle PBE$ 
have parallel sides, 
which means that they are homothetic. \index{homothety}
In other words, the lines $\line{BG}, 
\line{EH}, 
\line{PR}$ are concurrent, which means $\line{BG} \cap \line{EH} =
 Q$ is collinear with 
$P$ and $R$, as desired.
\end{proof}

Some time later, Charles Brianchon (1783--1864) \index{Brianchon, Charles}
discovered a
counterpart to Pascal's theorem \index{Pascal's theorem} \index{theorem!Pascal's}
for a hexagon circumscribed about a 
circle. We will give Brianchon's proof of his theorem, which uses the
polar map to reduce it to Pascal's theorem, in Section~\ref{sec:polar}; a
direct but somewhat complicated proof can be found in \cite{bib:cg}.
\index{Brianchon's theorem|textbf}
\index{theorem!Brianchon's|textbf}
\begin{theorem}[Brianchon] \label{thm:brianchon}
Let $ABCDEF$ be a hexagon circumscribed about a circle (i.e., the extension
of each side is tangent to the circle). Then the lines 
$\line{AD}, \line{BE}, \line{CF}$ are concurrent.
\end{theorem}
\begin{figure}[ht]
\caption{Brianchon's theorem (Theorem~\ref{thm:brianchon}).}
\label{fig:brianchon}
\end{figure}

It is sometimes useful to apply Pascal's theorem or
Brianchon's theorem in certain degenerate 
cases, in which some of the vertices coincide. For example, 
in Pascal's theorem, if two adjacent vertices of the hexagon coincide,
one should take the line through them to be the tangent to the circle at 
that point. Thus in Figure~\ref{fig:pascal degen},
\begin{figure}[ht]
\caption{A degenerate case of Pascal's theorem (Theorem~\ref{thm:pascal}).}
\label{fig:pascal degen}
\end{figure}
we may conclude that $\line{AA} \cap \line{DE}$, $\line{AC} \cap \line{EF}$, 
$\line{CD} \cap \line{FA}$ are collinear, where 
$\line{AA}$ denotes the tangent to the circle at $A$.

As for Brianchon's theorem, 
the analogous argument shows that the ``vertex'' between two 
collinear sides belongs at the point of tangency, as in
Figure~\ref{fig:brianchon degen}.
\begin{figure}[ht]
\caption{A degenerate case of Brianchon's theorem 
(Theorem~\ref{thm:brianchon}).}
\label{fig:brianchon degen}
\end{figure}

\begin{exer}
\ii
What do we get if we apply Brianchon's theorem with three degenerate 
vertices? (We will encounter this fact again later.)
\ii
Let $ABCD$ be a quadrilateral whose sides $\seg{AB}$, $\seg{BC}$, $\seg{CD}$,
$\seg{DA}$ are tangent to a single circle at 
points $M,N,P,Q$, respectively. Prove that the lines $AC, 
BD, MP, NQ$ are concurrent.
\ii 
(MOP 1995)
With notation as in the previous problem, let
lines $\line{BQ}, \line{BP}$ intersect the circle at $E,F$, 
respectively. Prove that $\line{ME}$, $\line{NF}$, $\line{BD}$
are concurrent.
\ii (Poland, 1997) \label{ex:pol1997}
Let $ABCDE$ be a convex pentagon with $CD = DE$ and $\angle 
BCD = \angle DEA = \pi/2$. Let $F$ be the point on side $\seg{AB}$ such 
that $AF/FB = AE/BC$. Show that
\[
\ang FCE = \ang FDE \quad \mbox{and} \quad \ang FEC = \ang BDC.
\]
\end{exer}

\section{Simson lines}

The following theorem is often called Simson's theorem in honor
of Robert Simson (1687--1768), \index{Simson!Robert}
but it is actually originally due to
William Wallace (1768--1843). \index{Wallace, William}

\index{Simson's theorem|textbf} \index{theorem!Simson's|textbf}
\begin{theorem}
Let $A,B,C$ be three points on a circle. Then the feet of the 
perpendiculars from $P$ to the lines $\line{AB}, \line{BC}, \line{CA}$ 
are collinear if  and only if $P$ also lies on the circle.
\end{theorem}
\begin{proof}
The proof is by (directed) angle-chasing. Let $X,Y,Z$ be the feet of 
the respective perpendiculars from $P$ to $\line{BC}, \line{CA}, \line{AB}$; 
then the
quadrilaterals $PXCY, PYAZ, PZBX$ each have two right angles, and are 
thus cyclic. Therefore
\beqa
\dang PXY &=& \dang PCY \qquad \mbox{(cyclic quadrilateral $PXCY$)} \\
&=& \dang PCA \qquad \mbox{(collinearity of $A, C, Y$)}
\eeqa
and analogously $\dang PXZ = \dang PBA$. Now $X, Y, Z$ are collinear 
if and only if $\dang PXY = \dang PXZ$, which by the above equations 
occurs if and only if $\dang PCA = \dang PBA$; in other words, if and 
only if $A, B, C, P$ are concyclic.
\end{proof}
For $P$ on the circle,
the line described in the theorem is called the \emph{Simson line} 
\index{Simson line|textbf}
of $P$ with respect to the triangle $\triangle ABC$. 
We note in passing that an 
alternate proof of the collinearity in this case can be given using 
Menelaus's theorem.
\index{theorem!Menelaus's}
\index{Menelaus's theorem}

\begin{exer}
\ii
Let $A,B,C,P,Q$ be points on a circle. Show that the (directed) angle between 
the Simson lines of $P$ and $Q$ with respect to the triangle $\triangle ABC$
equals half of the (directed) arc measure $m(\arc{PQ})$.
\ii \label{ex:simcon}
Let $A,B,C,D$ be four points on a circle. Prove that the intersection 
of the Simson line 
of $A$ with respect to $\triangle BCD$ with the Simson line of $B$ with 
respect to $\triangle ACD$ is collinear with $C$ and the orthocenter of 
$\triangle ABD$.
\ii \label{ex:postsim}
If $A, B, C, P, Q$ are five points on a circle such that $\seg{PQ}$ is a 
diameter, show that the Simson lines of $P$ and $Q$ with respect to 
$\triangle ABC$ intersect at a point concyclic with the midpoints of the sides 
of $\triangle ABC$. 
\ii
Let $I$ be the incenter of triangle $\triangle ABC$, and $D, E, F$ the
projections of $I$ onto $\seg{BC}, \seg{CA}, \seg{AB}$, 
respectively. The incircle of
$ABC$ meets the segments $\seg{AI}, \seg{BI}, \seg{CI}$ 
at $M, N, P$, respectively. Show
that the Simson lines of any point on the incircle with respect to the
triangles $\triangle DEF$ and $\triangle MNP$ are perpendicular.
\end{exer}

\section{Circle of Apollonius} \label{sec:circ-app}

The ancient geometer 
Apollonius of Perga (262?--190? B.C.E.)
\index{Apollonius!of Perga}
is most famous for his early 
work on conic sections (see Section~\ref{sec:con}). 
However, his name has also come to be attached to 
another pretty geometrical construction.

\begin{theorem} \label{thm:ap}
Let $A,B$ be any two points, and let $k \neq 1$ be 
a positive real number. Then the 
locus of points $P$ such that $PA/PB = k$ is a circle whose center 
lies on $\line{AB}$.
\end{theorem}
\begin{proof}
One can show this synthetically, but the shortest proof involves
introducing Cartesian coordinates such that $A = (a, 0)$ and 
$B = (b, 0)$. The condition $PA/PB = k$ is equivalent to $PA^{2} = k^{2}
PB^{2}$, which in coordinates can be written
\[
(x-a)^{2} + y^{2} = k^{2}[(x-b)^{2} + y^{2}].
\]
Combining terms and dividing through by $1-k^{2}$, we get
\[
x^{2} + \frac{2k^{2}b-2a}{1-k^{2}}x + y^{2} = 
\frac{k^{2}b^{2}-a^{2}}{1-k^{2}},
\]
which is easily recognized as the equation of a circle whose center 
lies on the $x$-axis.
\end{proof}

This circle is called the \emph{circle of Apollonius} 
\index{circle!of Apollonius} \index{Apollonius!circle of}
corresponding 
to the points $A,B$ and the ratio $k$. (This term usually also 
includes the degenerate case $k=1$, where the ``circle'' becomes
the perpendicular bisector of $\seg{AB}$.)

\begin{exer}
\ii \label{ex:simsynth}
Use circles of Apollonius to give a synthetic proof of the 
classification of similarities (Theorem~\ref{thm:sim}).
\ii (Original) \label{ex:apcon}
Set notation as in Problem~\ref{ex:apcon1}. Prove that if some two of the
circles with diameters $\seg{A_1A_2}, \seg{B_1B_2}, \seg{C_1C_2}$ 
intersect, then the three circles are coaxial (and so Problem~\ref{ex:apcon1}
follows). Beware that the case where the circles do not meet is trickier, 
unless you work in the complex projective plane as described in 
Section~\ref{sec:alggeo}.
\end{exer}

\section{Additional problems}
\begin{exer}
\ii %% Broken chord theorem
(Archimedes' ``broken-chord'' theorem)
Point $D$ is the midpoint of arc $\arc{AC}$ of a circle; point $B$ is on minor 
arc $\arc{CD}$; and $E$ is the point on $\seg{AB}$ such that 
$\line{DE}$ is perpendicular 
to $\line{AB}$. Prove that $AE = BE + BC$.
\ii \label{ex:convex}
The convex hexagon $ABCDEF$ is such that
\[
\angle BCA = \ang DEC = \ang FAE = \ang AFB = \ang CBD = \ang EDF.
\]
Prove that $AB = CD = EF$.
\ii
(Descartes's four circles theorem)
Let $r_1,r_2,r_3,r_4$ be the radii of four mutually externally
tangent circles. Prove that
\[
\sum_{i=1}^4 \frac{2}{r_1^2} = \left( \sum_{i=1}^4 \frac{1}{r_i} \right)^2.
\]
Also verify that the 
result holds without requiring the tangencies to be external,
if one imposes the sign convention that two radii have the same sign if they
correspond to externally tangent circles, and have opposite sign
otherwise.
\ii
Deduce from the previous problem the following interesting consequence.
Define the \emph{curvature} of a circle to be the reciprocal of its radius.
Draw three mutually externally tangent circles with integer curvatures,
each internally tangent to a given unit circle.
Then repeatedly insert
the circle externally tangent to three previously drawn mutually externally
tangent circles. Show that all of the resulting circles have integer
curvature. These form an example of an \emph{Apollonian gasket}
\index{Apollonian!gasket|textbf}
(or \emph{Apollonian circle packing}) \index{Apollonian!circle packing|textbf}.
\begin{figure}[ht]
\caption{An Apollonian gasket.}
\end{figure}
\end{exer}

\chapter{Triangle trivia}
\label{chap:triangle}

In this chapter, we study but a few of the most important constructions
associated to a triangle. One could pursue this study almost indefinitely;
simply restricting to ``centers'' associated to a triangle leads to
literally hundreds\footnote{Indeed,
the companion web site
\texttt{http://faculty.evansville.edu/ck6/encyclopedia/} of the
book \cite{bib:kimberling} lists over 1000
special points associated to a triangle!}
of examples.

For convenience, we adopt the following convention throughout this
chapter: in triangle $\triangle ABC$, we write $a,b,c$ for the respectively
side lengths $BC,CA,AB$.

\section{Centroid}

For $\triangle ABC$ a triangle, the \emph{median} 
\index{median (of a triangle)|textbf} of $\triangle ABC$ from vertex $A$
is the cevian \index{cevian (of a triangle)|textbf}
from $A$ whose endpoint is the midpoint of $\seg{BC}$.
\begin{fact} \label{fact:centroid}
The medians of a triangle are concurrent. Moreover, the point of 
concurrency trisects each median.
\begin{figure}[ht]
\caption{Fact~\ref{fact:centroid}.}
\end{figure}
\end{fact}
One can easily show this using Ceva and Menelaus, or by performing an 
affine transformation making the triangle equilateral, or by using vectors. 

The concurrency point of the medians is called the \emph{centroid}\footnote{The
existence of the centroid
seems to be one of the few nontrivial facts proved in standard 
American geometry courses.}
\index{centroid (of a triangle)|textbf} of $\triangle ABC$. It is
also called the \emph{center of mass} \index{center of mass 
(of a triangle)|textbf} for the following reason:
if equal masses are placed at each of $A, B, C$,
the center of mass of the resulting system will lie at the centroid
of $\triangle ABC$.
(Compare the discussion of ``mass points''
\index{mass points} in Section \ref{sec:vec}.)

\begin{exer}
\ii (\texttt{http://www.cut-the-knot.org})
Let $G$ be the centroid of triangle $\triangle ABC$.
Draw a line through $G$ meeting $\line{AB}$ at $M$ and $\line{CA}$ at $N$.
Prove that as ratios of signed lengths, 
\[
\frac{BM}{MA} + \frac{CN}{NA} = 1.
\]
\ii (Floor van Loemen, \textit{Monthly} April 2002)
A triangle is divided into six smaller triangles by its medians. Prove
that the circumcenters of these six triangles lie on a circle.
\end{exer}

\section{Incenter and excenters}

If the point $P$ lies in the interior of triangle $\triangle ABC$,
then then the distances from $P$ 
to the lines $\line{AB}$ and $\line{AC}$ are
\[
PA \sin \angle PAB \quad \mbox{and} \quad PA \sin \angle PAC
\]
and these are equal if and only if $\angle PAB = \angle PAC$, in 
other words, if $P$ lies on the internal angle bisector of $\angle A$.

It follows that the intersection of two internal angle 
bisectors is equidistant from all three sides, and consequently lies 
on the third bisector. This intersection is the \emph{incenter} 
\index{incenter (of a triangle)|textbf} of 
$\triangle ABC$, 
and its distance to any side is the \emph{inradius}, 
\index{inradius (of a triangle|textbf} usually 
denoted $r$. The terminology comes from the fact that the circle of 
radius $r$ centered at the incenter is tangent to all three sides of 
$\triangle ABC$, and thus is called the \emph{inscribed circle}, 
\index{inscribed circle|see{incircle}}
or 
\emph{incircle}, \index{incircle (of a triangle)|textbf} of $ABC$.

Do not forget, though, that the angle $\angle A$ in triangle $\triangle ABC$ 
has \emph{two} 
angle bisectors, one internal and one external. The locus of points 
equidistant to the two lines $\line{AB}$ and $\line{AC}$ is the union of both 
lines, and so one might expect to find other circles tangent to all 
three sides. Indeed, the internal angle bisector at $A$ concurs with 
the external bisectors of the other two angles (by the same argument 
as above); the point of concurrence is the \emph{excenter} 
\index{excenter (of a triangle)|textbf} of $\triangle ABC$ opposite 
$A$, and the circle centered there tangent to all three sides is the 
\emph{escribed (exscribed) circle}, 
\index{escribed circle|see{excircle}}
\index{exscribed circle|see{excircle}}
or \emph{excircle}, \index{excircle (of a triangle)} of
$\triangle ABC$ opposite $A$.

In studying the incircle and excircles, a fundamental tool is the fact that 
the two tangents to a circle from an external point have the same 
length. This fact is equally useful is its own right, and we have 
included some problems that take advantage of equal tangents. In any 
case, the key observation we need is that if $D,E,F$ are the points 
where the incircle touches $\seg{BC}, \seg{CA}, \seg{AB}$, 
respectively, then $AE = AF$ 
and so on, so a little algebra gives
\[
AE = \frac{1}{2}(AE + EC + AF + FB - CD - DB).
\]
This establishes the first half of the following result; the second 
half is no harder.
Recall that $s = (a+b+c)/2$ is called the \emph{semiperimeter} 
\index{semiperimeter (of a polygon)} of $\triangle ABC$.
\begin{fact} \label{fact:incircle}
Let $s = (a+b+c)/2$. Then the distance from $A$ to the point where the 
incircle touches $\seg{AB}$ is $s-a$, and the distance from $A$ to the point 
where the excircle opposite $C$ touches $\seg{AB}$ is $s-b$. 
\begin{figure}[ht]
\caption{Fact~\ref{fact:incircle}.}
\end{figure}
\end{fact}

It will often be helpful to know in what ratio an angle bisector 
divides the opposite side. The answer can be used to give another proof of 
the concurrence of the angle bisectors. \index{angle bisector!theorem|textbf}
\index{theorem!angle bisector|textbf}
\begin{fact}[Angle bisector theorem]
If $D$ is the foot of either angle bisector of $A$ in triangle $\triangle
ABC$, then (as unsigned lengths)
\[
\frac{DB}{DC} = \frac{AB}{AC}.
\]
\end{fact}

Another useful construction for studying incenters is the following.
\begin{figure}
\caption{Fact~\ref{fact:midarc}.}
\end{figure}
\begin{fact} \label{fact:midarc}
Let $\triangle ABC$ be a triangle inscribed in a circle $\omega$ with center
$O$, and let $M$ be the second intersection of $\omega$ with the
internal angle bisector of $A$. 
\begin{enumerate}
\ii
The line $\line{MO}$ is perpendicular to $\line{BC}$, i.e.,
$M$ is the midpoint of arc $\arc{BC}$.
\ii
The circle centered at $M$ passing through $B$ and $C$ also passes 
through the incenter $I$ and the excenter $I_A$ opposite
$A$; that is, $MB = MI = MC = MI_A$.
\ii
We have $OI^2 = R^2 - 2Rr$, where $R$ is the circumradius and $r$ is
the inradius of $\triangle ABC$.
\end{enumerate}
\end{fact}
 
\begin{exer}
\ii
Use the angle bisector theorem to give a synthetic proof of 
Theorem~\ref{thm:ap}.

\ii (\emph{Arbelos})
The two common external tangent 
segments between two nonintersecting circles cut off a segment along
one of the common internal tangents. Show that all three segments have
the same length.

\item
The incircle of a triangle is projected onto each of the three sides.
Prove that the six endpoints of the resulting segments are concyclic.

\ii
(R\u{a}zvan Gelca) \index{Gelca, R\u{a}zvan}
Let $\triangle ABC$ be a triangle, and let
$D, E, F$ be the points where the incircle
touches the sides $\seg{BC}, \seg{CA}, \seg{AB}$, 
respectively. Let $M, N, P$ be points
on the segments $\seg{EF}, \seg{FD}, \seg{DE}$, respectively. Show that the 
lines $\line{AM},
\line{BN}, \line{CP}$ intersect if and only if the lines
 $\line{DM}, \line{EN}, \line{FP}$ intersect.

\ii \label{ex:usamo91}
(USAMO 1991/5)
Let $D$ be an arbitrary point on side $\seg{AB}$ of a given triangle 
$\triangle ABC$, 
and let $E$ be the interior point where $\seg{CD}$ intersects the external 
common tangent to the incircles of triangles $\triangle ACD$ and 
$\triangle BCD$. As $D$ 
assumes all positions between $A$ and $B$, show that $E$ traces an arc 
of a circle.

\ii (Iran, 1997) \label{ex:iran97}
Let $\triangle ABC$ be a triangle, and let
$P$ a varying point on the arc $\arc{BC}$ of 
the circumcircle of $\triangle ABC$. 
Prove that the circle through $P$ and the 
incenters of $\triangle PAB$ and $\triangle PAC$ 
passes through a fixed point independent 
of $P$.

\ii (USAMO 1999/6) \label{ex:usamo1999}
Let $ABCD$ be an isosceles trapezoid with $\line{AB} \parallel \line{CD}$. 
The inscribed
circle $\omega$ of triangle $\triangle BCD$ meets $\seg{CD}$ 
at $E$. Let $F$ be a point on
the (internal) angle bisector of $\angle DAC$ such that $\line{EF} \perp 
\line{CD}$. Let the
circumscribed circle of triangle $\triangle ACF$ meet line $\line{CD}$ 
at $C$ and $G$. Prove
that the triangle $\triangle AFG$ is isosceles.

\ii (IMO 1992/4) \label{ex:imo1992}
In the plane let $C$ be a circle, let $L$ be a line tangent
to the circle $C$, and let $M$ be a point on $L$.
Find the locus of all points $P$ with the following 
property: there exists two points $Q, R$ on $L$
such that $M$ is the midpoint of $\seg{QR}$ and $C$
is the inscribed circle of triangle $\triangle PQR$.

\ii (Bulgaria, 1996) \label{ex:bul1996}
The circles $k_1$ and $k_2$ with respective centers $O_1$ and $O_2$
are externally tangent at the point $C$, while the circle $k$ with center
$O$ is externally tangent to $k_1$ and $k_2$. Let $\ell$ be the
common tangent of $k_1$ and $k_2$ at the point $C$ and let 
$\seg{AB}$ be the diameter
of $k$ perpendicular to $\ell$. Assume that $O_2$ and $A$ lie on the same
side of $\ell$. Show that the lines $\line{AO_1}$, $\line{BO_2}$,
$\ell$ have a common
point.

\ii (MOP 1997)
Let $\triangle ABC$ be a triangle, and $D, E, F$ the points where the incircle
touches sides $\seg{BC}, \seg{CA}, \seg{AB}$, respectively. 
The parallel to $\line{AB}$ through
$E$ meets $\line{DF}$ at $Q$, and the parallel to $\line{AB}$ through $D$ 
meets $\line{EF}$
at $T$. Prove that the lines $\line{CF}, \line{DE}, \line{QT}$ are concurrent.

\ii
(Stanley Rabinowitz\footnote{Rabinowitz uses the diagram for this problem
as the logo of his company \index{Mathpro Press}
Mathpro Press.}) \index{Rabinowitz, Stanley}
The incircle of triangle $\triangle ABC$ touches sides $\seg{BC},
\seg{CA}, \seg{AB}$ at $D, E, F$, respectively. Let $P$ be any point inside
triangle $\triangle ABC$, and let $X, Y, Z$ be the points where the segments
$\seg{PA}, \seg{PB}, \seg{PC}$, 
respectively, meet the incircle. Prove that the lines
$\line{DX}, \line{EY}, \line{FZ}$ are concurrent.

\end{exer}

\section{Circumcenter and orthocenter}

It was pointed out earlier that any triangle $\triangle ABC$
has a unique circumscribing circle (circumcircle); \index{circumcircle
(of a cyclic polygon)} note that its center, the circumcenter 
\index{circumcenter (of a cyclic polygon)} of $\triangle ABC$, is
the point of concurrency of the perpendicular bisectors
of $\seg{AB}$, $\seg{BC}$, $\seg{CA}$.

A closely related point is the \emph{orthocenter}, defined as the 
intersection of the altitudes of a triangle. One can apply
Fact~\ref{thm:concperp}
to show that these actually concur, or one can modify the proof of 
the following theorem to include this concurrence as a consequence.
\begin{theorem}
Let $\triangle ABC$ be a triangle, and let $O,G,H$ be
its circumcenter, centroid and 
orthocenter, respectively. Then $O,G,H$ lie on a line in that order, and 
$2OG = GH$.
\end{theorem}
The line $OGH$ is called the \emph{Euler line} of triangle $\triangle ABC$.
\index{Euler line (of a triangle)|textbf}
\begin{proof}
The homothety with center $G$ and ratio $-1/2$ carries $\triangle ABC$ 
to the \emph{medial triangle} \index{medial triangle (of a triangle)|textbf}
$\triangle A'B'C'$, where $A', B', C'$ are the respective midpoints of
$\seg{BC}, \seg{CA}, \seg{AB}$.
Moreover, the altitude from $A'$ in the medial triangle coincides 
with the perpendicular bisector of $\seg{BC}$, since both are
perpendicular to $\line{BC}$ and pass through $A'$.
Hence $H$ maps to $O$ under the homothety, and the claim follows.
\end{proof}

Some of the problems will use the following facts about the 
orthocenter, which we leave as exercises in angle-chasing. 
\begin{figure}[ht]
\caption{The orthic triangle of a triangle (Fact~\ref{fact:orthic}).}
\end{figure}

\begin{fact} \label{fact:orthic}
In triangle $\triangle ABC$,
let $H$ be the orthocenter, and
let $H_{A}, H_{B}, H_{C}$ be the feet of the respective
altitudes from $A,B,C$.
Then the following statements hold.
\begin{enumerate}
\ii
The triangles $\triangle AH_{B}H_{C}$, $\triangle H_{A}BH_{C}$, 
$\triangle H_{A}H_{B}C$ are (oppositely) similar to $\triangle ABC$.
\ii
The altitudes bisect the angles of the triangle $\triangle 
H_{A}H_{B}H_{C}$ (so $H$ is its incenter).
\ii
The reflections of $H$ across $\line{BC}, \line{CA}, \line{AB}$ 
lie on the circumcircle of
$\triangle ABC$.
\end{enumerate}
\end{fact}
The triangle formed by the feet of the altitudes is called the 
\emph{orthic triangle}.
\index{orthic triangle (of a triangle)|textbf}

\begin{fact} \label{fact:nine-point}
Let $\triangle ABC$ be a triangle with orthocenter $H$. Define the following 
points:
\begin{itemize}
\item
let $M_A, M_B, M_C$ be the midpoints of the sides 
$\seg{BC}, \seg{CA}, \seg{AB}$;
\item
let $H_A, H_B, H_C$ be the feet of the altitudes from $A,B,C$;
\item
let $A', B', C'$ be the midpoints of the 
segments $\seg{HA}, \seg{HB}, \seg{HC}$.
\end{itemize}
Then the following statements hold.
\begin{enumerate}
\ii[(a)]
The triangle $\triangle A'B'C'$ is the half-turn of 
$\triangle M_AM_BM_C$ about its circumcenter.
\ii[(b)]
The points $M_A, M_B, M_C, H_A, H_B, H_C, A', B', C'$ lie on a single
circle.
\ii[(c)]
The center of the circle in (b) is the
midpoint of $\seg{OH}$.
\end{enumerate}
\end{fact}
The circle described in Fact~\ref{fact:nine-point} is called the
\emph{nine-point circle} 
\index{nine-point circle (of a triangle)|textbf}
of $\triangle ABC$.

\begin{exer}

\ii
Let $\triangle ABC$ be a triangle. A circle passing through $B$ and $C$
intersects $\line{AB}$ and $\line{AC}$ again at $C'$ and $B'$,
respectively. Prove that $\line{BB'}, \line{CC'}, \line{HH'}$ are concurrent, 
where $H, H'$ are the respective orthocenters of $\triangle ABC,
\triangle A'B'C'$.

\ii
(USAMO 1990/5)
An acute-angled triangle $\triangle 
ABC$ is given in the plane. The circle with 
diameter $\seg{AB}$ intersects altitude $\seg{CC'}$ and its extension at 
points $M$ and $N$, and the circle with diameter $\seg{AC}$ 
intersects altitude $\seg{BB'}$ and its extensions at $P$ and $Q$. 
Prove that the points $M, N, P, Q$ lie on a common circle.

\ii
Let $\ell$ be a line through the orthocenter %?
$H$ of a triangle $\triangle ABC$. Prove that the reflections of $\ell$ across 
$\line{AB}$, $\line{BC}$, $\line{CA}$ pass through a common point lying on
the circumcircle of $\triangle ABC$.

\ii (Bulgaria, 1997)
Let $\triangle ABC$ be a triangle with orthocenter $H$, and let $M$ and $K$
denote the midpoints of $\seg{AB}$ and $\seg{CH}$. Prove that the 
internal angle
bisectors of $\angle CAH$ and $\angle CBH$ meet at a point on
the line $\line{MK}$.

\ii \label{ex:ninept}
Prove Fact~\ref{fact:nine-point}.
\end{exer}

\section{Gergonne and Nagel points}

These points are less famous than some of the others, but they make 
for a few interesting problems, so let's get straight to work.

\begin{fact} \label{fact:gergonne}
In triangle $\triangle ABC$, the cevians joining each vertex to
the point where the incircle touches the opposite side are concurrent.
\end{fact}
The concurrency point in Fact~\ref{fact:gergonne} is called the
\emph{Gergonne point}
 \index{Gergonne point (of a triangle)|textbf}
of $\triangle ABC$. 
\begin{figure}[ht]
\caption{The Gergonne point (Fact~\ref{fact:gergonne}).}
\end{figure}

\begin{fact} \label{fact:nagel}
In triangle $\triangle ABC$, the cevians joining each vertex to
the point where the excircle opposite that vertex
touches the opposite side are concurrent.
\end{fact}
The concurrency point in Fact~\ref{fact:nagel} is called the
\emph{Nagel point}
 \index{Nagel point (of a triangle)|textbf}
of $\triangle ABC$.
\begin{figure}[ht]
\caption{The Nagel point (Fact~\ref{fact:nagel}).}
\end{figure}
 
\begin{exer}
\ii
Prove Facts~\ref{fact:gergonne} and~\ref{fact:nagel}.

\ii \label{ex:nagel}
Here is a result analogous to the existence of the Euler line.
\index{Euler line (of a triangle)}
In triangle $\triangle ABC$, let $G,I,N$ be the centroid, incenter, and Nagel 
point, respectively. Show that $I,G,N$ lie on a line in that order,
and that $NG = 
2\cdot IG$.

\ii
Continuing the analogy from the previous problem, prove that
in triangle $\triangle ABC$,
if $P,Q,R$ are the midpoints of sides $\seg{BC}, \seg{CA}, \seg{AB}$, 
respectively, then
the incenter of $\triangle PQR$ is the midpoint of $\seg{IN}$. 
\end{exer}

\section{Isogonal conjugates}

Two points $P$ and $Q$ inside triangle $ABC$ are said to be 
\emph{isogonal conjugates} \index{isogonal conjugates|textbf}
if 
\[
\angle PAB = \angle QAC, \qquad
\angle PBC = \angle QCB, \qquad
\angle PCA = \angle QAC.
\]
In other words, $Q$ is the reflection of $P$ across each of the internal 
angle bisectors of $\triangle ABC$.

\begin{fact} \label{fact:isogonal}
Every point in the interior of $\triangle ABC$ has an isogonal
conjugate.
\end{fact}
This instantly gives rise to some new special points of a triangle.
For example, the isogonal conjugate of the centroid of $\triangle ABC$
is called the \emph{Lemoine point} 
\index{Lemoine point (of a triangle)|textbf}; 
the cevians through the Lemoine point are called \emph{symmedians}.
\index{symmedian (of a triangle)|textbf}

\begin{exer}
\ii \label{ex:isogonal}
Prove Fact~\ref{fact:isogonal}, then formulate and prove a correct
version for points not in the interior of the triangle.

\ii
Prove that in an acute triangle,
the orthocenter and the circumcenter are isogonal conjugates. If you
completed the previous problem, you should also be able to prove this
for a general triangle.

\ii
Given a triangle, draw through its Lemoine point 
a line parallel to each side of the 
triangle, and consider the intersections of that line with the other two 
sides. Show that the resulting six points are concyclic.

\ii \label{ex:tangents symm}
Show that the tangents to the circumcircle of a triangle at two vertices 
intersect on the symmedian of the third vertex.

\ii (Dan Moraseski) \index{Moraseski, Dan}
Let $D, E, F$ be the feet of the symmedians of triangle
$\triangle ABC$. Prove that the Lemoine point of $\triangle ABC$ is the 
centroid of $\triangle DEF$.

\end{exer}
\section{Brocard points}

The problems in this section establish the existence and several 
properties of the Brocard points. Unlike the other special points
we have thus far associated to a triangle, the Brocard points are only 
defined in a cyclically symmetric fashion. Consequently, there are two of them
which are interchanged by reversal of the order of the vertices.

\begin{fact} \label{fact:brocard}
For any triangle $\triangle ABC$, there exists a unique point $P$ in
the interior of $\triangle ABC$ such that
\[
\angle PAB = \angle PBC = \angle PCA.
\]
\end{fact}
The point $P$ in Fact~\ref{fact:brocard}
is called a \emph{Brocard point} \index{Brocard!points (of a triangle)|textbf}
of $\triangle ABC$; there is a second
Brocard point obtained by reversing the order of the vertices.

\begin{fact} \label{fact:brocard2}
The two Brocard points of a triangle are isogonal conjugates.
\end{fact}
This is equivalent to the fact that the common angle in
Fact~\ref{fact:brocard} is the same for the two Brocard points. It
is called the \emph{Brocard angle} \index{Brocard!angle (of a triangle)|textbf}
of $\triangle ABC$; see Problem~\ref{ex:brocard angle} for an explicit
formula for the Brocard angle.

\begin{exer}
\ii \label{ex:broc1}
Prove Fact~\ref{fact:brocard}.

\ii \label{ex:brocard angle}
Let $\omega$ be the angle such that
\[
\cot \omega = \cot A + \cot B + \cot C.
\]
Prove that the common angle in Fact~\ref{fact:brocard} is equal to $\omega$;
deduce Fact~\ref{fact:brocard2}.

\ii (IMO \fixme{get reference})
In triangle $\triangle ABC$, put $K = [ABC]$. Prove that
\[
a^{2} + b^{2} + c^{2} \geq 4\sqrt{3}K
\]
by expressing the Brocard angle in terms of $a,b,c,K$.

\ii \label{ex:imo91}
(IMO 1991/5)
Prove that inside any triangle $\triangle 
ABC$, there exists a point $P$ such that 
one of the angles $\ang PAB, \ang PBC, \ang PCA$ has measure at most 
$30^{\circ}$.
\end{exer}

\section{Frame shift}

Once one has gathered up a lot of triangle trivia, it becomes necessary
to use it effectively. Often this is accomplished by what I call a
``frame shift'': you are originally given some points in reference to
a given triangle, but you then view them in reference to a different triangle.
For instance, given triangle $\triangle ABC$:
\begin{enumerate}
\item The orthocenter is the incenter of the orthic triangle.
\item The circumcenter is the orthocenter of the medial triangle.
\end{enumerate}

\begin{exer}
\ii \label{ex:rus2003frame}
(Russia, 2003)
Let $O$ and $I$ be the circumcenter and incenter of triangle $\triangle ABC$,
respectively. Let $\omega_A$ be the excircle of triangle $\triangle ABC$ 
opposite
to $A$; let it be tangent to $\line{AB}$, $\line{AC}$, $\line{BC}$ 
at $K,M,N$, respectively.
Assume that the midpoint of segment $\seg{KM}$ lies on the circumcircle
of triangle $\triangle ABC$. Prove that $O,N,I$ are collinear.
\end{exer}

\section{Vectors for special points}

The vector equations for some of the special points of a triangle 
$\triangle ABC$ are 
summarized in the following table. The asterisked expressions assume the 
circumcenter of the triangle has been chosen as the origin; the 
origin-independent expressions are not nearly so pleasant to work with!

\begin{center}
\begin{tabular}{cc}
Circumcenter* & $0$ \\
Centroid & $\frac{1}{3} (\vA + \vB + \vC)$ \\
Orthocenter* & $\vA + \vB + \vC$ \\
Incenter & $\frac{1}{a+b+c} (a\vA + b\vB + c\vC)$
\end{tabular}
\end{center}

\begin{exer}
\ii
Let $P, Q, R$ be the feet of concurrent cevians in triangle $\triangle ABC$.
Determine the vector expression for the point of concurrence in terms
of the ratios $BP/PC$, $CQ/QA$, $AR/RB$. Use this formula to extend 
the above table to other special points. In particular, do so for the 
Nagel point and obtain an alternate solution to 
Problem~\ref{ex:nagel}.
\ii
Let $A,B,C,D$ be four points on a circle. Use the result of 
Problem~\ref{ex:simcon} to 
show that the Simson line of each point with respect 
to the triangle formed by the other three passes through the midpoint 
of the segment joining the center of the circle to the centroid of 
$ABCD$. In particular, the four Simson lines are concurrent.
\ii (MOP 1995) %% MOP 1995
Five points are given on a circle. A perpendicular is drawn through 
the centroid of the triangle formed by three of them, to the chord 
connecting the remaining two. Similar perpendiculars are drawn for 
each of the remaining nine triplets of points. Prove that the ten 
lines obtained in this way have a common point.
\ii \label{ex:cirort} 
Compute the distance between the circumcenter and orthocenter of a 
triangle in terms of the side lengths $a,b,c$.
\ii
Show that the distance between the incenter and the nine-point 
center (see Problem~\ref{ex:ninept}) of a triangle
is equal to $R/2 - r$, where $r$ and $R$ are 
inradius and circumradius, respectively. Deduce \emph{Feuerbach's 
theorem}, that the incircle and nine-point circle are tangent. 
(Similarly, one can show the nine-point circle is also tangent to each of 
the excircles.)
\end{exer}

\section{Additional problems}

Here are a few additional problems concerning triangle trivia. Before 
proceeding to the problems, we state as facts a few standard formulae 
for the area of a triangle.

\begin{fact} \label{fact:area formulas}
Let $\triangle ABC$ be a 
triangle with side lengths $a = BC, b = CA, c = AB$, 
inradius $r$, circumradius $R$, exradius opposite $A$ $r_{A}$,
semiperimeter $s$, and area $K$. Then
\begin{eqnarray*}
K  &=& \frac{1}{2} ab \sin C \quad \mbox{(Law of Sines)}\\
&=& \frac{abc}{4R} \quad \mbox{(by Extended Law of Sines)} \\
&=& rs = r_{A} (s-a) \\
&=& \sqrt{s(s-a)(s-b)(s-c)}. \quad \mbox{(Heron's formula)}
\end{eqnarray*}
\end{fact}
\index{Heron's formula|textbf}

\begin{exer}
\ii \label{ex:stewart}
Let $D$ be a point on side $BC$, and let $m = BD, n = CD$ and $d = 
AD$. Prove \emph{Stewart's formula}:\footnote{If written 
``man + dad = bmb + cnc'', this admits the mnemonic
``A man and his dad build a bomb in the sink,'' in case you can
recall what the letters stand for before the U.S. Department
of Homeland Security pays a visit.} \index{Stewart's formula|textbf}
\index{formula!Stewart's|textbf}
\[
mb^2 + nc^2 = a (d^{2} + 
mn).
\]
\begin{figure}[ht]
\caption{Stewart's formula (Problem~\ref{ex:stewart}).}
\end{figure}

\ii Use Stewart's formula to prove the \emph{Steiner-Lehmus
theorem}: \index{Steiner-Lehmus theorem|textbf}
a triangle with two equal angle bisectors must be
isosceles. (A synthetic proof is possible but not easy to find.)

\ii
(United Kingdom, 1997) 
In acute triangle $\triangle ABC$, $\seg{CF}$ is an altitude,
with $F$ on $\seg{AB}$, and $\seg{BM}$ is a median, with $M$ on $\seg{CA}$. 
Given that
$BM=CF$ and $\angle MBC = \angle FCA$, prove that the triangle $\triangle ABC$
is equilateral. Also, what happens if $\triangle ABC$ is not acute?

\ii 
The
point $D$ lies inside the acute triangle $\triangle ABC$.  Three of the
circumscribed circles of the triangles $\triangle ABC, \triangle BCD, 
\triangle CDA, \triangle DAB$ have equal
radii. Prove that the fourth circle has the same radius, and
characterize all such sets of four points. Also, 
what happens if $\triangle ABC$ need
not be acute, or $D$ need not lie in its interior?

\ii (Bulgaria, 1997)
Let $\triangle ABC$ be a triangle and let $M,N$ be the feet of the angle
bisectors of $B,C$, respectively. Let $D$ be the intersection of the
ray $\ray{MN}$ with the circumcircle of $\triangle ABC$. Prove that
\[
\frac{1}{BD} = \frac{1}{AD} + \frac{1}{CD}.
\]

\ii %% Don't have solution
Let $ABCDE$ be a cyclic pentagon such that $r_{ABC} = r_{AED}$ and 
$r_{ABD} = r_{ACE}$, where $r_{XYZ}$ denotes the inradius of triangle 
$\triangle XYZ$. Prove that $AB = AE$ and $BC = DE$.

\ii %% MOP 1990--Russia IMO proposal?
(MOP 1990)
Let $\seg{AA_{1}}, \seg{BB_{1}}, \seg{CC_{1}}$ 
be the altitudes in an acute triangle 
$\triangle ABC$, and 
let $K$ and $M$ be points on the line segments $\seg{A_{1}C_{1}}$ and 
$\seg{B_{1}C_{1}}$, respectively. Prove that if the angles 
$\angle MAK$ and 
$\angle CAA_{1}$ are equal, then the angle $\angle C_{1}KM$ is bisected 
by $\ray{KA}$.
\end{exer}

%% Quadrilaterals

\chapter{Quadrilaterals}

\section{General quadrilaterals}
There's not a great deal that can be said about an arbitrary 
quadrilateral: the extra freedom in placing an additional vertex 
disrupts much of the structure we found in triangles. What little 
there is to say we offer in the form of a few problems.

\begin{exer}
\ii
Prove that the midpoints of the sides of any quadrilateral form a 
parallelogram (known as the \emph{Varignon parallelogram}).
\index{Varignon parallelogram|textbf}
\ii
Let $ABCD$ be a convex quadrilateral, and let $\theta$ be the angle 
between the diagonals $\seg{AC}$ and $\seg{BD}$. Prove that
\[
[ABCD] = \half AC \cdot BD \sin \theta.
\]
\ii
Derive a formula for the area of a convex quadrilateral in terms of 
its four sides and a pair of opposite angles.
\end{exer}

\section{Cyclic quadrilaterals}

The condition that the four vertices of a quadrilateral lie on a 
circle gives rise to a wealth of interesting structures, which we 
investigate in this section. We start with a classical result of 
Claudius Ptolemy (85?-165?), \index{Ptolemy, Claudius}
who is more famous for his geocentric model of planetary motion.

\begin{theorem}[Ptolemy] \label{thm:pt}
Let $ABCD$ be a convex cyclic quadrilateral. Then
\[
AB \cdot CD + BC \cdot DA = AC \cdot BD.
\]
\end{theorem}
\begin{proof}
Mark the point $P$ on $\seg{BD}$ such that $BP = (AB \cdot CD)/AC$, or 
equivalently $BP/AB = CD/AC$. Since $\ang ABP = \ang ACD$, the 
triangles $ABP$ and $ACD$ are similar.
\begin{figure}[ht]
\caption{Proof of Ptolemy's theorem (Theorem~\ref{thm:pt}).}
\label{fig:ptolemy}
\end{figure}

On the other hand, we now have
\[
\ang DPA = \pi - \ang APB = \pi - \ang ADC = \ang CBA.
\]
Thus the triangles $\triangle APD$ and $\triangle ABC$ 
are also similar, yielding $DP/BC = 
AD/AC$. Consequently
\[
BD = BP + PD = \frac{AB \cdot CD}{AC} + \frac{AD \cdot BC}{AC}
\]
and the theorem follows.
\end{proof}
This proof is elegant, but one cannot help wondering, ``How could 
anyone think of that?'' (I wonder that myself; the proof appears in an 
issue of Samuel Greitzer's \emph{Arbelos}, but he gives no 
attribution.) The reader might enjoy attempting a proof using 
trigonometry or complex numbers.

Another important result about cyclic quadrilaterals is an area formula 
attributed to the ancient Indian mathematician 
\index{formula!Brahmagupta's|textbf}
\index{Brahmagupta's formula|textbf} \index{Brahmagupta}
Brahmagupta (598-670).\footnote{This is a rare 
case where an Eastern mathematical
discovery is reflected by Western naming customs. Compare the situation
for Pascal's triangle; see Section~\ref{sec:pb}.}
\begin{fact}[Brahmagupta] \label{fact:brahmagupta}
If a cyclic quadrilateral has sides $a,b,c,d$ and area $K$, then
\[
K = \sqrt{(s-a)(s-b)(s-c)(s-d)},
\]
where $s = (a+b+c+d)/2$ is the semiperimeter.
\end{fact}
Heron's formula \index{Heron's formula}
for the area of a triangle follows from Brahmagupta's 
formula by regarding a triangle as a cyclic quadrilateral with one 
side of length 0. 

\begin{exer}
\ii
Use trigonometry to give another proof of Ptolemy's theorem
(Theorem~\ref{thm:pt}).
\ii (Brahmagupta) \index{Brahmagupta}
Let $ABCD$ be a cyclic quadrilateral with perpendicular diagonals.  
Then the line through the intersection of the diagonals and the 
midpoint of any side is perpendicular to the opposite side.
\ii
Use Ptolemy's theorem and the previous problem to give
a formula for the lengths of the diagonals of a cyclic quadrilateral
in terms of the lengths of the sides.
\ii \label{ex:inrect}
Let $ABCD$ be a cyclic quadrilateral. Prove that the incenters of 
triangles $\triangle ABC, \triangle BCD, \triangle CDA, 
\triangle DAB$ form a rectangle.
\ii
Let $ABCD$ be a cyclic quadrilateral.
Prove that the sum of the inradii of $\triangle ABC$ and 
$\triangle CDA$ equals the sum of the inradii of $\triangle BCD$ 
and $\triangle DAB$.
\end{exer}

\section{Circumscribed quadrilaterals}

The following theorem characterizes circumscribed quadrilaterals; 
while it can be proved directly using the equal tangents rule, it 
proves easier to exploit what we already know about incircles and 
excircles of triangles. 
\begin{figure}[ht]
\caption{A circumscribed quadrilateral.}
\end{figure}
\begin{theorem}
A convex quadrilateral $ABCD$ admits an inscribed circle if and only 
if $AB + CD = BC + DA$.
\end{theorem}
\begin{proof}
Let $\line{AB}$ and $\line{CD}$ 
meet at $P$; without loss of generality, assume $A$ lies between $P$ 
and $B$. (We skip the limiting case $AB \parallel CD$.)
The quadrilateral $ABCD$ has an inscribed circle if and only if the 
incircle of $\triangle PBC$ coincides with the excircle of 
$\triangle PDA$ opposite $P$. 
Let $Q$ and $R$ be the points of tangency of  $\line{PB}$ with 
the incircle of $\triangle PBC$ and the excircle of 
$\triangle PDA$, respectively;
since both circles are tangent to the sides of the 
angle $\ang CPB$, they coincide if and only if $Q = R$, or 
equivalently $PQ = PR$. However, by the usual formulae
\beqa
PQ &=& \half (PB + PC - BC) = \half (PD + DC + PA + AB - BC) \\
PR &=& \half (PA + PD + DA)
\eeqa
and these are equal if and only if $AB + CD = BC + DA$.
\end{proof}

Just as with triangles, a convex quadrilateral can have an escribed 
circle, a circle not inside the quadrilateral but tangent to all 
four sides (or rather their extensions). 
\begin{figure}[ht]
\caption{A quadrilateral having an escribed circle.}
\end{figure}
We trust the reader 
can now supply the proof of the analogous characterization of 
quadraterals admitting an escribed circle.
\begin{fact}
A convex quadrilateral $ABCD$ admits an exscribed circle opposite $A$ 
or $C$ if and only if $AB + BC = CD + DA$.
\end{fact}

For more problems about circumscribed quadrilaterals, flip back to 
Section~\ref{sec:pb}, where we study them using Brianchon's theorem.

\begin{exer}
\ii (IMO 1962/5)
On the circle $K$ there are given three distinct points $A,B,C$. 
Construct (using only straightedge and compass) a fourth point $D$ 
on $K$ such that a circle can be inscribed in the quadrilateral thus 
obtained.
\ii (Dick Gibbs)
Let $ABCD$ be a quadrilateral inscribed in an ellipse, and let $E = AB 
\cap CD$ and $F = AD \cap BC$. Show that $ACEF$ can be inscribed in a 
hyperbola with the same foci as the ellipse. (If you're not familiar 
with ellipses and hyperbolae, peek ahead to Section~\ref{sec:con}.)
\ii (USAMO 1998/6) \label{ex:circquads}
Let $n \geq 5$ be an integer. Find the largest integer $k$ (as a 
function of $n$) such that there exists a convex $n$-gon $A_1A_2\cdots 
A_n$ for which exactly $k$ of the quadrilaterals 
$A_iA_{i+1}A_{i+2}A_{i+3}$ have an inscribed circle, where 
$A_{n+j} = A_j$.
\end{exer}

\section{Complete quadrilaterals}

A \index{complete quadrilateral|textbf} \index{quadrilateral!complete|textbf}
\emph{complete quadrilateral} is the figure formed by four lines, no
two parallel and no three concurrent; the \emph{vertices} of a
complete quadrilateral are the six pairwise intersections of the
lines. This configuration has been widely studied; we present here as
problems a number of intriguing properties of the diagram.

In the following problems, let $ABCDEF$ be the complete quadrilateral 
formed by the lines $ABC, AEF, BDF, CDE$. 
\begin{figure}[ht]
\caption{A complete quadrilateral.}
\end{figure}

\begin{exer}
\ii
Show that the orthocenters of the triangles $\triangle ABF$,
$\triangle ACE$, $\triangle BCD$, $\triangle DEF$ are collinear.
The common line is called the \emph{ortholine} \index{ortholine (of a
complete quadrilateral)|textbf}
of the complete quadrilateral.
\ii \label{ex:comp quad coaxial}
Show that the circles with diameters $\seg{AD}, \seg{BE}, \seg{CF}$ 
are coaxial. Deduce 
that the midpoints of the segments $\seg{AD}, \seg{BE}, \seg{CF}$ are 
collinear. (Can you show the latter directly?)
\ii %% Simson lines?
Show that the circumcircles of the triangles 
$\triangle ABF$,
$\triangle ACE$, $\triangle BCD$, $\triangle DEF$ 
pass through a 
common point, called the \emph{Miquel point} 
\index{Miquel!point (of a complete quadrilateral)|textbf} of the complete
quadrilateral. (Many solutions are possible.)
\ii
We are given five lines in the plane, no two parallel and no three 
concurrent. To every four of the lines, associate the point whose 
existence was shown in the previous problem. Prove these five points 
lie on a circle. (This assertion and the previous one belong to an 
infinite chain of such statements: see W.K. Clifford, 
\textit{Collected Papers} (1877), 38--54.)
\end{exer}

\chapter{Geometric inequalities}

The subject of geometric inequalities is so vast that it suffices to 
fill entire books, two notable examples being the volume by Bottema et
al.\ \cite{bib:bott} and its sequel \cite{bib:rcige}.
This chapter should thus be regarded more as 
a sampler of techniques
than a comprehensive treatise.

\section{Distance inequalities}

A number of inequalities involve comparing lengths. Useful tools 
against such problems include:
\begin{itemize}
\ii \index{triangle inequality|textbf}
Triangle inequality: in triangle $\triangle ABC$, $AB + BC > BC$.
\ii
Hypotenuse inequality: if $\ang ABC$ is a right angle, then $AC > BC$.
\ii \index{Ptolemy's inequality|textbf} \index{inequality!Ptolemy's|textbf}
Ptolemy's inequality (Problem~\ref{ex:ptineq}):
if $ABCD$ is a convex quadrilateral, 
then
$AB \cdot CD + BC \cdot DA \geq AC \cdot BD$, with equality if and 
only if $ABCD$ is cyclic.
\ii \index{Erd\H{o}s-Mordell inequality}
\index{inequality!Erd\H{o}s-Mordell}
Erd\H{o}s-Mordell inequality: see Section~\ref{sec:em}.
\end{itemize}

Transformations can also be useful, particularly reflection. \index{reflection}
For example, to find the point $P$ on a fixed line 
that minimizes the sum of the distances from $P$ to two fixed points 
$A$ and $B$, reflect the segment $\seg{PB}$ across the line and observe that 
the optimal position of $P$ is on the line joining $A$ to the 
reflection of $B$. 
\begin{figure}[ht]
\caption{Minimizing the distance from a point on a line to two fixed
points.}
\end{figure}

A more dramatic example along the same lines is the following solution 
(by H.A. Schwarz)
to \emph{Fagnano's problem}: \index{Fagnano's problem}
\index{problem!Fagnano's}
of the triangles inscribed in a given 
acute triangle, which one has the least perimeter? Reflecting the 
triangle as shown implies that the perimeter of an inscribed triangle 
is at least the distance from $A$ to its eventual image, with 
equality when the inscribed triangle makes equal angles with each 
side. As noted earlier, this occurs for the orthic triangle, which is 
then the desired minimum. 
\begin{figure}[ht]
\caption{Solution of Fagnano's problem.}
\end{figure}

\begin{exer}
\ii
For what point $P$ inside a convex quadrilateral $ABCD$ is 
$PA+PB+PC+PD$ minimized?
\ii
(Euclid) Prove that the longest chord whose vertices lie on or 
inside a given triangle is the longest side. (This is intuitively
obvious, but make sure your proof is complete.)
\ii
(K\"ursch\'ak, 1954)
Suppose a convex quadrilateral $ABCD$ satisfies $AB+BD \leq AC+CD$.
Prove that $AB < AC$.
\ii (USAMO 1999/2) \label{ex:usamo992}
Let $ABCD$ be a cyclic quadrilateral. Prove that
\[
|AB - CD| + |AD - BC| \geq 2|AC - BD|.
\]
\ii (Titu Andreescu and R\u{a}zvan Gelca) \index{Andreescu, Titu}
\index{Gelca, R\u{a}zvan}
Points $A$ and $B$ are separated by two rivers. One bridge is to be 
built across each river so as to minimize the length of the shortest 
path from $A$ to $B$. Where should they be placed? (Each river is an 
infinite rectangular strip, and each bridge must be a straight segment 
perpendicular to the sides of the river. You may assume that $A$ and 
$B$ are separated from the intersection of the rivers by a strip 
wider than the two rivers combined.)
\ii
Prove that a quadrilateral inscribed in a parallelogram
has perimeter no less than twice the length 
of the shorter diagonal of the parallelogram. (You may want to first 
consider the case where the parallelogram is a rectangle.)
\ii
(IMO 1993/4)
For three points $P,Q,R$ in the plane, we define $m(PQR)$ as the minimum 
length of the three altitudes of $\triangle PQR$. (If the points are 
collinear, we set $m(PQR) = 0$.)
Prove that for points $A,B,C,X$ in the plane,
\[
m(ABC) \leq m(ABX) + m(AXC) + m(XBC).
\]  
\ii (Sylvester's theorem) \label{ex:syl}
A finite set of points in the plane has the property that the line 
through any two of the points passes through a third. Prove that all 
of the points are collinear. (As noted in Problem~\ref{ex:flex}, 
this result is false in the complex projective plane.)
\ii (IMO 1973/4)
A soldier needs to check on the presence of mines in a region having 
the shape of an equilateral triangle. The radius of action of his 
detector is equal to half the altitude of the triangle. The soldier 
leaves from one vertex of the triangle. What path should he follow in 
order to travel the least possible distance and still accomplish his 
mission?
\ii
Suppose the largest angle of triangle $\triangle ABC$ is not greater than 
$120^{\circ}$. 
Let $D$ be the third vertex of an equilateral 
triangle constructed externally on side $\seg{BC}$.
For $P$ inside the triangle, 
show that $PA + PB + PC \geq AD$, and determine when equality holds.
\ii
Suppose the largest angle of triangle $\triangle ABC$ is not greater than 
$120^{\circ}$. 
Deduce from the previous problem that
for $P$ inside the triangle, $PA + PB + PC$ is minimized 
when $\ang APB = \ang BPC = \ang CPA = 120^{\circ}$. The point 
satisfying this condition is known variously as the \emph{Fermat 
point} or the \emph{Torricelli point}.
\ii
(IMO 1995/5)
Let $ABCDEF$ be a convex hexagon with $AB=BC=CD$ and $DE=EF=FA$, such 
that $\angle BCD = \angle EFA = \pi/3$. Suppose $G$ and $H$ are points in 
the interior of the hexagon such that $\angle AGB = \angle DHE = 2\pi/3$. 
Prove that $AG + GB + GH + DH + HE \geq CF$.
\end{exer}

\section{Algebraic techniques}
Another class of methods of attack for geometric inequalities involve 
invoking algebraic inequalities. The most commonly used is the
AM-GM inequality: for $x_{1}, \dots, x_{n} > 0$,
\[
\frac{x_{1} + \cdots + x_{n}}{n} \geq (x_{1}\dots x_{n})^{1/n}.
\]
Often all one needs is the case $n=2$, which follows from the fact 
that
\[
(\sqrt{x_{1}} - \sqrt{x_{2}})^{2} \geq 0.
\]

A more sophisticated result is the Cauchy-Schwarz inequality:
\[
(x_{1}^{2} + \cdots + x_{n}^{2})(y_{1}^{2} + \cdots + y_{n}^{2})
\geq (x_{1}y_{1} + \cdots + x_{n}y_{n})^{2},
\]
which one proves by noting that the difference between the left side 
and the right is
\[
\sum_{i<j} (x_{i}y_{j} - x_{j}y_{i})^{2}.
\]

A trick that often makes an algebraic approach more feasible, when a 
problem concerns the side lengths $a,b,c$ of a triangle, is to
make the substitution
\[
x = s-a, \quad y = s-b, \quad z = s - c,
\]
where $s = (a+b+c)/2$. A little algebra gives
\[
a = y+z, \quad b = z+x, \quad c = x+y.
\]
The point is that the necessary and sufficient
conditions $a+b > c, b+c > a, c+a > b$ for $a,b,c$ to constitute the 
side lengths of a triangle translate into the more convenient 
conditions $x>0, y>0, z>0$.

Don't forget about the possibility of ``algebraizing'' an inequality 
using complex numbers; see Section~\ref{sec:cplx}.

\begin{exer}
\ii
(IMO 1988/5)
The triangle $\triangle
ABC$ has a right angle at $A$, and $D$ is the foot of
the altitude from $A$. The straight line joining the incenters of the 
triangles $\triangle ABD, \triangle ACD$ 
intersects the sides $\seg{AB}, \seg{AC}$ at the points
$K,L$, respectively. $S$ and $T$ denote the areas of the triangles 
$\triangle ABC$
and $\triangle AKL$, respectively. Show that $S \geq 2T$.
\ii
Given a point $P$ inside a triangle $ABC$, let $x,y,z$ be the 
distances from $P$ to the sides $\seg{BC}, \seg{CA}, \seg{AB}$. 
Find the point $P$ which 
minimizes
\[
\frac{a}{x} + \frac{b}{y} + \frac{c}{z}.
\] 
\ii
If $K$ is the area of a triangle with sides $a,b,c$, show that
\[
ab + bc + ca \geq 4 \sqrt{3} K.
\]
\ii
(IMO 1964/2) Suppose $a,b,c$ are the sides of a triangle. Prove that
\[
a^{2}(b+c-a) + b^{2}(c+a-b) + c^{2}(a+b-c) \leq 3abc.
\]
\ii
(IMO 1983/6)
Let $a,b,c$ be the lengths of the sides of a triangle. Prove that
\[
b^2c(b-c) + c^2a(c-a) + a^2b(a-b) \geq 0.
\]
(Beware: you may not assume that $a\geq b \geq c$ without loss of 
generality!)
\ii
(Balkan, 1996)
Let $O$ and $G$ be the circumcenter and centroid of a triangle of
circumradius $R$ and inradius $r$. Show that $OG^2 \leq R^2 -
2Rr$. (This proves Euler's inequality $R \geq 2r$. 
\index{Euler's inequality} If you don't know
how to compute $OG^2$, see Problem~\ref{ex:cirort}.)
\ii
(Murray Klamkin) \index{Klamkin, Murray}
Let $n > 2$ be a positive integers, and suppose that $a_1, \dots, a_n$
are positive real numbers satisfying the inequality
\[
(a_1^2 + \cdots + a_n^2)^2 > (n-1) (a_1^4 + \cdots + a_n^4).
\]
Show that for $1 \leq i < j < k \leq n$, the numbers $a_i, a_j, a_k$
are the lengths of the sides of a triangle.
\ii \label{ex:cotident}
Let $\triangle 
ABC$ be a triangle with inradius $r$ and circumradius $R$. Prove 
that
\[
\frac{2r}{R} \leq \sqrt{\cos \frac{A-B}{2} \cos \frac{B-C}{2} \cos 
\frac{C-A}{2}}.
\]
\ii \label{ex:quadineq}
(IMO 1995 proposal)
Let $P$ be a point inside the convex quadrilateral $ABCD$. Let 
$E,F,G,H$ be points on sides $\seg{AB},\seg{BC},\seg{CD},\seg{DA}$, 
respectively, such that 
$\line{PE}$ is parallel to $\line{BC}$, 
$\line{PF}$ is parallel to $\line{AB}$, 
$\line{PG}$ is parallel to $\line{DA}$, 
and $\line{PH}$ is parallel to $\line{CD}$.
Let $K, K_{1}, K_{2}$ be the 
areas of $ABCD, AEPH, PFCG$, respectively. Prove that
\[
\sqrt{K} \geq \sqrt{K_{1}} + \sqrt{K_{2}}.
\]
\end{exer}

\section{Trigonometric inequalities and convexity}
A third standard avenue of attack involves reducing a geometric 
inequality to an inequality involving 
trigonometric functions. Such inequalities can often be treated using 
Jensen's inequality for convex functions.

A \emph{convex} function is a function $f(x)$ satisfying the rule
\[
f(tx + (1-t)y) \leq tf(x) + (1-t)f(y)
\]
for all $x,y$ and all $t \in [0,1]$. Geometrically, this says that the 
area above the graph of $f$ is a convex set, i.e.\ that chords of the 
graph always lie above the graph. Equivalently, tangents to the graph 
lie below.

Those of you who know calculus can check whether $f$ is convex by 
checking whether the second derivative of $f$ (if it exists) is always positive. 
(In some calculus texts, a convex function is called ``concave 
upward'', or occasionally is said to ``hold water''.) Also, if $f$ is 
continuous, it suffices to check the definition of convexity for 
$t=1/2$.

The key fact about convex functions is \emph{Jensen's inequality}, 
\index{Jensen's inequality|textbf} \index{inequality!Jensen's|textbf}
whose proof (by induction on $n$) is not difficult.
\begin{fact}
Let $f(x)$ be a convex function, and let $t_{1}, \dots, t_{n}$ be 
nonnegative real numbers adding up to $1$. Then for all $x_{1}, \dots, 
x_{n}$,
\[
f(t_{1}x_{1} + \cdots + t_{n}x_{n}) \leq t_{1}f(x_{1}) + \cdots + 
t_{n}f(x_{n}).
\]
\end{fact}
For example, the convexity of the function $(-\log x)$ implies the 
AM-GM inequality.

As a simple example, note that in triangle 
$\triangle ABC$, we have $\angle A + \angle B
+ \angle C = \pi$, and the function $f(x) = \sin x$ is concave, so
\[
\sin A + \sin B + \sin C \geq 3 \sin \pi/3 = 3\sqrt{3}/2.
\]
In other words, the minimum perimeter of a triangle inscribed in a fixed
circle is achieved by the equilateral triangle.

Also note that convexity can be used in apparently purely geometric 
circumstances, thanks to the following fact. (Remember, it suffices to 
verify this for $t=1/2$, which is easy.)
\begin{fact}
The distance from a fixed point $P$ is a convex function on the plane. 
That is, for any points $P,Q,R$, the distance from $P$ to the point 
(in vector notation) $tQ + (1-t)R$ is a convex function of $t$.
\end{fact}

\begin{exer}
\ii
(\cite{bib:bott}, 2.7)
Show that in triangle $\triangle ABC$, $\sin A \sin B \sin C \leq \frac{3}{8}
\sqrt{3}$.
\ii
Prove that the Brocard angle \index{Brocard!angle (of a triangle)}
of a triangle cannot exceed $\pi/6$. (Hint: use 
Problem~\ref{ex:broc1}, but beware that 
$\cot$ is only convex in the range $(0, \pi/2]$.)
\ii (\cite{bib:bott}, 2.15)
Let $\alpha, \beta, \gamma$ be the angles of a triangle. Prove that
\[
\sin \frac{\beta}{2} \sin \frac{\gamma}{2} +
\sin \frac{\gamma}{2} \sin \frac{\alpha}{2} +
\sin \frac{\alpha}{2} \sin \frac{\beta}{2}  \leq \frac 34.
\]
\ii
Prove that of the $n$-gons inscribed in a circle, the regular $n$-gon 
has maximum area.
\ii (\cite{bib:bott}, 2.59)
Prove that in triangle $\triangle ABC$,
\[
1 + \cos A \cos B \cos C \geq \sqrt{3} (\sin A \sin B \sin C ).
\]
\ii
Show that for any convex polygon $S$, the distance from $S$ to a 
point $P$ (the length of the shortest segment joining $P$ to a point 
on $S$) is a convex function of $P$.
\ii (Junior Balkaniad, 1997)
In triangle $\triangle ABC$, let $D,E,F$ be the points where the incircle 
touches the sides. Let $r, R, s$ be the inradius, circumradius, and 
semiperimeter, respectively, of the triangle. Prove that
\[
\frac{2rs}{R} \leq DE + EF + FD \leq s
\]
and determine when equality occurs.
\ii (MOP 1998)
Let $\triangle ABC$ be a acute triangle with circumcenter $O$, 
orthocenter $H$ and circumradius $R$.
Show that for any point $P$ on the segment $\seg{OH}$,
\[
PA + PB + PC \leq 3R.
\]
\end{exer}

\section{The Erd\H{o}s-Mordell inequality} 
\label{sec:em}

The following inequality is somewhat more sophisticated than the ones 
we have seen so far, but is nonetheless useful. It was conjectured by 
the Hungarian mathematician  P\'al (Paul) 
Erd\H{o}s (1913-1996) \index{Erd\H{o}s, P\'al (Paul)}
in 1935 and first proved by
Louis Mordell \index{Mordell, Louis} in the same year.
\begin{theorem}
For any point $P$ inside the triangle $\triangle ABC$, the sum of the 
distances from $P$ to $A,B,C$ is at least twice the sum of the 
distances from $P$ to $\line{BC},\line{CA},\line{AB}$. Furthermore, 
equality occurs only when $ABC$ is 
equilateral and $P$ is its center.
\end{theorem}
\begin{proof}
The unusually stringent equality condition should suggest that perhaps 
the proof proceeds in two stages, with different equality conditions. 
This is indeed the case.

Let $X,Y,Z$ be the feet of the respective
perpendiculars from $P$ to $\seg{BC},\seg{CA},\seg{AB}$. 
We will first prove that
\begin{equation} \label{eq:erdmo1}
PA \geq  \frac{AB}{BC} PY + \frac{AC}{BC} PZ.
\end{equation}
The only difference between most proofs of this theorem is 
in the proof of the above inequality. For example, rewrite 
(\ref{eq:erdmo1}) as
\[
PA \sin A \geq PY \sin C + PZ \sin B,
\]
recognize that $PA \sin A = YZ$ by the Extended Law of Sines, and 
observe that the right side is the length of the projection of $\seg{YZ}$ 
onto the line $\line{BC}$. Equality holds if and only if $\line{YZ}$ 
is parallel to 
$\line{BC}$.

Putting (\ref{eq:erdmo1}) and its analogues together, we get
\[
PA + PB + PC \geq PX \left( \frac{CA}{AB} + \frac{AB}{CA} \right)
+ PY \left( \frac{AB}{BC} + \frac{BC}{AB} \right)
+ PZ \left( \frac{BC}{CA} + \frac{CA}{BC} \right),
\]
with equality if and only if $\triangle 
XYZ$ is homothetic to $\triangle ABC$; this occurs 
if and only if $P$ is the circumcenter of $\triangle ABC$ (Problem~1).
Now for the second step: we note that each of the 
terms in parentheses is at least 2 by the AM-GM inequality. This gives
\[
PA + PB + PC \geq 2 (PX + PY + PZ),
\]
with equality if and only if $AB = BC = CA$.
\end{proof}

\begin{exer}
\ii
With notation as in the above proof,
show that the triangles $\triangle XYZ$ and 
$\triangle ABC$ are homothetic if and only 
if $P$ is the circumcenter of $\triangle ABC$.
\ii
Give another proof of (\ref{eq:erdmo1}) by comparing $P$ with its 
reflection across the angle bisector of $A$. (Beware: the reflection may 
lie outside of the triangle!)
\ii
Solve problem~\ref{ex:imo91} using the Erd\H{o}s-Mordell inequality.
\ii
(IMO 1996/5) \label{ex:imo965}
Let $ABCDEF$ be a convex hexagon such that $\seg{AB}$ is parallel to 
$\seg{DE}$, 
$\seg{BC}$ is parallel to $\seg{EF}$, and 
$\seg{CD}$ is parallel to $\seg{FA}$. Let $R_{A}, 
R_{C}, R_{E}$ denote the circumradii of triangles $\triangle FAB, 
\triangle BCD, \triangle DEF$, 
respectively, and let $P$ denote the perimeter of the hexagon. Prove 
that
\[
R_{A} + R_{C} + R_{E} \geq \frac P2.
\]
\ii (Nikolai Nikolov) \index{Nikolov, Nikolai}
The incircle $k$ of triangle $\triangle ABC$ touches the sides at points
$A_1, B_1, C_1$. For any point $K$ on $k$, let $d$ be the sum of the
distances from $K$ to the sides of the triangle $\triangle A_1B_1C_1$. Prove
that $KA + KB + KC > 2d$.
\end{exer}

\section{Additional problems}

Now it's your turn. Which technique(s) will help in the following instances?

\begin{exer}
\ii
Prove that of all quadrilaterals with a prescribed perimeter $P$, the 
square has the greatest area. Can you also prove the analogous result
for polygons with any number of sides?
\ii
What is the smallest positive real number $r$ such that a square of 
side length 1 can be covered by three disks of radius $r$?
\ii \label{ex:incircles}
Let $r$ be the inradius of triangle $\triangle ABC$. Let $r_A$ be the radius
of a circle tangent to the incircle as well as to sides 
$\seg{AB}$ and $\seg{CA}$.
Define $r_B$ and $r_C$ similarly. Prove that
\[
r_A + r_B + r_C \geq r.
\]
\ii
Prove that a triangle with angles $\alpha, \beta, \gamma$, 
circumradius $R$, and area $A$ satisfies
\[
\tan \frac{\alpha}{2} + \tan \frac{\beta}{2} + \tan \frac{\gamma}{2}
\leq \frac{9R^2}{4A}.
\]
\ii %% MOP 1995
Let $a,b,c$ be the sides of a triangle with inradius $r$ and 
circumradius $R$. Show that
\[
\left\lvert 1 - \frac{2a}{b+c} \right\rvert \leq \sqrt{1 - 
\frac{2r}{R}}.
\]
\ii %% MOP 1995
Two concentric circles have radii $R$ and $R_{1}$ respectively, where 
$R_{1} > R$. $ABCD$ is inscribed in the smaller circle and 
$A_{1}B_{1}C_{1}D_{1}$ in the larger one, with $A_{1}$ on the 
extension of $CD$, $B_{1}$ on that of $DA$, $C_{1}$ on that of $AB$, and 
$D_{1}$ on that of $BC$. Prove that the ratio of the areas of 
$A_{1}B_{1}C_{1}D_{1}$ and $ABCD$ is at least $R_{1}^{2}/R^{2}$.
\ii %% MOP 1995--posed for ratio 2, use Ptolemy
With the same notation, prove that the ratio of the perimeters of 
$A_{1}B_{1}C_{1}D_{1}$ and $ABCD$ is at least $R/r$.
\end{exer}

\part{Some roads to modern geometry}

\chapter{Inversive and hyperbolic geometry}
\label{chap:inversion}

One of the features of ``modern'' geometry is the inclusion of
transformations which are more drastic than those considered in
Chapter~\ref{chap:transform}. In this chapter, we consider some
transformations which preserve angles but not distances or areas
or even collinearity.
One singularly useful class of examples is the inversions; these give
simple proofs both of classic theorems and of competition problems.
Moreover, they can be used to give a simple derivation of the basic properties
of hyperbolic geometry.

The introduction of inversion requires new concepts of what a ``plane''
is and what ``transformations'' are. In particular, though inversion
does not preserve lines, it preserves angles in a sense we will make precise.
This puts inversion in the rich class of \emph{conformal transformations},
which play a key role in applications (e.g., in physics).
%Nowadays, one typically studies conformal transformations by identifying
%the Euclidean plane with the set of complex numbers and applying ideas
%from complex analysis; we end the chapter with
%a brief introduction to this point of view. (Incidentally, ideas from
%complex analysis also arise in algebraic geometry, which we will touch
%on briefly in Chapter~\ref{chap:projective}.)

\section{Inversion}

The notion of an inversion is a natural extension of the concept of reflection
across a line, once one accepts the idea that lines are really just
``circles of infinite radius''. Indeed, one can uniformly characterize lines
and circles using directed angles: given three points $A,B,C$, the set of
points $D$ for which $\dang ADB = \dang ACB$, together with $A,B,C$, forms
either a line or a circle. So it is not too much of a stretch to imagine
a ``reflection across a circle''; indeed, this thought seems to have
occurred to Apollonius of Perga, \index{Apollonius!of Perga} who is 
thought (by virtue of descriptions given by later authors)
to have introduced inversion in his lost treatise \emph{Plane Loci}.
However, only in modern times did the technique come into common currency;
the first surviving appearance of inversion seems to be in the work of
the Swiss geometer Jakob Steiner (1796--1863), \index{Steiner, Jakob}
some of whose profitable use of the technique we will see shortly.

Let $O$ be a point in the plane, and let $r$ be a positive real number. The 
\emph{inversion} \index{inversion|textbf}
with center $O$ and radius $r$ is the map of the plane minus the point $O$
to itself, carrying the point $P \neq O$ to the point $P'$ 
on the ray $\ray{OP}$ such that $OP \cdot OP' = r^2$. Since 
specifying a point and a positive real number is the same as 
specifying a circle (the point and the positive real corresponding to 
the center and radius, respectively, of the circle), we can also 
speak of inversion through a circle using the same definition. 
\begin{figure}[ht]
\caption{Inversion through a circle.}
\end{figure}

What happens to the point $O$ under inversion? 
Points near $O$ get sent very far 
away, in all different directions, so there is no good place to put 
$O$ itself. To rectify this, we define
the \emph{inversive plane} \index{inversive!plane|textbf}
\index{plane!inversive|textbf} as the usual plane with one 
additional point $\infty$, thought of as being a ``point at infinity''.
\index{point at infinity} We view  
an inversion centered at $O$ as a transformation on 
the entire inversive plane by declaring that $O$ and 
$\infty$ are inverses of each other.

As an aside, we note a natural interpretation of the inversive plane. 
Under stereographic projection (used in some maps), 
\index{stereographic projection}
\index{projection!stereographic}
the surface of a 
sphere, minus the North Pole, is mapped to a plane tangent to the 
sphere at the South Pole as follows: a point on the sphere maps to 
the point on the plane collinear with the given point and the North 
Pole. Then the point at infinity corresponds to the North Pole, and 
the inversive plane corresponds to the whole sphere. In fact, 
inversion through the South Pole with the appropriate radius 
corresponds to reflecting the sphere through the plane of the equator! 
\begin{figure}[ht]
\caption{A stereographic projection.}
\end{figure}

Returning to Euclidean geometry, we now establish some important 
properties of inversion. We first make an easy but important 
observation.

\begin{fact} \label{fac:invsim}
If $O$ is the center of an inversion taking $P$ to $P'$ and $Q$ to 
$Q'$, then the triangles $\triangle OPQ$ and $\triangle OQ'P'$ 
are oppositely similar.
\end{fact}
In particular, we have that $\dang OP'Q' = -\dang OQP$, a fact 
underlying our next proof.

By an \emph{inversive circle} \index{inversive!circle|textbf}
\index{circle!inversive|textbf} in the inversive plane, we will mean either 
a circle in the Euclidean plane, or a line in the Euclidean plane together
with the extra point $\infty$.
\begin{theorem} \label{thm:invcirc}
The image of an inversive circle under an inversion is an inversive circle.
\end{theorem}
\begin{proof}
Let $A,B,C,D$ be four points on an inversive circle, and 
let $A',B',C',D'$ be the respective images of $A,B,C,D$ under an
inversion with center $O$. We now chase directed angles, using 
the similar triangles of Fact~\ref{fac:invsim}:
\beqa
\dang A'B'C' &=& \dang A'B'O + \dang OB'C' \\
&=& \dang BAO + \dang OCB \\
&=& \dang BAD + \dang DAO + \dang OCD + \dang DCB \\
&=& \dang DAO + \dang OCD \\
&=& \dang A'D'O + \dang OD'C' \\
&=& \dang A'D'C'.
\eeqa
We see that $A',B',C',D'$ lie on an inversive circle as well.
\begin{figure}[ht]
\caption{Proof of Theorem~\ref{thm:invcirc}.}
\end{figure}
\end{proof}

Notice the way the angles are broken up and recombined in the above 
proof. In some cases, inversion can turn a constraint involving two or 
more angles in different places into a constraint about a single 
angle, which then is easier to work with. Some examples can be found 
in the problems.

Inversion also turns out to ``reverse the angles between lines''.
Since lines are 
sent to circles in general, we will have to define the angle between 
two circles to make sense of this statement. 

Given two inversive 
circles $\omega_{1}$ and $\omega_{2}$, the (directed) angle between 
them \index{angle!between two inversive circles|textbf}
at one of their intersections $P$ is defined as the (directed) 
angle from the tangent to $\omega_{1}$ at $P$ to the tangent of 
$\omega_{2}$ at $P$. We say that two inversive 
circles are \emph{orthogonal} \index{orthogonality!of inversive
circles|textbf}
if the angle between them is $\pi/2$.
Note that absent a choice between the two points of intersection,
the angle between two 
circles is only well-defined up to sign as an angle modulo $\pi$;
however, orthogonality does not depend on this choice. Note also
that a line and a circle are orthogonal if and only if the line passes
through the center of the circle.

\begin{fact} \label{thm:invang}
The directed angle between circles (at a chosen intersection)
is reversed under inversion.
\end{fact}

Distances don't fare as well under inversion, but one can say 
something using Fact~\ref{fac:invsim}.
\begin{fact}[Inversive distance formula] \label{thm:invdist}
If $O$ is the center of an inversion of radius $r$ sending $P$ to $P'$ 
and $Q$ to $Q'$, then
\[
P'Q' = PQ \cdot \frac{r^2}{OP \cdot OQ}.
\]
\end{fact}

\begin{exer}
\ii
Deduce Theorem~\ref{thm:invcirc} from Problem~\ref{ex:invcirc}
(or use the above proof to figure out how to do that problem).
\ii
Give another proof of Theorem~\ref{thm:invcirc} using the converse of
the power-of-a-point theorem (Fact~\ref{fact:power}) and
Fact~\ref{thm:invdist}.
\ii
The angle between two lines through the origin is clearly preserved 
under inversion. Why doesn't this contradict the fact that inversion 
reverses angles?
\ii
(IMO 1996/2)
Let $P$ be a point inside triangle $\triangle ABC$ such that
\[
\angle APB - \angle ACB = \angle APC - \angle ABC.
\]
Let $D,E$ be the incenters of triangles $\triangle APB, \triangle APC$, 
respectively. 
Prove that $\seg{AP}, \seg{BD}, \seg{CE}$ 
meet in a point. (Many other solutions are 
possible; over 25 were submitted by contestants at the IMO!)
\ii (IMO 1998 proposal)
Let $ABCDEF$ be a convex hexagon such that $\angle B + \angle D + 
\angle F = 360^{\circ}$ and
\[
\frac{AB}{BC} \cdot \frac{CD}{DE} \cdot \frac{EF}{FA} = 1.
\]
Prove that
\[
\frac{BC}{CA} \cdot \frac{AE}{EF} \cdot \frac{FD}{DB} = 1.
\]
\ii
Prove that the following are equivalent:
\begin{enumerate}
\item
The points $A$ and $B$ are inverses through the circle $\omega$.
\item
The line $\line{AB}$ and the circle with diameter $\seg{AB}$ are both
orthogonal to $\omega$.
\item
$\omega$ is a circle of Apollonius with respect to $A$ and $B$.
\end{enumerate}
In particular, conclude that 
a circle 
distinct from $\omega$ is fixed (as a whole, not pointwise) by
inversion through $\omega$
if and only if it is orthogonal to $\omega$.
\ii
Give yet another proof of Theorem~\ref{thm:invcirc} using
complex numbers and the circle of Apollonius
(Theorem~\ref{thm:ap}).

\ii
Show that a set of circles is coaxial if and only if there is a circle 
orthogonal to all of them. Deduce that
coaxial circles remain that way under inversion. Also, try drawing 
a family of coaxial circles and some circles orthogonal to them; the 
picture is very pretty.

\ii \label{ex:conc}
Prove that any two nonintersecting circles can be inverted into 
concentric circles. (This will be used in Theorem~\ref{thm:steiner} below.)
\end{exer}

\section{Inversive magic}
\label{sec:inv magic}

As noted earlier, we know about inversion largely through the work of
Jakob Steiner. \index{Steiner, Jakob} Steiner used the technique to give
dazzlingly simple proofs of a number of difficult-looking statements.
Here we present but a few examples.

We start with a classical result attributed to Pappus of Alexandria. 
\index{Pappus!of Alexandria}
It is one of a number of results concerning a figure bounded by three
semicircles with diameters $\seg{AB},\seg{BC},\seg{AC}$, where $A,B,C$
are three points lying on a line in that order. Such a figure was first
consirede by Archimedes, who called it an \emph{arbelos}\footnote{The word 
``arbelos'' in Greek refers to a shoemaker's knife, which presumably looked
something like the figure Archimedes was considering.}. 

\begin{theorem} [Pappus] \label{thm:papinv}
Let $\omega$ be a semicircle with diameter $\seg{AB}$. Let $\omega_{1}$ and 
$\omega_{2}$ be two semicircles externally tangent to each other at $C$, and 
internally tangent to $\omega$ at $A$ and $B$, respectively. 
Let $C_{1}, C_{2}, \dots$ be a sequence of circles, each tangent to 
$\omega$ and $\omega_{1}$, such that $C_{i}$ is tangent to $C_{i+1}$ 
and $C_{1}$ is tangent to $\omega_{2}$ (as in Figure~\ref{fig:arbelos}).
Let $r_{n}$ 
be the radius of $C_{n}$, and let $d_{n}$ be the distance from the center of 
$C_{n}$ to $\line{AB}$. Then for all $n$,
\[
d_{n} = 2n r_{n}.
\]
\end{theorem}
\begin{figure}[ht]
\caption{An arbelos, and a theorem of Pappus.}
\label{fig:arbelos}
\end{figure}
\begin{proof}
Perform an inversion with center $A$, and choose the radius of 
inversion so that $C_{n}$ remains fixed. Then $\omega$ and 
$\omega_{1}$ map to lines perpendicular to $\line{AB}$ and tangent to 
$C_{n}$, and $C_{n-1}, \dots, C_{1}$ to a column of circles between 
the lines, with $\omega_{2}'$ at the bottom of the column. The 
relation $d_{n} = 2n r_{n}$ is now obvious.
\begin{figure}[ht]
\caption{Proof of Theorem~\ref{thm:papinv}.}
\end{figure}
\end{proof}

The following theorem is known as \emph{Steiner's porism}.
\index{porism!Steiner's|textbf} \index{Steiner's porism|textbf}
\begin{theorem}\label{thm:steiner}
Suppose two nonintersecting circles have the property that one can 
fit a ``ring'' of $n$ circles between them, each tangent to the next.
Then one can do this starting with any circle 
tangent to both given circles.
\end{theorem}
\begin{proof}
By Problem~\ref{ex:conc}, a suitable inversion takes the given circles to 
concentric circles, while preserving tangency of circles. The result 
is now obvious.
\begin{figure}[ht]
\caption{Steiner's porism (Theorem~\ref{thm:steiner}) and its proof.}
\end{figure}
\end{proof}

\begin{exer}
\ii 
Suppose that, in the hypotheses of Pappus's theorem, we assume that
$C_0$ is tangent to $\omega, \omega_1$ and the line $\line{AB}$ (instead of
the semicircle $\omega_2$). Show that in this case $d_n = (2n+1) r_n$. 
\ii (Romania, 1997) \label{ex:rom97}
Let $\omega$ be a circle, and let $\line{AB}$ be
a line not intersecting $\omega$. 
Given a point $P_0$ on $\omega$, define the sequence $P_0, P_1, \dots$
as follows: $P_{n+1}$ is the second intersection with $\omega$ of the line
$\line{BQ_n}$, where $Q_n$ is the second intersection of the line 
$\line{AP_n}$ with $\omega$.
Prove that for a positive integer $k$, if $P_0 = P_k$ for some choice of
$P_0$, then $P_0 = P_k$ for any choice of $P_0$.
\end{exer}

\section{Inversion in practice}

So far we have seen that inversion can be used to give spectacular proofs
of a few results. However, it is much more useful than that; it can often
be applied to solve much more mundane problems. The 
paradigm for doing this is almost always the following: 
invert the given picture and its conclusion, thus transforming the
original problem into a new problem on a new diagram, then
solve the new problem. In some cases, one must also 
superimpose the original and
inverted diagrams (as in the proof of Theorem~\ref{thm:papinv}) and/or compare
information in the two diagrams (e.g.\ using Fact~\ref{thm:invdist}).

A general principle behind this method 
is that it is easier to deal with lines than circles.
Hence if one wishes to perform an inversion on a geometric diagram, one
should  center the inversion at a point which is ``busy'' 
\index{point!busy} \index{busy
point}
in the sense of having many relevant circles and lines passing through it.

\begin{exer}
\ii
Make up an inversion problem by reversing the paradigm: start with a 
result that you know, invert about some point, and see what you get.
The tricky part is choosing things well enough so that the resulting 
problem doesn't have an obvious busy point; such a problem would be too easy!

\ii
Let $C_1, C_2, C_3, C_4$ be circles such that $C_{i}$ and $C_{i+1}$
are externally tangent for $i=1, 2,3,4$ (where $C_5=C_1$). Prove
that the four points of tangency are concyclic.

\ii \label{ex:rom97b}
(Romania, 1997)
Let $\triangle ABC$ be a triangle, let $D$ be a point on side $\seg{BC}$,
 and let $\omega$ be the
circumcircle of $\triangle ABC$. 
Show that the circles tangent to $\omega, \seg{AD}, \seg{BD}$
and to $\omega, \seg{AD}, \seg{DC}$ 
are tangent to each other if and only if
$\angle BAD = \angle CAD$.
\begin{figure}[ht]
\caption{Problem~\ref{ex:rom97b}.}
\end{figure}

\ii
(Russia, 1995)
%% (Pamphlet, Russia 33)
Draw a semicircle with diameter $\seg{AB}$ and center $O$, then draw a line
which intersects the semicircle at $C$ and $D$ and which
intersects line $\line{AB}$ at $M$,
such that $MB < MA$ and $MD < MC$. 
Let $K$ be the second point of intersection of
the circumcircles of triangles $\triangle AOC$ and $\triangle DOB$. 
Prove that $\angle MKO$ is a right angle.

\ii
(USAMO 1993/2)
Let $ABCD$ be a convex quadrilateral with perpendicular
diagonals meeting at $O$. Prove that the reflections of $O$
across $\line{AB}$, $\line{BC}$, $\line{CD}$, $\line{DA}$ 
are concyclic. (For an added challenge, find a 
non-inversive proof as well.)

\ii \label{ex:apoll}
(Apollonius's problem)
Given three nonintersecting circles, how many circles are
tangent to all three? And how can they be constructed with
straightedge and compass?

\ii
%% (Test 3, problem 4)
(IMO 1994 proposal)
The incircle of $\triangle ABC$ touches $\seg{BC}, \seg{CA}, \seg{AB}$ 
at $D,E,F$, respectively.
Let $X$ be a point inside $\triangle ABC$ 
such that the incircle of $\triangle XBC$ touches
$\seg{BC}$ at $D$ also, and touches $\seg{CX}$ and $\seg{XB}$ 
at $Y$ and $Z$,
respectively. Prove that $EFZY$ is a cyclic quadrilateral.

\ii
(Israel, 1995)
Let $\seg{PQ}$ be the diameter of semicircle $H$. The circle $O$ is 
internally tangent to $H$ and is tangent to $\seg{PQ}$ at $C$. Let $A$
be a point on $H$, and let $B$ be a point on $\seg{PQ}$ 
such that $\line{AB}$ is perpendicular
to $\line{PQ}$ and is also tangent to $O$. Prove that $\line{AC}$ 
bisects $\angle PAB$.

\ii \label{ex:ptineq}
Give an inversive proof of Ptolemy's inequality
(Theorem~\ref{thm:ptineq}).

\ii
(IMO 1993/2)
Let $A,B,C,D$ be four points in the plane, with $C,D$ on the
same side of line $\line{AB}$, such that $AC \cdot BD = AD \cdot BC$
and $\angle ADB = \pi/2 + \angle ACB$. Find the ratio $(AB \cdot
CD)/(AC \cdot BD)$ and prove that the circumcircles of triangles
$\triangle ACD$ and $\triangle BCD$ are orthogonal.

\ii \label{ex:iran95}
(Iran, 1995)
Let $M,N,P$ be the points of intersection of the incircle of
$\triangle ABC$ with sides $BC,CA,AB$, respectively. Prove that
the orthocenter of $\triangle MNP$, the incenter of $\triangle ABC$,
and the circumcenter of $\triangle ABC$ are collinear. 

% MOP 1997 Test 3
\ii (MOP 1997) \label{ex:mop97}
Let $\triangle ABC$ be a triangle and let $O$ be
its circumcenter. The lines $\line{AB}$ and
$\line{AC}$  meet the circumcircle of triangle $\triangle 
BOC$ again at $B_1$ and
$C_1$, respectively. Let $D$ be the intersection of lines $\line{BC}$ and
$\line{B_1 C_1}$. Show that the circle tangent to $\line{AD}$ at $A$ and
 having its
center on $\line{B_1C_1}$ is orthogonal to the circle with 
diameter $\seg{OD}$.

\ii
(Russia, 1993) \label{ex:rus93invert}
Let $ABCD$ be a convex cyclic quadrilateral, and let $O$ be the intersection
of diagonals $\seg{AC}$ and $\seg{BD}$. 
Let $\omega_1$ and $\omega_2$ be the circumcircles
of triangles $\triangle ABO$ and $\triangle CDO$, respectively, and let
$\omega_1$ and $\omega_2$ meet at $O$ and $K$. The line through $O$
parallel to $\line{AB}$ 
meets $\omega_1$ again at $L$, and the line through $O$
parallel to $\line{CD}$ meets $\omega_2$ again at $M$.
Let $P$ and $Q$ be points on segments $\seg{OL}$ and 
$\seg{OM}$, respectively, such that
$OP/PL = MQ/QO$. Prove that $O,K,P,Q$ lie on a circle.

\end{exer}

\section{Hyperbolic geometry: an historical aside}

One cannot give a survey of ``modern'' geometry without including
hyperbolic, or non-Euclidean, geometry. Originally viewed as a
pathological construction, it was later realized in several ways \emph{within}
the confines of Euclidean geometry, and thus is no less valid! Subsequently,
hyperbolic geometry has become omnipresent within mathematics, and even
within physics via Einstein's \index{Einstein, Albert} theory of relativity.
\index{relativity}

To understand the relevance of hyperbolic geometry, we must momentarily
overturn
our revisionist construction of the Euclidean plane and go back to the
axiomatic definition. It relies on five postulates, which we loosely
translate into modern language.
\begin{enumerate}
\item
Any two points are the endpoints of a line segment.
\item
Any line segment can be extended to a straight line.
\item
There exists a circle with any given radius and center.
\item
Any two right angles are congruent to each other.
\item
If two lines intersect a third and the interior angles on one side are
both less than $\pi$, then the two lines intersect somewhere on that side
of the third line.
\end{enumerate}

One cannot know whether Euclid realized it was necessary to include the
fifth postulate, the so-called \emph{parallel postulate}.
\index{parallel postulate|textbf} For many centuries, it was felt that
Euclid had simply fallen short in simplifying the axioms, and that it
would be possible to deduce the parallel postulate from the other four.
It was finally realized by Gauss \index{Gauss (Gau\ss), (Johann) Carl Friedrich}
that this is impossible, as there is
actually a perfectly sensible (if highly counterintuitive)
geometry in which the parallel postulate
failed while the other postulates continue to hold. 
Gauss, no fan of controversy\footnote{Although
the consistency of non-Euclidean geometry is a nonissue for mathematicians
in our time,
it continues to cause controversy among nonmathematicians, who have trouble
shaking the belief that there could be any alternative to the Euclidean
setting. An infamous example is columnist Marilyn vos Savant, 
\index{vos Savant, Marilyn} who published
a notorious book attacking the work of Wiles on Fermat's Last Theorem
on precisely these long-discredited grounds.},
never published his findings, leaving
them to be rediscovered independently by J\'anos Bolyai (1802--1860)
\index{Bolyai, J\'anos}
and Nikolai Ivanovich Lobachevsky\footnote{There has been some dispute
over whether these rediscoveries were truly independent of each other
and of Gauss. This dispute is satirized in a famous
song by mathematician/satirist Tom Lehrer. \index{Lehrer, Tom}}
 (1792--1856). \index{Lobachevsky, Nikolai Ivanovich}

As noted above, one proves the independence of the parallel postulate
by constructing a ``model geometry'' in which the parallel postulate
fails while the other postulates continue to hold. This is done by
building a geometric situation and carefully relabeling the objects; 
we will do this in the next section.

\section{Poincar\'e's models of hyperbolic geometry}

As described in the previous section, one typically builds spaces of
``hyperbolic geometry'' by realizing them using constructions within
Euclidean geometry.
We now describe two related methods for doing this, introduced by
Henri Poincar\'e (1854--1912) \index{Poincar\'e, Jules Henri}.

In the \emph{disc model}
\index{disc!model (of hyperbolic geometry)|textbf}, we take the underlying
set of points to be the interior of an open disc in the Euclidean plane. 
The lines of the disc model are the lines
and circles orthogonal to the boundary of the disc, or rather the pieces of 
these lying within the disc. See Figure~\ref{fig:disc model}.

\begin{figure}[ht] 
\caption{Points and lines in the disc model.}
\label{fig:disc model}
\end{figure}

In the \emph{halfplane model}
\index{halfplane!model (of hyperbolic geometry)|textbf}, we take the 
underlying set of points to be those $(x,y)\in \RR^2$ with $y>0$.
The lines of the disc model are the lines and circles orthogonal to the
$x$-axis, or rather the pieces of these lying within the halfplane.
\begin{figure}[ht] 
\caption{Points and lines in the halfplane model.}
\label{fig:halfplane model}
\end{figure}

At this point it is clear that one can transform the disc model and the
halfplane model into each other by inversion: inverting through a point
on the boundary of the disc turns the disc model into the halfplane model,
and one gets back by inverting through a point below the $x$-axis. We may
thus safely refer to either one as the \emph{hyperbolic plane}
\index{hyperbolic!plane|textbf} \index{plane!hyperbolic|textbf}, as
long as we only refer to concepts which carry identical meanings in both
model. This includes lines of the model; 
by virtue of the angle-preserving property of inversion, this also
includes angles between lines.

\begin{fact}
Any two points in the hyperbolic plane lie on a unique line.
Any two lines in the hyperbolic plane intersect in at most one point.
\end{fact}
By contrast with the Euclidean plane, there are multiple lines through
a given point which fail to intersect a given line not through that point.
That is, there are multiple parallels to a given line through a given
point not on the line.
\begin{figure}[ht]
\caption{Multiple parallels in the hyperbolic plane.}
\end{figure}

It is reasonable to ask why one needs two (or more, but two will suffice
for now) different models of
the hyperbolic plane. One answer is that different
symmetries appear more readily in each model, so having multiple models
makes it easier to visualize the full set of symmetries of the hyperbolic
plane. We illustrate this with an example.

By a \emph{hyperbolic transformation}, \index{hyperbolic!transformation|textbf}
\index{transformation!hyperbolic|textbf} 
we will mean a bijection
from the hyperbolic plane to itself carrying lines to lines and preserving
angles.
For example, in the disc model, one may rotate around the center of the disc;
in the halfplane model, one may make a horizontal translation or a 
homothety with positive ratio and center on the $x$-axis.

\begin{theorem}\label{thm:hyperbolic transf}
\begin{enumerate}
\item[(a)]
There exists a hyperbolic transformation carrying any given point to any
other given point.
\item[(b)]
There exists a hyperbolic transformation carrying any given line to any
other given line.
\item[(c)]
Given a point $P$ in the hyperbolic plane, there exists a hyperbolic
transformation carrying any given line through $P$ to any other given line
through $P$.
\end{enumerate}
\end{theorem}
\begin{proof}
Fix an isomorphism between the disc model and the halfplane model.
Let $O$ be the center of the disc model, and let $O'$ be its image in
the halfplane model.
\begin{enumerate}
\item[(a)]
This is clear in the halfplane model, using dilations and translations.
\item[(b)]
On one hand,
any line in the halfplane model can be translated
to a line through $O'$. On the other hand, any two lines through $O$
in the disc model are related by a rotation. This yields the claim.
\item[(c)] 
By (b), it suffices to check this for the point $O$ in the disc model,
which is clear.
\end{enumerate}
\end{proof}

\section{Hyperbolic distance}

We next wish to define a notion of distance between two points
in the hyperbolic plane. Before doing this, we first check that there
are not ``too many'' hyperbolic transformations.
\begin{theorem} \label{thm:hyperbolic line}
Let $A,B,C$ be three distinct 
points in the hyperbolic plane, which lie on a line
in that order. Then there is no hyperbolic transformation fixing
$A$ and taking $B$ to $C$.
\end{theorem}
\begin{proof}
By Theorem~\ref{thm:hyperbolic transf}, we may reduce to the case
where $A$ is the center of the disc model and the line $\ell$
through $A,B,C$ is diametric. Let $\ell_1, \ell_2$ be the lines through $B,C$,
respectively, perpendicular to $\ell$. Let $m$ be a line through $A$
such that in the
Euclidean plane, the extensions of $\ell_2$ and $m$ meet on the boundary of the
disc. Note that $m$ does not meet $\ell_2$ in the hyperbolic plane, but it
does meet $\ell_1$.
\begin{figure}[ht]
\caption{Proof of Theorem~\ref{thm:hyperbolic line}.}
\end{figure}

Suppose now that there is a hyperbolic
transformation fixing $A$ and taking $B$ to $C$. Then $\ell$ maps to 
$\ell$, and by angle preservation, $\ell_1$ maps to $\ell_2$. Again by
angle preservation, $m$ maps either to $m$ or to its reflection across $\ell$.
In either case, the two intersecting lines $m, \ell_1$ are carried to 
nonintersecting lines, contradiction.
\end{proof}

Let $A,B$ be two distinct points in the hyperbolic plane; we now define
the \emph{distance} $d_h(A,B)$ \index{distance!hyperbolic|textbf} as follows.
(If $A=B$, we just set $d_h(A,B) = 0$.)
Apply Theorem~\ref{thm:hyperbolic transf} to map $A,B$ to points
$A',B'$ which lie on a vertical line in the halfplane model. Without loss of
generality, we may assume $A'$ lies below $B'$. Let $d_A, d_B$ be the
Euclidean distances between $A,B$ and the $x$-axis, and define
\[
d_h(A,B) = \log d_B - \log d_A;
\]
this is unambiguous by Theorem~\ref{thm:hyperbolic line}, since we cannot use
a hyperbolic transformation to move $B'$ up or down the line while fixing
$A'$. 

\begin{fact}
The hyperbolic distance satisfies the following properties.
\begin{enumerate}
\item[(a)]
If points $A,B,C$ lie on a line in that order,
then $d_h(A,C) = d_h(A,B) + d_h(B,C)$. 
\item[(b)]
For any points $A,B,C,D$, we have $d_h(A,B)=d_h(C,D)$ if and only if there is
a hyperbolic transformation sending $A$ to $B$ and sending $C$ to $D$.
\end{enumerate}
\end{fact}
Notice something funny going on: the definition of a hyperbolic transformation
only required preservation of lines and angles, and yet these also preserve
distances. This is in contrast with the Euclidean plane, where there is
a clear distinction between similarities and rigid motions. Somehow the
hyperbolic plane has an inherent ``sense of scale'' that the Euclidean
plane does not; this can be explained by formalizing the statement that
the ``curvature'' of the Euclidean plane is zero but that of the
hyperbolic plane is nonzero.

\begin{exer}
\item
Give a formula to compute distances in the disc model.
\item
Prove that any line in the hyperbolic plane contains pairs of points
whose distance is arbitrarily large; i.e., the length of a line is infinite.
\item
Prove that any map of the hyperbolic plane to itself that carries lines to
lines preserves (undirected) angles, and hence is a hyperbolic transformation.
That is, there is no analogue in the hyperbolic plane of affine transformations
which are not rigid motions.
\end{exer}

\section{Hyperbolic triangles}

A \emph{(line) segment} in the hyperbolic plane will be the segment
or arc between two points on a hyperbolic line; we refer to the distance
between the two endpoints also as the \emph{length} of the segment.
With this definition, we may now speak about polygons in the hyperbolic
plane, and in particular of triangles.

\begin{theorem} \label{thm:angular defect}
The sum of the angles in a hyperbolic triangle is always strictly
less than $\pi$.
\end{theorem}
\begin{proof}
Again, fix an isomorphism between the disc model and the halfplane model.
Let $O$ be the center of the disc model, and let $O'$ be its image in
the halfplane model.
Given a triangle in the halfplane model, we can apply dilations and 
translations corresponding to hyperbolic transformations to create a
congruent triangle with $O'$ in its interior.
Then the result is clear: each angle of the hyperbolic triangle is
less than the corresponding angle of the ordinary triangle with the
same vertices.
\begin{figure}[ht]
\caption{Proof of Theorem~\ref{thm:angular defect}.}
\end{figure}
\end{proof}
The difference between $\pi$ and the sum of the angles in a hyperbolic
triangle is called the \emph{angular defect} \index{angular defect|textbf}
of the triangle. It is additive in the sense that if
$A,B,C$ are three points in the hyperbolic plane, and $D$ is a point
on the hyperbolic segment $BC$, then the angular defect of the hyperbolic
triangle $ABC$ is the sum of the angular defects of $ABD$ and $ADC$.
It can thus be used as a measure of area for hyperbolic triangles.
\begin{figure}[ht]
\caption{Additivity of angular defect.}
\end{figure}

\begin{exer}
\item
Prove that 
any two hyperbolic triangles which have the same angles
are congruent. Yes, you read that correctly!
This is another case where the hyperbolic plane exhibits
an intrinsic ``sense of scale''.
\end{exer}

\chapter{Projective geometry}
\label{chap:projective}

Projective geometry is the study of geometric properties which are invariant
under ``changes of perspective''; this eliminates properties like angles
and distances but retains properties like collinearity and concurrence.
The formalism of projective geometry makes a discussion 
of such properties possible, and exposes some remarkable facts, such 
as the duality of points and lines.

The history of projective geometry is a remarkable instance of art
and science feeding off one 
another.\footnote{The MacTutor archive, mentioned in the introduction,
includes a nice description of this history.}
Based on the optics studies of the Arabic mathematician
Alhazen (Ibu Ali al-Hasan ibn al-Haytham) (965--1040),
\index{Alhazen} \index{al-Haytham, Ibu Ali al-Hasan ibn}
several early Renaissance artists\footnote{To be fair, the distinction
between artists and scientists was somewhat blurred at this period,
whence the modern phrase ``Renaissance man'' for a versatile individual.}
attempted to develop a style
of visual depiction that presented the eye with a truer semblance of
three-dimensional space than did earlier, flatter styles.
The discovery of the principle of linear perspective (the idea that
all parallel lines appear to converge at a single point) is credited
to Filippo Brunelleschi (1377--1446). \index{Brunelleschi, Filippo}
This led to a flurry of activity, culminating in the work of
Girard Desargues (1591--1661), \index{Desargues!Girard} which
introduced projective geometry as we now it.

In the modern era, the real power of projective geometry lies within
the realm of algebraic geometry, i.e.,
the study of geometric objects defined
by polynomial equations. This study, implicit in the coordinate
geometry with which this book begins, took off in earnest late in the 19th
century, and remains one of the most vital branches of present-day
mathematics research. We end the chapter with a glimpse in this direction.

\section{The projective plane}

We begin with a lengthy description of the formalism of the 
projective plane. The impatient reader may wish to read only the next 
paragraph at first, then skip to the later sections and come back to 
this section as needed.

The \emph{projective plane} \index{projective plane|textbf} 
\index{plane, projective|textbf} 
consists of the standard Euclidean plane, 
together with a set of points called \emph{points at infinity}, 
\index{point at infinity|textbf} one 
for each collection of parallel lines. We say that a line passes 
through the point at infinity corresponding to its direction (and no 
others), and that all of the points at infinity lie on a \emph{line 
at infinity}. \index{line at infinity|textbf}
Note that three parallel lines now indeed have a common 
point at infinity, which retroactively justifies our calling such 
lines ``concurrent''.

An alternate description of the projective plane turns out to be 
quite useful, and corresponds more closely to the artists' 
conception. View the Euclidean plane as some plane in 
three-dimensional space, and fix a point $O$ not on the plane 
(corresponding to the eye). Then each point on the plane corresponds 
to a line through $O$ passing through that point, but not all lines 
through $O$ correspond to points on the Euclidean plane. In fact, they 
correspond to the points at infinity. In other words, we can identify 
the projective plane with the set of lines in space passing through a 
fixed point.

This description also yields a natural coordinate system for the 
projective plane, using what are known as \emph{homogeneous 
coordinates}. \index{homogeneous coordinates|textbf}
Each point in the projective plane can be specified with 
a triple of numbers $[x:y:z]$, where $x,y,z$ are not all zero. Be 
careful, though: for any nonzero real number $\lambda$, $[x:y:z]$ 
and $[\lambda x: \lambda y: \lambda z]$ are the same point! (Hence 
the name ``homogeneous coordinates''.) The colons are meant to remind 
you that it is the ratios between the coordinates that are 
well-defined, not the individual coordinates themselves.

How are homogeneous coordinates related to the usual Cartesian 
coordinates on the Euclidean plane? If we embed the Euclidean plane 
in space as the plane $z = 1$, then the point with Cartesian 
coordinates $(x,y)$ has homogeneous coordinates $[x:y:1]$, and the 
points at infinity are the points of the form $[x:y:0]$ for some 
$x,y$ not both zero.

\section{Projective transformations}
\label{sec:proj trans}

\index{transformation!projective|textbf}
\index{projective transformation}
The original definition of a projective transformation corresponded 
to the process of projecting an image in the ``real world'' onto an 
artist's canvas. Again, fix a point $O$ in three-dimensional space, 
and now select two planes not passing through $O$. The mapping that 
takes each point $P$ on the first plane to the intersection of the 
line $\line{OP}$ with the second plane was defined as a projective 
transformation. (Do you see why this map makes sense over the whole 
projective plane?)

One can also give an algebraic description of projective 
transformations that accommodates degenerate cases slightly more 
easily. In terms of homogeneous coordinates, a projective 
transformation takes the form
\[
[x : y : z] \mapsto [ax + by + cz : dx + ey + fz: gx + hy + iz],
\]
where $a, \dots, i$ are numbers such that the $3 \times 3$ matrix
\[
\begin{pmatrix} a & b & c \\ d & e & f \\ g & h & i \end{pmatrix}
\]
is invertible, i.e., its determinant
\[
aei + bfg + cdh - ceg - afh - bdi
\]
is nonzero. From this description it is clear that affine transformations 
are projective as well, since they occur when $g = h = 0$. Since we 
have two additional parameters (it looks like three, but by 
homogeneity one parameter is superfluous), the following analogue of 
Fact~\ref{ex:aff} is no surprise.
\begin{fact}
Any four points, no three collinear, can be mapped to any other four 
such points by a unique projective transformation.
\end{fact}

The most common use of a projective transformation in problem-solving 
is to map a particular line to the point at infinity. (As with 
inversion, it pays to look for a ``busy'' line for this purpose.) If the 
statement to be proved is well-behaved under projective 
transformations, this can yield drastic simplifications. The 
``well-behaved'' concepts mainly consist of incidence properties 
between points and lines (concurrence, collinearity, and the like); 
as with affine transformations, angles and distances are not 
preserved, nor are areas or ratios of lengths along segments (unlike 
affine transformations).

We demonstrate the power of projection by reproving
Desargues's theorem (Theorem~\ref{thm:desargues}).
\index{Desargues!theorem} \index{theorem!Desargues's}
\begin{theorem}
Given triangles $\triangle ABC$ and $\triangle DEF$,
the points $\line{AB} \cap \line{DE}$, 
$\line{BC} \cap \line{EF}$, $\line{CA} \cap \line{FD}$ are 
collinear if and only if the lines $\line{AD}, \line{BE}, \line{CF}$ 
are concurrent.
\end{theorem}
\begin{proof}
Apply a projective transformation to place the 
points $\line{AB} \cap \line{DE}$ and $\line{BC} \cap \line{EF}$ 
on the line at infinity. 
If triangles $\triangle ABC$ and $\triangle DEF$ are perspective from 
a line, they now have parallel sides and so are homothetic; thus
the lines $\line{AD}, \line{BE}, \line{CF}$ concur 
at the center of homothety (or at a point at infinity, in case $\triangle 
ABC$  and $\triangle DEF$ are now 
congruent). Conversely, if the lines $\line{AD}, \line{BE}, \line{CF}$ 
concur 
at $P$, consider the homothety centered at $A$ carrying $A$ to $D$. It 
preserves the line $\line{BE}$ and carries the line $\line{AB}$ 
to the parallel 
line $\line{DE}$ through $B$, so it maps $D$ to $E$. Similarly, the 
homothety maps $C$ to $F$, and so $\line{CA}$ and $\line{FD}$ 
are also parallel, 
implying that the points $\line{AB} \cap \line{DE}$, 
$\line{BC} \cap \line{EF}$, $\line{CA} \cap \line{FD}$
are collinear along the line at infinity.
\end{proof}

Beware that angles, circles, and other ``metric'' objects are not
preserved under projection; we will learn more about getting around
this difficulty later in the chapter.

\begin{exer}
\ii
Use a projective transformation to give an alternate proof of Pappus's 
theorem.
\ii \label{ex:cencirc}
Prove that the center of a circle drawn in the plane cannot be
constructed with straightedge alone.
\ii \label{ex:pentagram}
(Original)
Let $ABCDE$ be the vertices of a convex
pentagon, and let $F = \seg{BC} \cap \seg{DE}$, 
$G = \seg{CD} \cap \seg{EA}$, $H = \seg{DE} \cap \seg{AB}$,
$I = \seg{EA} \cap \seg{BC}$, $J = \seg{AB} \cap \seg{CD}$. 
Show that $\line{BD} 
\cap \line{CE}$ lies on the line $\line{AF}$ if and only if $\line{GH} \cap 
\line{IJ}$ does. 
\begin{figure}[ht]
\caption{Problem~\ref{ex:pentagram}.}
\end{figure}
\end{exer}

\section{A conic section}
\label{sec:con}

We now introduce the notion of a conic section, which comes to us
from the work of the ancient geometer Apollonius \index{Apollonius!of Perga}
(whose name has arisen already in connection with
Theorem~\ref{thm:ap}).

A \emph{conic section} \index{conic (section)|textbf}
is classically defined as a cross-section of a 
right circular cone by a plane not passing through a vertex, 
where the cone extends 
infinitely far in \emph{both} directions. The section is a called an 
\emph{ellipse}, \index{ellipse|textbf}
a \emph{parabola}, \index{parabola|textbf}
or a \emph{hyperbola}, \index{hyperbola|textbf}
depending on 
whether the angle between the plane and the axis of the cone is 
greater than, equal to, or less than $\pi/4$. 
\begin{figure}[ht]
\caption{Conic sections.}
\end{figure}

\begin{theorem} \label{thm:conic-dist}
An ellipse is the locus of points whose sum of distances to two fixed 
points is constant. Similarly, a hyperbola is the locus of points 
whose (absolute) difference of distances to two fixed points is 
constant.
\end{theorem}
\begin{proof}
This was already known to Apollonius, \index{Apollonius!of Perga}
but the following clever proof was
found by Germinal Dandelin (1794-1847). \index{Dandelin, Germinal}
We will describe only the case of the 
ellipse, as the hyperbola case is similar.

Inscribe spheres in the cone on either side of the plane of the 
ellipse, one on the side of the vertex of the cone, tangent to the plane 
at $A$, the other tangent to the plane at $B$. 
\begin{figure}[ht]
\caption{Dandelin's proof of Theorem~\ref{thm:conic-dist}.}
\end{figure}
For any point 
on the cone between the two spheres, the sum of the lengths of the 
tangents to the two spheres is clearly a constant. On the other hand, 
for any point on the cone also lying in the plane, the segments to $A$ 
and $B$ are also tangent to the respective spheres, so the sum of 
their lengths equals this constant. The result follows.
\end{proof}

The two points alluded to in the above theorem are called \emph{foci} 
(plural of \emph{focus}). \index{focus (of a conic section)|textbf}
The name comes from the fact
that if one has an ellipse made of a reflective 
material and one places a light source at one focus, all of the light 
rays will be ``focused'' at the opposite focus (see Problem~2).

For parabolas, one has the following alternate version of
Theorem~\ref{thm:conic-dist}.
\begin{fact}
A parabola is the locus of points whose distance to a fixed point is
equal to the distance to a fixed line.
\end{fact}
The fixed line and point are called the \emph{focus} and
\emph{directrix}, \index{directrix!of a parabola|textbf} respectively,
of the parabola.

In modern times, it was noted that conic sections have a nice 
description in terms of Cartesian coordinates. If $z^{2} = 
x^{2}+y^{2}$ is the equation of the cone, it is evident that any 
cross-section is defined by setting some quadratic polynomial in $x$ 
and $y$ to 0. Hence a conic section can alternatively be defined as the zero locus of a 
quadratic polynomial; one must impose mild extra conditions to avoid 
degenerate cases, such as a pair of lines (which geometrically arise 
from planes through the vertex of the cone). Unless we say otherwise, 
our conic sections will be required to be nondegenerate.

Here are some standard equations for the conic sections:
\begin{center}
\begin{tabular}{l|l}
Type & Standard equation \\
\hline
Ellipse & $x^2/a^2 + y^2/b^2 = 1$ \\
Parabola & $y=ax^2+bx+c$ \\
Hyperbola & $x^2/a^2 - y^2/b^2=1$ 
\end{tabular}
\end{center}
Also, the equation $xy=1$ defines a \emph{rectangular} hyperbola,
\index{rectangular hyperbola|textbf} \index{hyperbola!rectangular|textbf}
one with perpendicular asymptotes. (The \emph{asymptotes} of a hyperbola
are its tangent lines at its intersections with the line at infinity.)
\index{asymptote!of a hyperbola|textbf}

\begin{exer}
\ii \label{ex:ellipse-tangent}
Prove that a line tangent to an ellipse makes equal 
(undirected) angles with the
segments from the two foci to the point of tangency. What are
the analogous properties of a tangent to a parabola or hyperbola?
\begin{figure}[ht]
\caption{A line tangent to an ellipse (Problem~\ref{ex:ellipse-tangent}).}
\end{figure}
\ii
Prove that two hyperbola branches which share a focus can meet in at most
two points (whereas two hyperbolas can meet in four points).
\ii (Anning-Erd\H{o}s) \index{Erd\H{o}s, P\'al (Paul)}
An infinite set of points in the plane has the 
property that the distance between any two of the points is an 
integer. Prove that the points are all collinear.
\ii
Let $P$ and $Q$ be two points on an ellipse. Prove that there exist 
ellipses similar to the given one,
externally tangent to each other, and internally tangent to 
the given ellipse at $P$ and $Q$, respectively, if and only if $P$ 
and $Q$ are antipodes.
\ii
Use the previous problem to prove
that the maximum distance between two points on an ellipse is
the length of the major axis \emph{without} doing any 
calculations.
\ii (Original)
Prove that the convex quadrilateral $ABCD$ contains a point $P$ such that 
the incircles of triangles $\triangle PAB$ and $\triangle PBC$ are tangent,
as are those 
of $\triangle PBC$ and $\triangle PCD$, of $\triangle PCD$ 
and $\triangle PDA$, and of $\triangle PDA$ and $\triangle PAB$, if 
and only if $ABCD$ has an inscribed circle.
\ii
Find all points on the conic $x^{2}+y^{2}=1$ with \emph{rational} 
coordinates $x,y$ as follows: pick a point $(x,y)$ with rational 
coordinates, and project the conic from $(x,y)$ onto a fixed line 
(e.g.\ the line at infinity). More generally, given a single rational 
point on a conic whose defining equation has rational coefficients, 
this procedure allows you to describe all such points.
\end{exer}

\section{Conics in the projective plane}
In this section, we discuss conic sections from the point of view of 
projective geometry. To start, we rephrase the geometric definition
of a conic section.
\begin{fact}
A curve is a conic section if and only if it is the image of a circle under a 
suitable projective transformation.
\end{fact}
In particular, the theorems of Pascal and Brianchon continue to hold 
if the circle in the statement of either theorem is replaced with an 
arbitrary conic. From these one can deduce converse theorems, that a 
hexagon is inscribed in (resp. circumscribed about) a conic if and 
only if it satisfies the conclusion of Pascal (resp. Brianchon); 
thinking of Pappus's theorem, one realizes that the conics in the 
previous statement must be permitted to be degenerate.

We also note that the classification of conics can be restated in 
terms of projective geometry.
\begin{fact}
A conic is an ellipse (or a circle) if and only if it does not meet 
the line at infinity. A conic is a parabola if and only if it is 
tangent to the line at infinity. A conic is a hyperbola if and only if 
it intersects the line at infinity in two distinct points.
\end{fact}

\begin{exer}
\ii \label{ex:pascconv}
Prove that a hexagon whose opposite sides meet in collinear points is 
inscribed in a conic (which may degenerate to a pair of lines).
\ii
Let $\triangle ABC$ and $\triangle BCD$ be equilateral triangles. An arbitrary line
through $D$ meets $\line{AB}$ at $M$ and $\line{AC}$ at $N$. Determine the acute
angle between the lines $\line{BN}$ and $\line{CM}$.
\ii (Poncelet-Brianchon theorem) \label{ex:poncelet}
\index{Poncelet-Brianchon theorem|textbf}
Let $A,B,C$ be three points on a rectangular hyperbola (a hyperbola 
with perpendicular asymptotes). Prove that the orthocenter of the 
triangle $\triangle ABC$ also lies on the hyperbola.
There are other 
special points of $\triangle ABC$ which must lie on this hyperbola; can you find 
any?
\ii (\textit{Monthly}, Oct.\ 1994)
Let $A_{1}, A_{2}, A_{3}, A_{4}, A_{5}, A_{6}$ be a hexagon 
circumscribed about a conic, and form the intersections $P_{i} = 
\line{A_{i}A_{i+2}} \cap \line{A_{i+1}A_{i+3}}$ 
($i = 1, \dots, 6$, all indices 
modulo 6). Show that the $P_{i}$ are the vertices of a hexagon 
inscribed in a conic.
\ii (\emph{Arbelos})
Let $A, B, C$ be three noncollinear points. Draw ellipses $E_{1}, 
E_{2}, E_{3}$ with foci $B$ and $C$, $C$ and $A$, $A$ and $B$, 
respectively. Show that:
\begin{enumerate}
\ii
Each pair of ellipses meet in exactly two points, where a point of 
tangency counts twice. (In general, two ellipses can meet in as many 
as four points.)
\ii
The three lines determined by these pairs of points are concurrent.
\end{enumerate}
\end{exer}

\section{The polar map and duality} \label{sec:polar}

Fix a circle $\omega$ with center $O$.
The \emph{polar map} \index{polar map|textbf} (or \emph{polar
transformation}) \index{transformation!polar|textbf}
with respect to $\omega$ 
interchanges points and lines in the following manner:
\begin{enumerate}
\ii
If $P$ is a 
finite point other than $O$, the \emph{pole} of $P$ \index{pole
(of a point)|textbf}
is the line $p$ through $P'$ perpendicular to $\line{PP'}$, where $P'$ is the
inverse of $P$ through $\omega$.
\ii
If $p$ is a finite line not passing through $O$, 
the \emph{polar} \index{polar!of a line|textbf}
of $p$ is the inverse through $\omega$ of the foot 
of the perpendicular from $O$ to $p$.
\ii
If $P$ is a point at infinity, the pole of $P$ is the line through $O$ 
perpendicular to any line through $P$, and vice versa.
\ii
If $P$ is $O$, the pole of $P$ is the line at infinity, and vice versa.
\end{enumerate}
The polar map is also known as \emph{reciprocation}. 
\index{reciprocation|textbf} We keep the 
notational convention that points are labeled with capital letters and 
their poles with the corresponding lowercase letters. 
\begin{figure}
\caption{The polar map with respect to a circle.}
\end{figure}

\begin{fact} \label{thm:poleprop}
The polar map satisfies the following properties:
\begin{enumerate}
\ii
Every point is the polar of its pole, and every line is the pole of 
its polar.
\ii
The polar of the line through the points $A$ and $B$ is the 
intersection of the poles $a$ and $b$.
\ii
Three points are collinear if and only if their poles are concurrent.
\end{enumerate}
\end{fact}

An obvious consequence of the existence of the polar map is the 
duality principle. \index{duality principle|textbf}
\begin{fact}[Duality principle]
A theorem of projective geometry remains true if the roles of points 
and lines are interchanged.
\end{fact}
For example, the dual of one direction of Desargues's theorem is the 
other direction.

We can now give Brianchon's original proof of his theorem, 
\index{Brianchon's theorem} using 
Pascal's theorem \index{Pascal's theorem}
and the polar map. There's nothing to it, really: 
given a hexagon circumscribed about a circle $\omega$, apply the polar 
map with respect to $\omega$. The result is a hexagon inscribed in 
$\omega$, and the collinearity of the intersections of opposite sides 
translates back to the original diagram as the concurrence of the 
lines through opposite vertices.

\begin{exer}
\ii
Make up a problem by starting with a result that you know and 
applying the polar map. Beware that circles not concentric with $\omega$ 
do not behave well under the polar map; see below.
\ii
State the dual of Pappus's theorem. \index{Pappus's theorem}
Can you prove this directly? (A 
projection may help.)
\ii
State and prove a dual version of problem~\ref{ex:circquads}. Since circles do 
not dualize to circles, you will have to come up with a new proof!
\ii (China, 1996)
Let $H$ be the orthocenter of acute triangle $\triangle ABC$. The tangents 
from $A$ to the circle with diameter $\seg{BC}$ touch the circle at $P$ 
and $Q$. Prove that $P, Q, H$ are collinear.
\ii
Let $\triangle 
ABC$ be a triangle with incenter $I$. Fix a line $\ell$ tangent 
to the incircle of $\triangle ABC$ (not containing any of the sides).
Let $A', B', C'$ be points on $\ell$ such that
\[
\angle AIA' = \angle BIB' = \angle CIC' = \pi/2.
\]
Show that $\line{AA'}, \line{BB}', \line{CC'}$ are concurrent.
\ii \label{ex:imo852} (R\u{a}zvan Gelca) \index{Gelca, R\u{a}zvan}
Let $A, B, C, D$ be four points on a circle. Show that the pole of $\line{AC}
\cap \line{BD}$ with respect to this circle passes through $\line{AB} \cap 
\line{CD}$ and
$\line{AD} \cap \line{BC}$. Use this fact to give another solution to 
Problem~\ref{ex:imo85} (IMO
1985/6).

\ii
We know what happens to points and lines under the polar map, but 
what about a curve? If we view the curve as a \emph{locus},
\index{locus|textbf} i.e.\ a 
set of points, its dual is a set of lines which form an \emph{envelope}, 
\index{envelope|textbf}
i.e.\ they are all tangent to some curve. 
\begin{figure}[ht]
\caption{The envelope of a family of lines.}
\end{figure}
Show that the dual of a conic, under this definition, is again a 
conic. However, the dual of a circle need not be a circle.
\ii
Let $\omega$ be a (nondegenerate) conic.  Show that there exists a 
unique map on the projective plane, taking points to lines and vice 
versa, satisfying the properties in Fact~\ref{thm:poleprop}, and 
taking each point on $\omega$ to the tangent to $\omega$ through that 
point. This map is known as the \emph{polar map with respect to 
$\omega$} \index{polar map!with respect to a conic|textbf}
(and coincides with the first definition if $\omega$ is a 
circle).
\ii (IMO 1998/5)
Let $I$ be the incenter of triangle $\triangle ABC$.  Let the incircle of 
$\triangle ABC$
touch the sides $\seg{BC}, \seg{CA}, \seg{AB}$ at $K, L, M$,
respectively.  The line through $B$ parallel to $\line{MK}$ meets the lines
$\line{LM}$ and $\line{LK}$ at $R$ and $S$, respectively.  
Prove that angle $\angle RIS$ is acute.
\end{exer}

\section{Cross-ratio}
\label{sec:cross-ratio}

From the discussion so far, it may appear that there is no useful 
notion of distance in projective geometry, for projective 
transformations do not preserve Euclidean distances, or even ratios of 
distances along a line (which affine transformations do preserve).
There is something to be salvaged here, though; the ``ratios of 
ratios of distances'' are preserved.

Given four collinear points $A,B,C,D$, the \emph{cross-ratio}
 \index{cross-ratio!of four collinear points|textbf} of these 
points is defined as the following signed ratio of lengths:
\index{ratio of lengths, signed (of collinear segments)|textbf} 
\index{signed ratio of lengths (of collinear segments)|textbf}
\[
\frac{AC \cdot BD}{AD \cdot BC}.
\]
In case one of these points is at infinity, the definition can be 
extended by declaring that the ratio of two infinite distances is 1. 
We have left the definition where all of the points lie at infinity as an 
exercise.

In light of duality, we ought to be able to make this definition for 
four concurrent lines, and in fact we can: the cross-ratio of four 
lines $a,b,c,d$ is defined as the cross-ratio of the intersections 
$A,B,C,D$ of $a,b,c,d$ with some line $\ell$ not passing through the point 
of concurrency. \index{cross-ratio!of four lines|textbf}
The cross-ratio is well-defined by the following observation,
which follows from several applications of the Law of Sines.
\begin{fact}
Let $a, b, c, d$ be four concurrent lines and $\ell$ a line meeting
$a, b, c, d$ at $A, B, C, D$, respectively. Then 
\[
\frac{AC \cdot BD}{AD \cdot BC} = \frac{\sin \dang(a, c) \sin \dang(b,
d)}{\sin \dang(b, c) \sin \dang(a, d)}.
\]
\end{fact}

\begin{fact}
The cross-ratio is invariant under projective transformations and the 
polar map.
\end{fact}

In case the cross-ratio is $-1$, we say $C$ and $D$ are \emph{harmonic 
conjugates} \index{harmonic conjugates|textbf}
with respect to $A$ and $B$ (or vice versa). If you did 
Problem~\ref{ex:harmcon},
you witnessed the most interesting property of harmonic 
conjugates: if $P$ is any point not on the line and $Q$ is any point 
on $\line{PC}$
other than $P$ or $C$, then $\line{AP} \cap \line{BQ}$, $\line{AQ} \cap 
\line{BP}$ and $D$ are
collinear. (Not surprisingly, this property is projection-invariant.)

One nice application of cross-ratios is the following characterization 
of conics.
\begin{fact} \label{fact:conic cross-ratio}
Given four points $A,B,C,D$, the locus of points $E$ such that the 
cross-ratio of the lines $\line{AE}, \line{BE}, \line{CE}, \line{DE}$ 
is constant is a conic.
\end{fact}

\begin{exer}
\ii
How should the cross-ratio be defined along the line at infinity?
\ii
Let $A,B,C,D$ be four points on a circle. Show that for $E$ on the 
circle, the cross-ratio of the lines $\line{EA}, \line{EB}, \line{EC}, 
\line{ED}$ remains constant. Then 
use this to deduce Fact~\ref{fact:conic cross-ratio}.
\ii (``Butterfly problem'')
Let $M$ be the midpoint of chord $\seg{XY}$ of a circle, and
let $\seg{AB}$ and $\seg{CD}$ be chords passing through $M$. Let $E = 
\line{AD} \cap \line{XY}$ 
and $F = \line{BC} \cap \line{XY}$. Prove that $EM=MF$.
\ii
The points $A,B,C,D$, in this order, lie on a
straight line.  A circle $k$ passes through $B$ and $C$, and 
the lines $\line{AM}, \line{AN}, \line{DK}, \line{DL}$ 
are tangent to $k$ at $M,N,K,L$.
The lines $\line{MN}, \line{KL}$ intersect $\line{BC}$ at $P, Q$.
\begin{enumerate}
\item[(a)] Prove that $P$ and $Q$ do not depend on $k$.
\item[(b)] If $AD=a$, $BC=b$, and the segment $\seg{BC}$ moves along
$\seg{AD}$, find the minimum length of segment $\seg{PQ}$.
\end{enumerate}
\end{exer}

\section{The complex projective plane}
\label{sec:alggeo}

The homogeneous coordinates we have worked with so far also make 
sense for complex numbers, though visualizing the result is 
substantially harder. The set of points they define (i.e.\ the set 
of proportionality classes of ordered triples of complex numbers, not all 
zero) is called the \emph{complex projective plane}. 
\index{complex projective plane|textbf}
\index{projective plane!complex|textbf}
We define lines 
and conics in this new plane simply as the zero loci of linear and 
quadratic polynomials, respectively.

One handy feature of the complex projective plane is the
following characterization of circles.
\begin{fact}
A nondegenerate conic is a circle if and only if it passes
through the points $[1:i:0]$ and $[1:-i:0]$. 
\end{fact}
These two points are called the \emph{circular points at infinity}, 
or simply the \emph{circular points} for short.
\index{circular point at infinity|textbf}
\index{point at infinity!circular|textbf}

The fact that complex circles always meet the line at infinity in two 
points, while real circles to not, is a symptom of the key fact that
the complex numbers are \emph{algebraically closed}, 
\index{algebraically closed (field)|textbf} i.e.\ every 
polynomial with complex coefficients has a complex root. (This is the 
Fundamental Theorem of Algebra, first proved by Gauss.) 
\index{Fundamental Theorem of Algebra}
This means, 
for example, that we have the following:
\begin{fact}
In the complex projective plane,
two conics meet in exactly four points (counting points of tangency 
twice).
\end{fact}
In fact, a more general result is true, which we will not prove; it is
attributed to Etienne B\'ezout\footnote{There seems to be some disagreement
over whether this name is spelled ``Bezout'' or ``B\'ezout''; we use the MacTutor
spelling.}
(1730--1783). \index{B\'ezout, Etienne}
\index{B\'ezout's theorem|textbf}
\begin{theorem}[B\'ezout]
The zero loci 
of two polynomials, of degrees $m$ and $n$, contains exactly $mn$ 
points if the loci meet transversally everywhere (i.e.\ at each 
intersection, each locus has a well-defined tangent line, and the tangent 
lines are distinct).
\end{theorem}
If the loci do not meet transversally, e.g.\ if they are tangent 
somewhere, one must correctly assign multiplicities to the 
intersections to make the count work.

An interesting consequence of Bezout's theorem,
which we will prove independently, is 
due to Michel Chasles\footnote{``Chasles'' is pronounced ``shell''.}
(1793--1880). \index{Chasles, Michel} The zero locus of a polynomial 
of degree 3 is known as a \emph{cubic curve}. \index{cubic curve|textbf}
\index{Chasles's theorem|textbf}
\begin{theorem}[Chasles]
Let $C_{1}$ and $C_{2}$ be two cubic curves meeting in exactly nine 
distinct points. Then any cubic curve passing through eight of the 
points passes through the ninth point.
\end{theorem}
\begin{proof}
The set of homogeneous degree 3 polynomials in $x,y,z$ is a 
10-dimensional vector space (check by writing a basis of monomials); 
let $Q_1$ and $Q_2$ be polynomials with zero 
loci $C_1$ and $C_2$, respectively, and let $P_1, \dots, P_9$ be the 
nine intersections of $C_1$ and $C_2$. Note that no four of these 
points lie on a line and no seven lie on a conic,
or else each of $C_1$ and $C_2$ would have this 
line or conic as a component, and their intersection would be infinite 
rather than nine points.

Let $d_i$ be the dimension of the space of degree 3 polynomials vanishing 
at $D_1, \dots, D_i$ (and put $d_0 = 10$); then for $i \leq 8$, $d_i$ 
equals either $d_{i-1}-1$ or $d_i$, the latter only if every cubic 
curve
passing through $P_1, \dots, P_{i-1}$ also passes through $P_i$. 
However, this turns out not to be the case; see the problems.
Thus $d_8 = 2$, and we already have two linearly independent 
polynomials in this space, namely $Q_1$ and $Q_2$. (If they were 
dependent, they would define the same curve, and again the 
intersection would be infinite.) Thus if $C$ is a cubic curve defined 
by a polynomial $Q$ that passes through $P_1, \dots, P_8$, then $Q = 
aQ_1 + bQ_2$ for some $a,b \in \CC$, and so $Q$ also vanishes at 
$P_9$, as desired.
\end{proof}

These results are just the tip of a rather 
sizable iceberg. The modern subject of \emph{algebraic geometry} is 
concerned with the study of zero loci of sets of polynomials in 
spaces of any dimension. It interacts with almost every other branch 
of mathematics, including complex analysis, topology, number theory, 
combinatorics, and mathematical physics. Unfortunately, the subject 
as practiced today has become technically involved\footnote{Algebraic
geometry flourished in Italy in the early 20th century, but it was practiced
with a flagrant lack of rigor that led to numerous errors. To fix these,
it proved necessary to recast the foundations of the topic; this was
accomplished in the 1960s
under the guidance of Alexander Grothendieck (1928--??).
\index{Grothendieck, Alexander}
What
the new foundations gain in power and flexibility, they lack in accessibility;
the most accessible route to them seems to be via \cite{bib:eh}.};
the novice should 
start with a book written in the ``classical'' style, such as 
Harris \cite{bib:har} or Shafarevich \cite{bib:sha}, before proceeding 
to a ``modern'' text such as Eisenbud and Harris
\cite{bib:eh} or Hartshorne \cite{bib:hart}.
(If it is not already clear from the 
rhapsodic tone of this section,
algebraic geometry, particularly in connection with number theory, 
ranks among the author's main research interests.)

\begin{exer}
\ii
Give another proof that there is a unique conic passing through any 
five points, using the circular points.
\ii
Make up a problem by taking a projective statement you know and 
projecting two of the points in the diagram to the circular points. 
(One of my favorites is the radical axis theorem---which becomes a 
projective statement if you replace the circles by conics through two 
fixed points!)
\ii 
Deduce Pascal's theorem from Chasles' theorem applied to a certain
degenerate cubic.
\ii \label{ex:chas}
Prove that given eight or fewer points in the plane, no four on a line and no 
seven on a conic, one of which is labeled $P$,
there exists a cubic curve passing through all of the points but $P$.
\ii
A cubic curve which is nondegenerate, and additionally has no 
\emph{singular point} (a point where the partial derivatives of the 
defining homogeneous polynomial all vanish, like the point $[0:0:1]$ 
on the curve $y^2z = x^3+x^2z$) is called an \emph{elliptic 
curve}.\footnote{The geometry of elliptic curves pervades much of
modern number theory, e.g., the proof of Fermat's Last Theorem 
\index{Fermat's Last Theorem}
given in 1995 by Andrew Wiles (1953--). \index{Wiles, Andrew}
See \cite{bib:st} for a
gentle introduction, or \cite{bib:silverman} for a more comprehensive
treatment.} \index{elliptic curve|textbf}
Let $E$ be an elliptic curve, and pick a point $O$ on $E$. 
Define ``addition'' of points on $E$ as follows: 
\index{addition!on an elliptic curve|textbf}
given points $P$ and 
$Q$, let $R$ be the third intersection of the line $\line{PQ}$ with $E$, and 
let $P+Q$ be the third intersection of the line $\line{OR}$ with $E$. Prove 
that $(P+Q)+R = P+(Q+R)$ for any three points $P,Q,R$, i.e.\ that 
``addition is associative''. (If you know what a group is, show that 
$E$ forms a group under addition, by showing that there exist
inverses and an identity element.)
For more on elliptic curves, and their role 
in number theory, see \cite{bib:st}.
\ii \label{ex:rom97a}
Give another solution to problem~\ref{ex:rom97} using a well-chosen
projective transformation in the complex projective plane.
\ii
One can define addition on a curve on a singular cubic in the same
fashion, as long as none of the points involved is a singular point of
the cubic. Use this fact to
give another solution to Problem~\ref{ex:rom97}.
\ii \label{ex:flex}
Let $E$ be an elliptic curve. Show that there are exactly 
nine points at which the tangent line at $E$ has a triple, not just a 
double, intersection with the curve (and so meets the curve nowhere 
else). These points are called \emph{flexes}. \index{flex 
(of an elliptic curve)|textbf} 
Also show that the line 
through any two flexes meets $E$ again at another flex. (Hence the 
flexes constitute a counterexample to Problem~\ref{ex:syl} in the 
complex projective plane!)
\ii (Poncelet's porism) \label{ex:projstein} \index{Poncelet's porism|textbf}
\index{porism!Poncelet's|textbf}
Let $\omega_1$ and $\omega_2$ be two conic sections. Given a point $P_0$
on $\omega_1$, let $P_1$ be either of the points on $\omega_1$ such that
the line $\line{P_0P_1}$ is tangent to $\omega_2$. Then for $n \geq 2$,
define $P_n$ as the point on $\omega_1$ other than $P_{n-2}$ such that
$\line{P_{n-1}P_n}$ is tangent to $\omega_2$. Suppose there exists $n$ such
that $P_0=P_n$ for a particular choice of $P_0$. Show that
$P_0=P_n$ for any choice of $P_0$.
\end{exer}

\part{Odds and ends}
\label{part:solutions}

%% Appendix: Hints?
\setcounter{secnumdepth}{-1}
\chapter{Hints} \label{hints}

Here are the author's suggestions on how to proceed on some of the problems.
If you find another solution to a problem, so much the better---but it may not
be a bad idea to try to find the suggested solution anyway!

We have refrained from including detailed
solutions to all of the problems; for the justification of this decision,
and for a web location at which solutions can be found, see the Introduction.

\begin{itemize}
\item[\ref{ex:appower}]
Imitate the proof of Theorem~\ref{thm:ap}.
\item[\ref{ex:threeeq}]
Consider the triangle $\triangle
AB_1C_1$ together with 
the second intersection of the circumcircles of $\triangle
AB_1C_2$ and $\triangle AB_3C_1$.
Show that this figure is congruent to the two analogous figures 
formed from the other triangles. Do this by rotating
$\triangle AB_1C_1$ onto $\triangle
C_2AB_2$ onto $\triangle B_3C_3A$ and tracing what happens to
the figure. (Or apply Theorem~\ref{thm:back1}.)
\item[\ref{ex:rus03homot}]
Consider the homothety around $D$ taking $B$ to $C$. If you knew the
problem were true, what would that say about the image of $E$? Once you
figure that out, work backwards. (It may help to peek ahead to
Chapter~\ref{chap:triangle}.)
\item[\ref{ex:mop98spiral}]
After applying Theorem~\ref{thm:back1}, this should bear a strong
resemblance to Problem~\ref{ex:threeeq}.
\item[\ref{ex:sevenpt}]
How does $P_2$ depend on $P_1$?
\item[\ref{ex:tetocta}] The octahedron has 4 times the volume of the
tetrahedron. What happens when you glue them together at a face?
\item[\ref{ex:imo94}]
Prove one assertion, then work backward to prove the other.
\item[\ref{ex:morley}] Construct two of the intersections of the trisections,
complete the equilateral triangle, then show that its third vertex is
the third intersection. This is difficult;
if you're still stuck, see \cite{bib:cg}.
\item[\ref{ex:des2}] Draw 10 points: the 6 vertices of the triangles, the
three intersections of corresponding sides, and the intersection of the
lines joining two pairs of corresponding vertices. If you relabel these
10 points appropriately, this diagram will turn into a case of the forward
direction of Desargues!
\item[\ref{ex:radaxconv}]
Draw the circumcircle of $ABC$, and apply the radical axis theorem to
that circle, $\omega_1$, and $\omega_2$.
\item[\ref{ex:imo85}]
There are several solutions to this problem, but no one of them 
is easy to find.
In any case, before anything else, find an extra cyclic quadrilateral.
\item[\ref{ex:pol1997}]
Work backwards,
defining $G$ as the point for which the conclusion holds. Also 
consider the circumcircle of $CDE$.
\item[\ref{ex:simcon}]
Find a cyclic hexagon.
\item[\ref{ex:postsim}]
Use Theorem~\ref{thm:presim}.
\item[\ref{ex:convex}]
Even using directed angles, the 
result fails for nonconvex hexagons. Figuring out why may help
you determine how to use convexity here.
\item[\ref{ex:simsynth}]
Given segments $AB$ and $CD$, what conditions must the center 
$P$ of a spiral similarity carrying $AB$ to $CD$ satisfy?
\item[\ref{ex:apcon}]
By Ceva and Menelaus, one can show $BA_1/A_1C = BA_2/A_2C$. This means 
the circle with diameter $A_1A_2$ is a circle of Apollonius with 
respect to $B$ and $C$.
\item[\ref{ex:usamo91}]
The center of the circle lies at $C$.
\item[\ref{ex:iran97}]
The fixed point lies on the circumcircle of $ABC$.
\item[\ref{ex:usamo1999}]
Show that the point $F$ is the excenter of $ACD$ opposite $A$.
\item[\ref{ex:imo1992}]
Use homothety.
\item[\ref{ex:bul1996}]
The incircle of triangle $O_1O_2O_3$ touches $O_2O_3$ at $C$. Reformulate
the problem in terms of $O_1O_2O_3$ and get rid of the circles. From there,
one way to proceed is to calculate where along $\ell$ the intersection
with $AO_1$ is.
\item[\ref{ex:ninept}]
For (a), write the half-turn as the 
composition of two other homotheties and locate the fixed point.
\item[\ref{ex:tangents symm}]
Use circles of Apollonius.
\item[\ref{ex:broc1}]
What is the locus of points where one of these equalities 
holds?
\item[\ref{ex:rus2003frame}]
The frame shift here is to consider the triangle formed by the excenters.
\item[\ref{ex:cirort}] The distance $d$ satisfies
$9 d^2 = a^2 + b^2 + c^2$.
\item[\ref{ex:stewart}]
Apply the Law of Cosines to the triangles $ABD$ and $ACD$.
\item[\ref{ex:inrect}]
Use
Fact~\ref{fact:midarc}.
\item[\ref{ex:comp quad coaxial}]
Show that the orthopole is the radical axis of any two of the circles.
\item[\ref{ex:circquads}]
Show that no two consecutive quadrilaterals can both have 
incircles.
\item[\ref{ex:usamo992}]
Use the similar triangles formed by the sides and diagonals.
\item[\ref{ex:cotident}]
Write everything in terms of $\cot A/2$ and the like. Then turn the 
result into a statement about homogeneous polynomials using the 
identity
\[
\cot \frac{A}{2} + \cot \frac B2 + \cot \frac C2
=
\cot \frac{A}{2} \cot \frac B2 \cot \frac C2,
\]
and solve the result.
\item[\ref{ex:quadineq}]
Use an affine transformation to make $ABCD$ cyclic, and 
perform a quadrilateral analogue of the $s-a$ substitution.
\item[\ref{ex:imo965}]
A certain special case of this result is equivalent to 
Erd\H{o}s-Mordell. Modify the proof slightly to accommodate the 
generalization.
\item[\ref{ex:conc}]
Which circles are orthogonal to two concentric circles?
\item[\ref{ex:apoll}]
Reduce to the case where two of the circles are tangent, then invert.
\item[\ref{ex:iran95}]
The paradigm does not hold here.
Invert through the incircle, then superimpose the 
original and inverted diagrams.
\item[\ref{ex:mop97}]
Note that $AB \cdot AB_1 = AC \cdot AC_1$. Also look at the
intersection of $OA$ and $B_1 C_1$.
\item[\ref{ex:rus93invert}]
The busy point is $O$. After you invert there, the conclusion is that
$K', P', Q'$ are collinear, and the hypothesis on $P$ and $Q$ should look
like a criterion for collinearity.
\item[\ref{ex:cencirc}]
Find a projective
transformation taking the circle to itself but not preserving its center.
\item[\ref{ex:pascconv}]
Fix five of the points and compare the locus of sixth points making 
this condition hold with the conic through the five points.
\item[\ref{ex:poncelet}]
Apply Pascal's 
theorem to the hyperbola, using the intersections of the asymptotes 
with the line at infinity as two of the six points.
\item[\ref{ex:imo852}]
Draw the circle with diameter $OB$, and show that its
common chord with the circle centered at $O$ is concurrent with $KN$
and $AC$.
\item[\ref{ex:chas}]
In fact, there exists a degenerate cubic with this property.
\item[\ref{ex:rom97a}]
Find a projective transformation taking the circle to a circle and
the line to infinity.
\item[\ref{ex:projstein}]
As in Steiner's porism, reduce to the case of two concentric circles.
\end{itemize}

\setcounter{secnumdepth}{-1}
\chapter{Suggested further reading}

The definition of ``reading'' here is expansive: it includes electronic
resources such as software packages (for dynamic geometry) and
Web resources (for competitions).

\subsection*{Algebraic geometry}

As noted in Section~\ref{sec:alggeo}, one should start
with a text written in ``traditional'' language, such as those by
Harris \cite{bib:har}, Shafarevich \cite{bib:sha}, or Cox, Little
and O'Shea \cite{bib:clo}.

\subsection*{Competitions}
The Art of Problem Solving web site,
\begin{center}
\texttt{http://www.artofproblemsolving.com/}
\end{center}
is the premier web resource for
students interested in problem solving of the sort appearing in competitions
like the USAMO and IMO.

%\subsection*{Computational geometry}
%\fixme{computational geometry?!}

\subsection*{Dynamic geometry}
The phrase ``dynamic geometry'' refers to computer software that can
render a geometric configuration in a fashion that allows the user to
vary the determining data and witness the change in the resulting 
configuration in real time. (For example, if it appears that three lines
are concurrent, one can test this hypothesis by ``jiggling the input
data'' to see whether the concurrence appears to be coincidental or 
causal.)
There are several outstanding programs for doing this: commercial offerings
include \emph{Cabri}, \emph{Cinderella}, and \emph{The Geometer's Sketchpad},
while slimmer noncommercial alternatives include \textit{Kgeo} and
\textit{Kseg}.

\subsection*{Geometric inequalities}
The compilation \cite{bib:bott} is the definitive
source, while its sequel \cite{bib:rcige} details more recent results.

\subsection*{Hyperbolic geometry}
The book \cite{bib:sve} is a charming introduction
to the topic, spinning a tale of Lewis Carroll, his friend and muse Alice
Liddell, and a mysterious stranger as they explore unfamiliar
geometric territory.

\subsection*{Miscellaneous}
The book \cite{bib:baragar} is a nice survey
of ``modern'' geometry in various forms: it includes sections on hyperbolic
geometry, spherical geometry, projective geometry, and constructibility.


\begin{thebibliography}{99}
\addcontentsline{toc}{chapter}{Bibliography}

\bibitem{bib:baragar}
A. Baragar, \textit{A Survey of Classical and Modern Geometries
(with Computer Activities)}, Prentice-Hall, New Jersey, 2001.

\bibitem{bib:bott}
O. Bottema, R. \v{Z}. Djordjevi\'c, R. R. Jani\'c, D. S. Mitrinovi\'c,
and P.M. Vasi\'c, \textit{Geometric Inequalities}, Wolters-Noordhoff,
Groningen, 1968.

%\bibitem{bib:chazelle}
%B. Chazelle, Triangulating a simple polygon in linear time,
%\textit{Discrete and Computational Geometry}
%\textbf{6}, 485--524.

\bibitem{bib:clo}
D. Cox, J. Little, and D. O'Shea,
\textit{Ideals, Varieties and Algorithms: An Introduction to Computational
Algebraic Geometry and Commutative Algebra}, second edition,
Undergraduate Texts in Mathematics, Springer-Verlag, New York, 1997.

\bibitem{bib:coxeter}
H.S.M. Coxeter, \textit{Non-Euclidean Geometry},
sixth edition, Mathematical Association of America
(MAA Spectrum), Washington, 1998.

\bibitem{bib:cg}
H.S.M. Coxeter and S.L. Greitzer, \textit{Geometry Revisited},
Mathematical Association of America (NML 19), Washington, 1967.

\bibitem{bib:eh}
D. Eisenbud and J. Harris, \textit{The Geometry of Schemes},
Springer-Verlag (Graduate Texts in Mathematics 197), New York, 2000.

\bibitem{bib:eve}
H. Eves, \textit{A Course in Geometry} (2 volumes).

\bibitem{bib:greitzer}
S. Greitzer, \textit{Arbelos}, volumes 1--5, American Mathematics
Competitions, 1982--1988.

\bibitem{bib:gp}
V. Guillemin and A. Pollock, \textit{Differential Topology}, Prentice-Hall,
Englewood Cliffs, 1974.

\bibitem{bib:har}
J. Harris, \textit{Algebraic Geometry: A First Course}, 
Springer-Verlag (Graduate Texts in Mathematics 133), New York, 1992.

\bibitem{bib:hart}
R. Hartshorne, \emph{Algebraic Geometry}, Springer-Verlag (Graduate Texts
in Mathematics 52), New York, 1977.

\bibitem{bib:putnam}
K.S. Kedlaya, B. Poonen, and R. Vakil, \textit{The William Lowell
Putnam Competition 1985--2000: Problems, Solutions and Commentary},
Problem Books series, MAA (Washington), 2002.

\bibitem{bib:kimberling}
C. Kimberling, \textit{Triangle Centers and Central Triangles},
Utilitas Mathematica Publishing (Congressus Numerantium 129),
1998.

\bibitem{bib:max}
E. A. Maxwell, \textit{Fallacies in Mathematics},
Cambridge University Press, 1959.

\bibitem{bib:rcige}
D.S. Mitrinovi\'c et al, \textit{Recent Advances in Geometric Inequalities},
Kluwer Academic Publishers, Boston, 1989.

\bibitem{bib:sha}
I.R. Shafarevich, \textit{Basic Algebraic Geometry} (translated
by Miles Reid), Springer-Verlag, New York, 1994.

\bibitem{bib:silverman}
J.H. Silverman, \textit{The Arithmetic of Elliptic Curves},
Graduate Texts in Mathematics \textbf{106}, Springer-Verlag (New York),
1986.

\bibitem{bib:st}
J.H. Silverman and J. Tate, \textit{Rational Points on Elliptic Curves}, Springer-Verlag (Undergraduate Texts in Mathematics), New York, 1992.

\bibitem{bib:sve}
M. Sved, \textit{Journey into Geometries}, MAA Spectrum,
Mathematical Association of America (Washington), 1991.

\bibitem{bib:jvy}
J. van Yzeren, A simple proof of Pascal's hexagon theorem, 
\textit{American Mathematical Monthly} \textbf{100} (1993), 930--931.

\bibitem{bib:yag}
I.M. Yaglom, \emph{Geometric Transformations}, Random House
(New Mathematical Library 8), New York, 1973.

\end{thebibliography}

\chapter{About the license}

This book is being distributed in an ``open source'' fashion, much in
the manner that open source software like the Linux operation system
is distributed. This confers certain rights on you, the reader, but also
carries certain restrictions that limit your ability to restrict the rights
of others. This chapter includes a little information about how the open
source model operates both in general and in this particular instance;
more specifics can be found in 
the text of the GNU Free Documentation License (GFDL),
\index{GNU Free Documentation License}, which appears in the following chapter.

\section{Open source for text?}

This book has been released under the GNU Free Documentation License (GFDL)
\index{GNU Free Documentation License}
in order to promote the free and open dissemination of the ideas contained
herein, and as an experiment in collaborative authorship. 
The GFDL expressly permits unlimited distribution of this document
either intact or in modified forms, through either 
commercial or noncommercial means. 
The main restrictions it imposes are that:
\begin{itemize}
\item
the human-editable source file(s) for any modified form of this document must
be made freely available;
\item
all modified forms must themselves carry the GFDL.
\end{itemize}
Note that the second item means that no material may be incorporated into
a modified form of this book which carries a copyright restriction less
permissive than the GFDL (i.e., most material carrying an outstanding 
non-GFDL copyright), unless the use of that material constitutes a \emph{fair
use} \index{fair use} in the sense of US copyright law.
On the other hand, other GFDL material, material on which the copyright
has expired, and material in the public domain are fair game.

\section{Source code distribution}

Like a computer program, a piece of electronic text is usually used in
a ``processed'' form, such as what you are (probably) reading now,
but is created in a ``source'' form which is easier for the author to
manipulate. In the case of this book, the source code is input for Knuth's
TeX typesetting system, or more precisely for the variant known as AMSLaTeX.
One requirement of the GFDL is that any processed document must either 
include, or direct the reader to an electronic location of, all source files.
In this case, these source files will be distributed at the following URL:

\begin{center}
\texttt{http://math.mit.edu/\~{}kedlaya/geometryunbound}.
\end{center}

This site will also include some material not included in the present document,
such as diagrams and solutions.

\section{History}

This section is mandated by the GFDL to carry the modification/derivation
history of this document. If you prepare a modification of this
document, make sure to update this section!

The present document is the original version
to carry the GFDL, released 18 Jan 2006.

\setcounter{secnumdepth}{-1}
\chapter{GNU Free Documentation License}
\label{label_fdl}

\index{GNU Free Documentation License}

 \begin{center}

       Version 1.2, November 2002


 Copyright \copyright 2000,2001,2002  Free Software Foundation, Inc.
 
 \bigskip
 
     51 Franklin St, Fifth Floor, Boston, MA  02110-1301  USA
  
 \bigskip
 
 Everyone is permitted to copy and distribute verbatim copies
 of this license document, but changing it is not allowed.
\end{center}

\section{Preamble}

The purpose of this License is to make a manual, textbook, or other
functional and useful document ``free" in the sense of freedom: to
assure everyone the effective freedom to copy and redistribute it,
with or without modifying it, either commercially or noncommercially.
Secondarily, this License preserves for the author and publisher a way
to get credit for their work, while not being considered responsible
for modifications made by others.

This License is a kind of ``copyleft", which means that derivative
works of the document must themselves be free in the same sense.  It
complements the GNU General Public License, which is a copyleft
license designed for free software.

We have designed this License in order to use it for manuals for free
software, because free software needs free documentation: a free
program should come with manuals providing the same freedoms that the
software does.  But this License is not limited to software manuals;
it can be used for any textual work, regardless of subject matter or
whether it is published as a printed book.  We recommend this License
principally for works whose purpose is instruction or reference.

\section{1. Applicability and Definitions}

This License applies to any manual or other work, in any medium, that
contains a notice placed by the copyright holder saying it can be
distributed under the terms of this License.  Such a notice grants a
world-wide, royalty-free license, unlimited in duration, to use that
work under the conditions stated herein.  The \textbf{``Document"}, below,
refers to any such manual or work.  Any member of the public is a
licensee, and is addressed as \textbf{``you"}.  You accept the license if you
copy, modify or distribute the work in a way requiring permission
under copyright law.

A \textbf{``Modified Version"} of the Document means any work containing the
Document or a portion of it, either copied verbatim, or with
modifications and/or translated into another language.

A \textbf{``Secondary Section"} is a named appendix or a front-matter section of
the Document that deals exclusively with the relationship of the
publishers or authors of the Document to the Document's overall subject
(or to related matters) and contains nothing that could fall directly
within that overall subject.  (Thus, if the Document is in part a
textbook of mathematics, a Secondary Section may not explain any
mathematics.)  The relationship could be a matter of historical
connection with the subject or with related matters, or of legal,
commercial, philosophical, ethical or political position regarding
them.

The \textbf{``Invariant Sections"} are certain Secondary Sections whose titles
are designated, as being those of Invariant Sections, in the notice
that says that the Document is released under this License.  If a
section does not fit the above definition of Secondary then it is not
allowed to be designated as Invariant.  The Document may contain zero
Invariant Sections.  If the Document does not identify any Invariant
Sections then there are none.

The \textbf{``Cover Texts"} are certain short passages of text that are listed,
as Front-Cover Texts or Back-Cover Texts, in the notice that says that
the Document is released under this License.  A Front-Cover Text may
be at most 5 words, and a Back-Cover Text may be at most 25 words.

A \textbf{``Transparent"} copy of the Document means a machine-readable copy,
represented in a format whose specification is available to the
general public, that is suitable for revising the document
straightforwardly with generic text editors or (for images composed of
pixels) generic paint programs or (for drawings) some widely available
drawing editor, and that is suitable for input to text formatters or
for automatic translation to a variety of formats suitable for input
to text formatters.  A copy made in an otherwise Transparent file
format whose markup, or absence of markup, has been arranged to thwart
or discourage subsequent modification by readers is not Transparent.
An image format is not Transparent if used for any substantial amount
of text.  A copy that is not ``Transparent" is called \textbf{``Opaque"}.

Examples of suitable formats for Transparent copies include plain
ASCII without markup, Texinfo input format, LaTeX input format, SGML
or XML using a publicly available DTD, and standard-conforming simple
HTML, PostScript or PDF designed for human modification.  Examples of
transparent image formats include PNG, XCF and JPG.  Opaque formats
include proprietary formats that can be read and edited only by
proprietary word processors, SGML or XML for which the DTD and/or
processing tools are not generally available, and the
machine-generated HTML, PostScript or PDF produced by some word
processors for output purposes only.

The \textbf{``Title Page"} means, for a printed book, the title page itself,
plus such following pages as are needed to hold, legibly, the material
this License requires to appear in the title page.  For works in
formats which do not have any title page as such, ``Title Page" means
the text near the most prominent appearance of the work's title,
preceding the beginning of the body of the text.

A section \textbf{``Entitled XYZ"} means a named subunit of the Document whose
title either is precisely XYZ or contains XYZ in parentheses following
text that translates XYZ in another language.  (Here XYZ stands for a
specific section name mentioned below, such as \textbf{``Acknowledgements"},
\textbf{``Dedications"}, \textbf{``Endorsements"}, or \textbf{``History"}.)  
To \textbf{``Preserve the Title"}
of such a section when you modify the Document means that it remains a
section ``Entitled XYZ" according to this definition.

The Document may include Warranty Disclaimers next to the notice which
states that this License applies to the Document.  These Warranty
Disclaimers are considered to be included by reference in this
License, but only as regards disclaiming warranties: any other
implication that these Warranty Disclaimers may have is void and has
no effect on the meaning of this License.

\section{2. Verbatim Copying}

You may copy and distribute the Document in any medium, either
commercially or noncommercially, provided that this License, the
copyright notices, and the license notice saying this License applies
to the Document are reproduced in all copies, and that you add no other
conditions whatsoever to those of this License.  You may not use
technical measures to obstruct or control the reading or further
copying of the copies you make or distribute.  However, you may accept
compensation in exchange for copies.  If you distribute a large enough
number of copies you must also follow the conditions in section 3.

You may also lend copies, under the same conditions stated above, and
you may publicly display copies.

\section{3. Copying in Quantity}

If you publish printed copies (or copies in media that commonly have
printed covers) of the Document, numbering more than 100, and the
Document's license notice requires Cover Texts, you must enclose the
copies in covers that carry, clearly and legibly, all these Cover
Texts: Front-Cover Texts on the front cover, and Back-Cover Texts on
the back cover.  Both covers must also clearly and legibly identify
you as the publisher of these copies.  The front cover must present
the full title with all words of the title equally prominent and
visible.  You may add other material on the covers in addition.
Copying with changes limited to the covers, as long as they preserve
the title of the Document and satisfy these conditions, can be treated
as verbatim copying in other respects.

If the required texts for either cover are too voluminous to fit
legibly, you should put the first ones listed (as many as fit
reasonably) on the actual cover, and continue the rest onto adjacent
pages.

If you publish or distribute Opaque copies of the Document numbering
more than 100, you must either include a machine-readable Transparent
copy along with each Opaque copy, or state in or with each Opaque copy
a computer-network location from which the general network-using
public has access to download using public-standard network protocols
a complete Transparent copy of the Document, free of added material.
If you use the latter option, you must take reasonably prudent steps,
when you begin distribution of Opaque copies in quantity, to ensure
that this Transparent copy will remain thus accessible at the stated
location until at least one year after the last time you distribute an
Opaque copy (directly or through your agents or retailers) of that
edition to the public.

It is requested, but not required, that you contact the authors of the
Document well before redistributing any large number of copies, to give
them a chance to provide you with an updated version of the Document.

\section{4. Modifications}

You may copy and distribute a Modified Version of the Document under
the conditions of sections 2 and 3 above, provided that you release
the Modified Version under precisely this License, with the Modified
Version filling the role of the Document, thus licensing distribution
and modification of the Modified Version to whoever possesses a copy
of it.  In addition, you must do these things in the Modified Version:

\begin{itemize}
\item[A.] 
   Use in the Title Page (and on the covers, if any) a title distinct
   from that of the Document, and from those of previous versions
   (which should, if there were any, be listed in the History section
   of the Document).  You may use the same title as a previous version
   if the original publisher of that version gives permission.
   
\item[B.]
   List on the Title Page, as authors, one or more persons or entities
   responsible for authorship of the modifications in the Modified
   Version, together with at least five of the principal authors of the
   Document (all of its principal authors, if it has fewer than five),
   unless they release you from this requirement.
   
\item[C.]
   State on the Title page the name of the publisher of the
   Modified Version, as the publisher.
   
\item[D.]
   Preserve all the copyright notices of the Document.
   
\item[E.]
   Add an appropriate copyright notice for your modifications
   adjacent to the other copyright notices.
   
\item[F.]
   Include, immediately after the copyright notices, a license notice
   giving the public permission to use the Modified Version under the
   terms of this License, in the form shown in the Addendum below.
   
\item[G.]
   Preserve in that license notice the full lists of Invariant Sections
   and required Cover Texts given in the Document's license notice.
   
\item[H.]
   Include an unaltered copy of this License.
   
\item[I.]
   Preserve the section Entitled ``History", Preserve its Title, and add
   to it an item stating at least the title, year, new authors, and
   publisher of the Modified Version as given on the Title Page.  If
   there is no section Entitled ``History" in the Document, create one
   stating the title, year, authors, and publisher of the Document as
   given on its Title Page, then add an item describing the Modified
   Version as stated in the previous sentence.
   
\item[J.]
   Preserve the network location, if any, given in the Document for
   public access to a Transparent copy of the Document, and likewise
   the network locations given in the Document for previous versions
   it was based on.  These may be placed in the ``History" section.
   You may omit a network location for a work that was published at
   least four years before the Document itself, or if the original
   publisher of the version it refers to gives permission.
   
\item[K.]
   For any section Entitled ``Acknowledgements" or ``Dedications",
   Preserve the Title of the section, and preserve in the section all
   the substance and tone of each of the contributor acknowledgements
   and/or dedications given therein.
   
\item[L.]
   Preserve all the Invariant Sections of the Document,
   unaltered in their text and in their titles.  Section numbers
   or the equivalent are not considered part of the section titles.
   
\item[M.]
   Delete any section Entitled ``Endorsements".  Such a section
   may not be included in the Modified Version.
   
\item[N.]
   Do not retitle any existing section to be Entitled ``Endorsements"
   or to conflict in title with any Invariant Section.
   
\item[O.]
   Preserve any Warranty Disclaimers.
\end{itemize}

If the Modified Version includes new front-matter sections or
appendices that qualify as Secondary Sections and contain no material
copied from the Document, you may at your option designate some or all
of these sections as invariant.  To do this, add their titles to the
list of Invariant Sections in the Modified Version's license notice.
These titles must be distinct from any other section titles.

You may add a section Entitled ``Endorsements", provided it contains
nothing but endorsements of your Modified Version by various
parties--for example, statements of peer review or that the text has
been approved by an organization as the authoritative definition of a
standard.

You may add a passage of up to five words as a Front-Cover Text, and a
passage of up to 25 words as a Back-Cover Text, to the end of the list
of Cover Texts in the Modified Version.  Only one passage of
Front-Cover Text and one of Back-Cover Text may be added by (or
through arrangements made by) any one entity.  If the Document already
includes a cover text for the same cover, previously added by you or
by arrangement made by the same entity you are acting on behalf of,
you may not add another; but you may replace the old one, on explicit
permission from the previous publisher that added the old one.

The author(s) and publisher(s) of the Document do not by this License
give permission to use their names for publicity for or to assert or
imply endorsement of any Modified Version.

\section{5. Combining Documents}


You may combine the Document with other documents released under this
License, under the terms defined in section 4 above for modified
versions, provided that you include in the combination all of the
Invariant Sections of all of the original documents, unmodified, and
list them all as Invariant Sections of your combined work in its
license notice, and that you preserve all their Warranty Disclaimers.

The combined work need only contain one copy of this License, and
multiple identical Invariant Sections may be replaced with a single
copy.  If there are multiple Invariant Sections with the same name but
different contents, make the title of each such section unique by
adding at the end of it, in parentheses, the name of the original
author or publisher of that section if known, or else a unique number.
Make the same adjustment to the section titles in the list of
Invariant Sections in the license notice of the combined work.

In the combination, you must combine any sections Entitled ``History"
in the various original documents, forming one section Entitled
``History"; likewise combine any sections Entitled ``Acknowledgements",
and any sections Entitled ``Dedications".  You must delete all sections
Entitled ``Endorsements".

\section{6. Collections of Documents}

You may make a collection consisting of the Document and other documents
released under this License, and replace the individual copies of this
License in the various documents with a single copy that is included in
the collection, provided that you follow the rules of this License for
verbatim copying of each of the documents in all other respects.

You may extract a single document from such a collection, and distribute
it individually under this License, provided you insert a copy of this
License into the extracted document, and follow this License in all
other respects regarding verbatim copying of that document.

\section{7. Aggregation with Independent Works}


A compilation of the Document or its derivatives with other separate
and independent documents or works, in or on a volume of a storage or
distribution medium, is called an ``aggregate" if the copyright
resulting from the compilation is not used to limit the legal rights
of the compilation's users beyond what the individual works permit.
When the Document is included in an aggregate, this License does not
apply to the other works in the aggregate which are not themselves
derivative works of the Document.

If the Cover Text requirement of section 3 is applicable to these
copies of the Document, then if the Document is less than one half of
the entire aggregate, the Document's Cover Texts may be placed on
covers that bracket the Document within the aggregate, or the
electronic equivalent of covers if the Document is in electronic form.
Otherwise they must appear on printed covers that bracket the whole
aggregate.

\section{8. Translation}


Translation is considered a kind of modification, so you may
distribute translations of the Document under the terms of section 4.
Replacing Invariant Sections with translations requires special
permission from their copyright holders, but you may include
translations of some or all Invariant Sections in addition to the
original versions of these Invariant Sections.  You may include a
translation of this License, and all the license notices in the
Document, and any Warranty Disclaimers, provided that you also include
the original English version of this License and the original versions
of those notices and disclaimers.  In case of a disagreement between
the translation and the original version of this License or a notice
or disclaimer, the original version will prevail.

If a section in the Document is Entitled ``Acknowledgements",
``Dedications", or ``History", the requirement (section 4) to Preserve
its Title (section 1) will typically require changing the actual
title.

\section{9. Termination}


You may not copy, modify, sublicense, or distribute the Document except
as expressly provided for under this License.  Any other attempt to
copy, modify, sublicense or distribute the Document is void, and will
automatically terminate your rights under this License.  However,
parties who have received copies, or rights, from you under this
License will not have their licenses terminated so long as such
parties remain in full compliance.

\section{10. Future Revisions of This License}


The Free Software Foundation may publish new, revised versions
of the GNU Free Documentation License from time to time.  Such new
versions will be similar in spirit to the present version, but may
differ in detail to address new problems or concerns.  See
\texttt{http://www.gnu.org/copyleft/}.

Each version of the License is given a distinguishing version number.
If the Document specifies that a particular numbered version of this
License ``or any later version" applies to it, you have the option of
following the terms and conditions either of that specified version or
of any later version that has been published (not as a draft) by the
Free Software Foundation.  If the Document does not specify a version
number of this License, you may choose any version ever published (not
as a draft) by the Free Software Foundation.

\section{Addendum: How to use this License for your documents}

To use this License in a document you have written, include a copy of
the License in the document and put the following copyright and
license notices just after the title page:

\bigskip
\begin{quote}
    Copyright \copyright  YEAR  YOUR NAME.
    Permission is granted to copy, distribute and/or modify this document
    under the terms of the GNU Free Documentation License, Version 1.2
    or any later version published by the Free Software Foundation;
    with no Invariant Sections, no Front-Cover Texts, and no Back-Cover Texts.
    A copy of the license is included in the section entitled ``GNU
    Free Documentation License".
\end{quote}
\bigskip
    
If you have Invariant Sections, Front-Cover Texts and Back-Cover Texts,
replace the ``with\ldots Texts." line with this:

\bigskip
\begin{quote}
    with the Invariant Sections being LIST THEIR TITLES, with the
    Front-Cover Texts being LIST, and with the Back-Cover Texts being LIST.
\end{quote}
\bigskip
    
If you have Invariant Sections without Cover Texts, or some other
combination of the three, merge those two alternatives to suit the
situation.

If your document contains nontrivial examples of program code, we
recommend releasing these examples in parallel under your choice of
free software license, such as the GNU General Public License,
to permit their use in free software.

%---------------------------------------------------------------------


\end{document}
